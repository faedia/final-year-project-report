\chapter{State of the Art}
\label{chap:sota}
The idea of generating code to be a part of the compiler pipeline is not new.
The first two stages of the front-end of a compiler, the lexical analyser and the syntax analyser (more commonly known as the lexer and the parser), have general tools to generate their code using Domain Specific Languages (DSL)\cite{Bentley:1986:PPL:6424.315691,van2000domain}.
Two such of these languages are Flex\cite{Levine:2009:FB:1696439}, as the lexer generator, and Bison\cite{Levine:2009:FB:1696439}, as the parser generator.
These tools generate C code to be used to build the rest of a compiler, possibly including a type checker and machine code generation.
These programs have been around for many years, and many derivatives of them have been made to be used with different general purpose programming languages, such as Alex and Happy for Haskell or Jlex and Cup for Java\cite{ranta2012implementing}.
Many languages have been made using these such programs and their respective DSL's, for example the parser for Haskell is self-generated by a grammar defined in a piece of Happy code.
However, not all parts of the compiler pipeline have general tools that are widely used.
Although, some such tools do exist\cite{grimm2007typical,dijkstra2006ruler,Gray:1992:ECF:129630.129637}.
\section{Type Systems}
Before we can start discussing type checkers, we need to discuss the type systems which the checkers implement.

The concept and implementation of type systems has existed for almost as long as the concept of higher level language to perform computation\cite{Backus:1978:HFI:960118.808380}.
Many languages have a type system in one form or another with varying degrees of complexity,
ranging from relatively simple type systems in languages such as C to fairly complex and expressive type systems such as those in Haskell and Agda.

A type systems is a specification of a given programming language's type rules\cite{cardelli1996type}.
So from this we know that a type system is a specification and we can say that a type checker is an implementation of the specification.
Cardelli goes on to say that, we should separate the two, and that a type system is part of the language specification and the type checker is part of the compiler's implementation of the language\cite{cardelli1996type}.
This is similar to the how we view the formal grammar is defined by the language, and the parser is how the compiler implements that formal grammar\cite{cardelli1996type}.

\subsection{Usefulness of Type Systems}
\todo{Need to research}


\subsection{Type Rules}

Much like how we have a formal representation for context free grammars as Backus-Naur form\cite{Backus1960,aho2003compilers,ranta2012implementing}, we also have a formal way of representing type systems and type rules.
In the literature, the standard method we have of representing type systems and type rules is through the use of natural deduction rules\cite{cardelli1996type,ranta2012implementing}.

Natural deduction rules have a form of: given some premises we conclude the consequent\cite{prawitz2006natural,ranta2012implementing}, for example in \autoref{fig:generalNatDectRule} the premises are denoted as $J_k$ and the consequent is denoted as $C$.

\begin{figure}[htbp]
    \begin{prooftree}
        \LeftLabel{Rule Label:\quad}
        \RightLabel{$[Side\ Conditions]$}
        \AxiomC{$J_1\ J_2\ \ ..\ \ J_n$}
        \UnaryInfC{$C$}
    \end{prooftree}
    \caption{General form of a natural deduction rule}
    \label{fig:generalNatDectRule}
\end{figure}

We can then use these Natural Deduction rules to specify the type rules in a given type system\cite{ranta2012implementing,cardelli1996type,}.
\autoref{fig:simpleTypeRule} is an example of how we would write a simple type rule for an arithmetic expression using the natural deduction logic.
It is important to break this rule into its constituent parts to understand what it is trying to specify.
Starting from the consequent $\Gamma \vdash e_1 + e_2 : Int$, $\Gamma$ is the current context which we are applying the rule to.
The $\vdash$ is piece of syntax to say given the left hand side we can prove the right hand side.
Finally for the consequent $e_1 + e_2 : Int$ means that the piece of concrete syntax $e_1 + e_2$ must have type $Int$.
The top half of the rule are the things that must hold for the bottom half to be valid.\
So simply $e_1$ must be of type $Int$ given the context $\Gamma$, and the same is for the second premise\cite{cardelli1996type,ranta2012implementing}.

\begin{figure}[htbp]
    \begin{prooftree}
        \LeftLabel{Add Expression:\quad}
        \AxiomC{$\Gamma \vdash e_1 : Int$}
        \AxiomC{$\Gamma \vdash e_2 : Int$}
        \BinaryInfC{$\Gamma \vdash e_1 + e_2 : Int$}
    \end{prooftree}
    \caption{Simple arithmetic expression type rule}
    \label{fig:simpleTypeRule}
\end{figure}
%\input{chapters/stateOfTheArt/TypeCheckers.tex}
\section{Type Checker Generators}
There have been multiple attempts to make the writing of type checkers easier on the compiler writers by creating new tools or notations.
They can vary widely in how the express type systems and what notations they use.
Some use the notation of attribute grammars to represent type systems, along with the rest of the compiler, where as other use more specialised Domain Specific Languages to represent their type rules and type checkers.

\subsection{Attribute Grammars}
Attribute Grammars are a way of specifying the semantics of grammar rules for a Context Free Grammar.
This makes them seem a likely candidate to look at when attempting to generate type checking code since type checking is a form of semantic analysis on some input code.
In Knuth's paper\cite{Knuth1968}, he gives an example of the attributes required to evaluate a simple language to emulate a Turing Machine.
Even though he discusses the evaluation of a turing machine it is plain to see that on each node of the tree an attribute could be given to it that defines the type of the node.

\subsection{Eli}
\begin{figure}[]
    \centering
    \begin{prooftree}
        \AxiomC{$\Gamma \vdash e_1 : \tau$}
        \AxiomC{$\Gamma \vdash e_2 : \tau$}
        \RightLabel{$[\tau \in \{Bool, Int\}]$}
        \BinaryInfC{$\Gamma \vdash e_1 = e_2 : Bool$}
    \end{prooftree}
    \caption{Type rule for equality for the example language for Eli}
    \label{fig:oilNatRule}
\end{figure}
Eli is a tool which, given a set of specifications, will generate code for a whole compiler based upon those specifications\cite{Gray:1992:ECF:129630.129637}.
The way Eli generates all the code is by the use of an attribute grammar system and numerous internal tools that generate code for each different part of the compiler pipeline.
The internal tools use DSL's which define a specification for part of the language.
One of these DSL's is called \textit{Operator Identification Language} (OIL), which is used to define the typing relations on operators.
The definition of these type rules is fairly simple, however in their example there are a lot of repeated definitions for operators, for example there are multiple definitions to define the equality operator, one for each of the required types that the operator can act on.
Where as in the natural deduction rules this would only have to be defined once where the type is some arbitrary type $\tau$  with some constraint on $\tau$ such that it can only be from a set of types that an equality check can be performed on, see the natural deduction rule given in \autoref{fig:oilNatRule}.
The tools lacks built-in support for constructs in the language that can have arbitrary types.
However, OIL is a small and simple language that is easy to understand, but, the other constituent parts of ELI required to build a tool are not as simple and the specifications written in them can get much more complicated for larger projects.

\subsection{Ruler}
Ruler is a DSL that bears close resembles to natural deduction rules that formally define type systems\cite{dijkstra2006ruler}.
The notation used to describe the natural deduction rules is intuitive to anyone who has experience with such notation.
However, other possible compiler writers may have never come across the notation meaning the syntax of Ruler could be un-intuitive to such compiler writers.
The rules can be compiled to either \LaTeX, or to an attribute grammar system in Haskell.
However this means that rules have to be written multiple times for each target that the user may want to output to.
This creates unnecessary duplication of types rules when so that they can be compiled to executable code or to a formal specification given in latex.
Since the only executable code compiles to an attribute grammar, it forces the user to use the attribute grammar for the whole frontend of their compiler, where as using the abstract syntax allows the user to choose any parser generator (or write their own parser) as long they provide the abstract syntax to the tool.
They also make a claim that they support Hindley-Milner type inference, while this to some extent is true, it requires the user to write a rather large amount of Haskell code in order for the tool to generate correct code for Hindley-Milner type inference.
This means that a large part of the type checker still has to be written by hand because many non-trivial type rules will require access to the context or types to be unified for which code is not generated for.

\subsection{Typical}
\label{sec:typical}
The final tool we will be discussing is Typical\cite{grimm2007typical}.
Typical is a language that uses ML\cite{milner1997definition} as the base of its language this means that all terms in Typical are typed and such as a proof of concept of Typical, the type checker for Typical is built using Typical.
Typical also has some extra declarations that are not in ML, these are the ability to define name-spaces for the context, and scoping rules, which in the tools we have discusses so far the users have had to write such features of their type system.
They also have in built functions that retrieve types from the context and also add new items to a context, which is again features that the previous tools did not have.
Unlike our previous examples that use attribute grammars to perform their type checking, Typical uses the abstract syntax tree given by the parser.
This means that Typical can be used with any parser generator that targets Typical's target language (Java), although they do admit that some modification may be required.
They have performed the required modification on the Rats! parser generator\cite{Grimm:2006:BET:1133255.1133987}.
To understand a type system written in Typical you will need to be familiar with ML or another language similar to is such as Haskell since the vast majority of the code in Typical is ML.
