\section{Type Systems}
The concept and implementation of type systems has existed for almost as long as the concept of higher level language to perform computation\cite{Backus:1978:HFI:960118.808380}.
Many languages have a type system in one form or another with varying degrees of complexity,
ranging from languages with no type system, such as Scheme or JavaScript, to relatively simple type systems in languages such as C, and finally fairly complex and expressive type systems such as those in Haskell and Agda.

"A type system specifies the type rules of a programming language independently of particular type checking algorithms"\cite{cardelli1996type}, where type rules define under what circumstances a code fragment can be considered type correct.
This is an important distinction as it states that a type system is not how to type check the language, but instead states that the type checker should confirm to the type rules specified by the type system.
Therefore it is the type system that defines the behaviour of the type checker.  

\subsection{Usefulness of Type Systems}
There are many arguments as to the usefulness of type systems and whether or not languages should have static type systems.
Cardelli makes a number of as to whether languages should be typed from an engineering point of view\cite{cardelli1996type}.
However, an argument can be made that software could be made safer with the user of type systems.
One of the most famous software bugs in history is that of the Ariane 5 resulting in the loss of the spacecraft after 39 seconds of flight.
The bug that caused the loss of the Ariane 5 was in a function that converted a 64-bit floating point number to a 16-bit integer number.
The bug was created when the function tried to convert the floating point number into a value that was larger than could be represented by a 16-bit integer\cite{lions1996ariane}.
A more sophisticated type system could have possibly identified the differences in what the two data types could represent and reported an error to the developers during compilation to tell them about the possibility of an unsafe conversion.
In this situation the bug could have been identified and fixed before the launch of the spacecraft.

\subsection{Type Rules}
Much like how we have a formal representation for context free grammars as Backus-Naur form\cite{Backus1960,aho2003compilers,ranta2012implementing}, we also have a formal way of representing type systems and type rules.
In the literature, the standard method we have of representing type systems and type rules is through the use of natural deduction rules\cite{cardelli1996type,ranta2012implementing}.

Natural deduction rules have a form of: given some premises we conclude the consequent\cite{prawitz2006natural,ranta2012implementing}, for example in \autoref{fig:generalNatDectRule} the premises are denoted as $J_k$ and the consequent is denoted as $C$.

\begin{figure}[tbp]
    \begin{prooftree}
        \LeftLabel{Rule Label:\quad}
        \RightLabel{$[Side\ Conditions]$}
        \AxiomC{$J_1\ J_2\ \ ..\ \ J_n$}
        \UnaryInfC{$C$}
    \end{prooftree}
    \caption{General form of a natural deduction rule}
    \label{fig:generalNatDectRule}
\end{figure}

We can then use these Natural Deduction rules to specify the type rules in a given type system\cite{ranta2012implementing,cardelli1996type,}.
\autoref{fig:simpleTypeRule} is an example of how we would write a simple type rule for an arithmetic expression using the natural deduction logic.
It is important to break this rule into its constituent parts to understand what it is trying to specify.
Starting from the consequent $\Gamma \vdash e_1 + e_2 : Int$, $\Gamma$ is the current context which we are applying the rule to.
The $\vdash$ is piece of syntax to say given the left hand side we can prove the right hand side.
Finally for the consequent $e_1 + e_2 : Int$ means that the piece of concrete syntax $e_1 + e_2$ must have type $Int$.
The top half of the rule are the things that must hold for the bottom half to be valid.\
So simply $e_1$ must be of type $Int$ given the context $\Gamma$, and the same is for the second premise\cite{cardelli1996type,ranta2012implementing}.

\begin{figure}[tbp]
    \begin{prooftree}
        \LeftLabel{Add Expression:\quad}
        \AxiomC{$\Gamma \vdash e_1 : Int$}
        \AxiomC{$\Gamma \vdash e_2 : Int$}
        \BinaryInfC{$\Gamma \vdash e_1 + e_2 : Int$}
    \end{prooftree}
    \caption{Simple arithmetic expression type rule}
    \label{fig:simpleTypeRule}
\end{figure}