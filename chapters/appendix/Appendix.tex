
\appendix
\appendixpage
\chapter{JET Language}
\label{appendix:jetLanguage}
\section{Grammar}
\begin{grammar}
    <TypeSystem> ::= <InlineHaskell> <TypeRuleList>
    
    <InlineHaskell> ::= `{' <Code> `}'

    <TypeRuleList> ::= <TypeRule> \alt <TypeRule> `;' <TypeRuleList>

    <TypeRuleList> ::= `typerule' <Identifier> `<-' [`if' <TypePremiseList> `then'] <TypeConsequent> `return' <InlineHaskell>

    <TypeConsequent> ::= `(' <Identifier> <IdentifierList> `)' [`:' <Identifier> <IdentifierList>]
        \alt `[' <Identifier> `]' [: <Identifier> <IdentifierList>]
        \alt `[' <Identifier> <Identifier> `]' [: <Identifier> <IdentifierList>]
        \alt `[' <Identifier> <Identifier> `]' [: <Identifier> <IdentifierList>]

    <TypePremiseList> ::= <TypePremise> \alt <TypePremise> `,' <TypePremiseList>

    <TypePremise> ::= <InlineHaskell>
        \alt `(' <Identifier> <Identifier> `)' [`:' <Identifier> <IdentifierList>]
        \alt `[' <Identifier> <Identifier> `]' [`:' <Identifier> <IdentifierList>]
\end{grammar}
\section{Inline Haskell Example}
\begin{lstlisting}[caption = Example of initial inline haskell code, label=lst:inlineHaskellCode, language=Haskell]
{
import qualified Data.Map as M
import AbsAd

data ContextKey = Ret | Func Ident [Type] | Var Ident
    deriving (Eq, Ord, Show, Read)

newtype Context = Context (Type, JetContextMap (Ident, [Type]) Type, JetContextMap Ident Type)
    deriving (Eq, Ord, Show, Read)

instance JetContextBase Context where
    emptyContext = Context (TNNone, emptyContext, emptyContext)
    newBlock (Context (t, f, v) = Context (t, newBlock f, newBlock v)

instance JetContext Context ContextKey Type where
    lookupContext Ret (Context (r, _, _, _, _)) = return r
    lookupContext (Func ident ts) (Context (_, _, f, _, _)) = lookupContext (ident, ts) f
    lookupContext (Var ident) (Context (_, _, _, v, _)) = lookupContext ident v
    expandContext Ret t (Context (_, p, f, v, s)) = return (Context (t, p, f, v, s))
    expandContext (Func ident ts) t (Context (rt, p, f, v, s)) = do
        f' <- expandContextIf (\(i, t) (JetContextMap (b:bs)) -> null (filter (\(i', t') -> i == i') (M.keys b))) (ident, ts) t f
        return (Context (rt, p, f', v, s))
    expandContext (Var ident) t (Context (rt, p, f, v, s)) = do
        v' <- expandContext ident t v
        return (Context (rt, p, f, v', s))

makeCheckError e t1 t2 = "Type error"
makeCheckErrorList _ _ _ = "Type error"
makeInferError _ _ = "Type error"
}
\end{lstlisting}

\chapter{Example Languages}
\label{appendix:witnessLanguages}
\section{Ad Language}
\subsection{AST}
\begin{lstlisting}[language=Haskell]
newtype Ident = Ident String deriving (Eq, Ord, Show, Read)
data Program = Prog [TypeDecl] ProcDecl

data TypeDecl = TDecl Ident TypeName type name = t

data FuncDecl = FDecl Ident [VarDecl] TypeName [Decl] [Stmnt] -- function name(params) return t decls being stmnts end

data ProcDecl = PDecl Ident [VarDecl] [Decl] [Stmnt] -- procedure name(params) decls being stmnts end 

data VarDecl = VDecl Ident TypeName -- name : t

data Decl = DFunc FuncDecl | DProc ProcDecl | DVar VarDecl

data Stmnt
    = SNull -- null
    | SPrint Expr -- print e
    | SAssign [LValue] Expr -- name := e
    | SIf Expr [Stmnt] [Stmnt] -- if e then stmnts1 else stmnts2 end
    | SWhile Expr [Stmnt] -- while e do stmnts end
    | SBlock [Decl] [Stmnt] -- declare decls begin stmnts end 
    | SCall Ident [Expr] -- name(params)
    | SReturn Expr -- return e

data Expr
    = EInt Integer
    | ETrue -- false
    | EFalse -- true
    | EVar [LValue] -- name
    | ECall Ident [Expr] -- name(params)
    | ENeg Expr -- neg e
    | ENot Expr -- not e
    | EMul Expr Expr -- e1 * e2
    | EAnd Expr Expr -- e1 and e2
    | EAdd Expr Expr -- e1 + e2
    | EOr Expr Expr -- e1 or e2
    | EEq Expr Expr -- e1 = e2
    | ELT Expr Expr -- e1 < e2

data LValue = LVVar Ident | LVArr LValue Expr -- name | name[e]

data TypeName
    = TNBool -- bool
    | TNInt -- int
    | TNSyn Ident -- name
    | TNArr TypeName -- array t
    | TNRec [VarDecl] -- record decls end
    | TNNone -- *internal*   
\end{lstlisting}

\subsection{Type Rules}
The definition of $\Gamma$ for the Ad language is a 5-tuple of the form $(\tau_{ret}, P, F, V, T)$, where $\tau_{ret}$ is the return type of the current subprogram, $P$ is the name-space of procedures, $F$ is the name-space for functions, $V$ is the name space for variables, $T$ is the name-space for types.
The name-spaces $P$, $F$, $V$ are block structured whereas $T$ is not, therefore $newblock(\Gamma)$ is really $(\tau_{ret}, newblock(P),newblock(F), newblock(V), T)$
\subsubsection{Statements}
\begin{figure}[H]
    \centering
    \begin{center}
        \LeftLabel{AdTDeclT}
        \AxiomC{$expandContext(v, \tau, T) \vdash \lfloor decl \rfloor\ valid$}
        \UnaryInfC{$\Gamma \vdash \lfloor$ type $ v = \tau $; $decl \rfloor\ valid$}
        \DisplayProof{}
        \label{fig:adnullt}

        \ \\\ \\
        \LeftLabel{AdVDecl}
        \AxiomC{$expandContext(v, \tau, V) \vdash \lfloor decl \rfloor\ valid$}
        \UnaryInfC{$\Gamma \vdash \lfloor$ type $ v : \tau $; $decl \rfloor\ valid$}
        \DisplayProof{}
        \label{fig:adnullt}

        \ \\\ \\
        \LeftLabel{AdVDecl}
        \AxiomC{$\Gamma' \vdash \lfloor block \rfloor\ valid$}
        \AxiomC{$\Gamma' \vdash \lfloor decl \rfloor\ valid$}
        \RightLabel{$[\Gamma' = expandContext((v, p), P)]$}
        \BinaryInfC{$\Gamma \vdash \lfloor$ procedure $ v (p) block $; $decl \rfloor\ valid$}
        \DisplayProof{}

        \ \\\ \\
        \LeftLabel{AdVDecl}
        \AxiomC{$\Gamma', \tau_{ret} = \tau \vdash \lfloor block \rfloor\ valid$}
        \AxiomC{$\Gamma' \vdash \lfloor decl \rfloor\ valid$}
        \RightLabel{$[\Gamma' = expandContext((v, p), F)]$}
        \BinaryInfC{$\Gamma \vdash \lfloor$ function $ v (p)$ return $\tau\ block $; $decl \rfloor\ valid$}
        \DisplayProof{}
        \label{fig:adnullt}
    \end{center}
    \caption{Type rules for declarations in the Ad language}
    \label{}
\end{figure}
\begin{figure}[H]
    \begin{center}
        \LeftLabel{AdNullT}
        \AxiomC{}
        \UnaryInfC{$\Gamma \vdash \lfloor$ null $\rfloor\ valid$}
        \DisplayProof{}
        \label{fig:adnullt}
        \ 
        \LeftLabel{AdPrintT}
        \AxiomC{$\Gamma \vdash e : \tau$}
        \UnaryInfC{$\Gamma \vdash \lfloor$ print $e\rfloor\ valid$}
        \DisplayProof{}
        \label{fig:adprintt}
        
        \ \\
        \LeftLabel{AdAssignT: }
        \AxiomC{$\Gamma \vdash e_l : t$}
        \AxiomC{$\Gamma \vdash e_r : t$}
        \RightLabel{$[e_l \in \{e[e_{int}], e_{rec}.e, v\}]$}
        \BinaryInfC{$\Gamma \vdash \lfloor e_l := e_r \rfloor\ valid$}
        \DisplayProof{}
        
        \ \\\ \\
        \LeftLabel{AdIfT: }
        \AxiomC{$\Gamma \vdash e : Bool$}
        \AxiomC{$\Gamma \vdash \lfloor s_1 \rfloor\ valid$}
        \AxiomC{$\Gamma \vdash \lfloor s_2 \rfloor\ valid$}
        \TrinaryInfC{$\Gamma \vdash \lfloor$ if $e$ then $s_1$ else $s_2$ end $\rfloor\ valid$}
        \DisplayProof{}

        \ \\\ \\
        \LeftLabel{AdWhileT: }
        \AxiomC{$\Gamma \vdash e : Bool$}
        \AxiomC{$\Gamma \vdash \lfloor s \rfloor\ valid$}
        \BinaryInfC{$\Gamma \vdash \lfloor$ while $e$ do $s$ end $\rfloor\ valid$}
        \DisplayProof{}

        \ \\\ \\
        \LeftLabel{AdBlockT: }
        \AxiomC{$newBlock(\Gamma) \vdash \lfloor d \rfloor\ valid$}
        \AxiomC{$\Gamma' \vdash \lfloor s \rfloor\ valid$}
        \RightLabel{$\Gamma' = \Gamma \oplus d$}
        \BinaryInfC{$\Gamma \vdash \lfloor$ declare $d$ begin $s$ end $\rfloor\ valid$}
        \DisplayProof{}

        \ \\\ \\
        \LeftLabel{AdPCallT: }
        \AxiomC{}
        \RightLabel{$[(v, params) \in P)]$}
        \UnaryInfC{$\Gamma \vdash \lfloor v(params) \rfloor\ valid$}
        \DisplayProof{}

        \ \\\ \\
        \LeftLabel{AdRetT: }
        \AxiomC{$\Gamma \vdash e : \tau_{ret}$}
        \UnaryInfC{$\Gamma \vdash \lfloor$ return $e \rfloor\ valid$}
        \DisplayProof{}
    \end{center}
    \caption{Type rules for statements in the Ad language}
    \label{fig:adStmnts}
\end{figure}

\subsubsection{Expressions}
\begin{figure}[H]
    \begin{center}
        \LeftLabel{AdTrueT}
        \AxiomC{}
        \UnaryInfC{$\Gamma \vdash true : Bool$}
        \DisplayProof{}
        \label{fig:adtruet}
        \ 
        \LeftLabel{AdFalseT}
        \AxiomC{}
        \UnaryInfC{$\Gamma \vdash false : Bool$}
        \DisplayProof{}
        \label{fig:adfalset}

        \ \\
        \LeftLabel{AdInt}
        \AxiomC{}
        \UnaryInfC{$\Gamma \vdash n : Int$}
        \DisplayProof{}
        \label{fig:adintt}
        \ 
        \LeftLabel{AdNot}
        \AxiomC{$\Gamma \vdash e : Bool$}
        \UnaryInfC{$\Gamma \vdash not\ e : Bool$}
        \DisplayProof{}
        \ 
        \LeftLabel{AdNeg}
        \AxiomC{$\Gamma \vdash e : Int$}
        \UnaryInfC{$\Gamma \vdash neg\ e : Int$}
        \DisplayProof{}

        \ \\
        \LeftLabel{AdAnd}
        \AxiomC{$\Gamma \vdash e_1 : Bool$}
        \AxiomC{$\Gamma \vdash e_2 : Bool$}
        \BinaryInfC{$\Gamma \vdash e_1\ and\ e_2 : Bool$}
        \DisplayProof{}

        \ \\
        \LeftLabel{AdMul}
        \AxiomC{$\Gamma \vdash e_1 : Int$}
        \AxiomC{$\Gamma \vdash e_2 : Int$}
        \BinaryInfC{$\Gamma \vdash e_1 * e_2 : Int$}
        \DisplayProof{}

        \ \\
        \LeftLabel{AdOr}
        \AxiomC{$\Gamma \vdash e_1 : Bool$}
        \AxiomC{$\Gamma \vdash e_2 : Bool$}
        \BinaryInfC{$\Gamma \vdash e_1\ or\ e_2 : Bool$}
        \DisplayProof{}

        \ \\
        \LeftLabel{AdAdd}
        \AxiomC{$\Gamma \vdash e_1 : Int$}
        \AxiomC{$\Gamma \vdash e_2 : Int$}
        \BinaryInfC{$\Gamma \vdash e_1 + e_2 : Int$}
        \DisplayProof{}

        \ \\
        \LeftLabel{AdEq}
        \AxiomC{$\Gamma \vdash e_1 : \tau$}
        \AxiomC{$\Gamma \vdash e_2 : \tau$}
        \BinaryInfC{$\Gamma \vdash e_1 = e_2 : Bool$}
        \DisplayProof{}
        \ 
        \LeftLabel{AdLT}
        \AxiomC{$\Gamma \vdash e_1 : Int$}
        \AxiomC{$\Gamma \vdash e_2 : Int$}
        \BinaryInfC{$\Gamma \vdash e_1 < e_2 : Bool$}
        \DisplayProof{}

        \ \\
        \LeftLabel{AdFCallT: }
        \AxiomC{}
        \RightLabel{$[\tau = lookup((v, params), F)]$}
        \UnaryInfC{$\Gamma \vdash v(params) : \tau$}
        \DisplayProof{}

        \ \\
        \LeftLabel{AdArrAccT: }
        \AxiomC{$\Gamma \vdash e : Arr\ \tau$}
        \AxiomC{$\Gamma \vdash e_{int} : Int$}
        \BinaryInfC{$\Gamma \vdash e[e_{int}] : \tau$}
        \DisplayProof{}
        
        \ \\
        \LeftLabel{AdRecAccT: }
        \AxiomC{$\Gamma \vdash e : Rec\ D$}
        \RightLabel{$[\tau = lookup(v, D)]$}
        \UnaryInfC{$\Gamma \vdash e_1.v : \tau$}
        \DisplayProof{}
        
    \end{center}
    \label{fig:adExprs}
    \caption{Type rules for expressions in the Ad language}
\end{figure}

\subsubsection{Type Synonyms}
\begin{figure}[h]
    \centering
    \begin{center}
        \ \\
        \LeftLabel{AdSynT: }
        \AxiomC{$\Gamma \vdash \tau: Syn\ v$}
        \RightLabel{$[\sigma = lookup(v, T)]$}
        \UnaryInfC{$\Gamma \vdash \tau : \sigma$}
        \DisplayProof{}
        ~
        \LeftLabel{AdT: }
        \AxiomC{}
        \UnaryInfC{$\Gamma \vdash \tau : \tau$}
        \DisplayProof{}
    \end{center}
    \caption{Type rule for type synonyms}
    \label{}
\end{figure}

\subsection{Ad JET Specification}
\begin{lstlisting}[language=Haskell]
{
import qualified Data.Map as M
import AbsAd

type Type = TypeName

data ContextKey = Ret | Proc Ident [Type] | Func Ident [Type] | Var Ident | Typ Ident
    deriving (Eq, Ord, Show, Read)

newtype Context = Context (Type, JetContextMap (Ident, [Type]) Type, JetContextMap (Ident, [Type]) Type, JetContextMap Ident Type, JetContextMap Ident Type)
    deriving (Eq, Ord, Show, Read)

instance JetContextBase Context where
    emptyContext = Context (TNNone, emptyContext, emptyContext, emptyContext, emptyContext)
    newBlock (Context (t, p, f, v, s)) = Context (t, newBlock p, newBlock f, newBlock v, s)

instance JetContext Context ContextKey Type where
    lookupContext Ret (Context (t, _, _, _, _)) = Succ t
    lookupContext (Proc ident ts) (Context (_, p, _, _, _)) = lookupContext (ident, ts) p
    lookupContext (Func ident ts) (Context (_, _, f, _, _)) = lookupContext (ident, ts) f
    lookupContext (Var ident) (Context (_, _, _, v, _)) = lookupContext ident v
    lookupContext (Typ ident) (Context (_, _, _, _, s)) = lookupContext ident s
    expandContext Ret t' (Context (_, p, f, v, s)) = Succ (Context (t', p, f, v, s))
    expandContext (Proc ident ts) t (Context (rt, p, f, v, s)) = do
        p' <- expandContextIf (\(i, t) (JetContextMap (b:bs)) -> null (filter (\(i', t') -> i == i') (M.keys b))) (ident, ts) t p
        Succ (Context (rt, p', f, v, s))
    expandContext (Func ident ts) t (Context (rt, p, f, v, s)) = do
        f' <- expandContextIf (\(i, t) (JetContextMap (b:bs)) -> null (filter (\(i', t') -> i == i') (M.keys b))) (ident, ts) t f
        Succ (Context (rt, p, f', v, s))
    expandContext (Var ident) t (Context (rt, p, f, v, s)) = do
        v' <- expandContext ident t v
        Succ (Context (rt, p, f, v', s))
    expandContext (Typ ident) t (Context (rt, p, f, v, s)) = do
        s' <- expandContext ident t s
        Succ (Context (rt, p, f, v, s'))

makeCheckError e t1 t2 = "Type error"
makeCheckErrorList _ _ _ = "Type error"
makeInferError _ _ = "Type error"

vdeclsTypes :: Functor f => f VarDecl -> f Type
vdeclsTypes = fmap (\(VDecl _ t) -> t)
}

typerule ProgT <- if {ctx} |- [TypeDecl tdecls], {var1} |- (ProcDecl proc) then (Program Prog tdecls proc) return {Succ ()};

typerule TypeDeclT <- if {ctx} |- (Type t) : t', {ctx' <- expandContext (Typ name) t' ctx} then (TypeDecl TDecl name t) return {Succ ctx'};

typerule TypeDeclEmpty <- [TypeDecl] return {Succ ctx};
typerule TypeDeclCons <- if {ctx} |- (TypeDecl decl), {var1} |- [TypeDecl decls] then [TypeDecl decl decls] return {Succ var2};

typerule FuncDeclT <- if
    {ctx' <- expandContext (Func name (vdeclsTypes vdecls)) t ctx},
    {(newBlock ctx')} |- [VarDecl vdecls],
    {var1} |- [Decl decls],
    {ctxret <- expandContext Ret t var2},
    {ctxret} |- [Stmnt stmnts] then (FuncDecl FDecl name vdecls t decls stmnts) return {Succ ctx'};
typerule ProcDeclT <- if
    {ctx' <- expandContext (Proc name (vdeclsTypes vdecls)) TNNone ctx},
    {(newBlock ctx')} |- [VarDecl vdecls],
    {var1} |- [Decl decls],
    {ctxret <- expandContext Ret TNNone var2},
    {ctxret} |- [Stmnt stmnts] then (ProcDecl PDecl name vdecls decls stmnts) return {Succ ctx'};
typerule VarDeclT <- if {ctx} |- (Type t) : t', {ctx' <- expandContext (Var name) t' ctx} then (VarDecl VDecl name t) return {Succ ctx'};

typerule VarDeclEmpty <- [VarDecl] return {Succ ctx};
typerule VarDeclCons <- if {ctx} |- (VarDecl decl), {var1} |- [VarDecl decls] then [VarDecl decl decls] return {Succ var2};

typerule DFuncT <- if {ctx} |- (FuncDecl decl) then (Decl DFunc decl) return {Succ var1};
typerule DProcT <- if {ctx} |- (ProcDecl decl) then (Decl DProc decl) return {Succ var1};
typerule DVar <- if {ctx} |- (VarDecl decl) then (Decl DVar decl) return {Succ var1};

typerule DeclEmpty <- [Decl] return {Succ ctx};
typerule DeclCons <- if {ctx} |- (Decl decl), {var1} |- [Decl decls] then [Decl decl decls] return {Succ var2};

typerule NullT <- (Stmnt SNull) return {Succ ()};
typerule PrintT <- if {ctx} |- (Expr e) : t then (Stmnt SPrint e) return {Succ ()};
typerule AssignT <- if {ctx} |- [LValue name] : t, {ctx} |- (Expr e) : t then (Stmnt SAssign name e) return {Succ ()};
typerule IfT <- if {ctx} |- (Expr e) : TNBool, {ctx} |- [Stmnt s1], {ctx} |- [Stmnt s2] then (Stmnt SIf e s1 s2) return {Succ ()};
typerule WhileT <- if {ctx} |- (Expr e) : TNBool, {ctx} |- [Stmnt stmnts] then (Stmnt SWhile e stmnts) return {Succ ()};
typerule BlockT <- if {(newBlock ctx)} |- [Decl decls], {var1} |- [Stmnt stmnts] then (Stmnt SBlock decls stmnts) return {Succ ()};
typerule PCallT <- if {ctx} |- [Expr params] : ts, {lookupContext (Proc name ts) ctx :: JetError Type} then (Stmnt SCall name params) return {Succ ()};
typerule RetT <- if {ctx} |- (Expr e) : t, {tret <- lookupContext Ret ctx}, {if tret == t then Succ () else Fail "Type error"} then (Stmnt SReturn e) return {Succ ()};

typerule StmntEmpty <- [Stmnt] return {Succ ()};
typerule StmntCons <- if {ctx} |- (Stmnt stmnt), {ctx} |- [Stmnt stmnts] then [Stmnt stmnt stmnts] return {Succ ()};

typerule IntT <- (Expr EInt n) : TNInt return {Succ ()};
typerule TrueT <- (Expr ETrue) : TNBool return {Succ ()};
typerule FalseT <- (Expr EFalse) : TNBool return {Succ ()};
typerule VarT <- if {ctx} |- [LValue name] : t then (Expr EVar name) : t return {Succ ()};
typerule NegT <- if {ctx} |- (Expr e) : TNInt then (Expr ENeg e) : TNInt return {Succ ()};
typerule NotT <- if {ctx} |- (Expr e) : TNBool then (Expr ENot e) : TNBool return {Succ ()};
typerule ECallT <- if {ctx} |- [Expr params] : ts, {t <- lookupContext (Func name ts) ctx} then (Expr ECall name params) : t return {Succ ()};
typerule MulT <- if {ctx} |- (Expr e1) : TNInt, {ctx} |- (Expr e2) : TNInt then (Expr EMul e1 e2) : TNInt return {Succ ()};
typerule AndT <- if {ctx} |- (Expr e1) : TNBool, {ctx} |- (Expr e2) : TNBool then (Expr EAnd e1 e2) : TNBool return {Succ ()};
typerule AddT <- if {ctx} |- (Expr e1) : TNInt, {ctx} |- (Expr e2) : TNInt then (Expr EAdd e1 e2) : TNInt return {Succ ()};
typerule OrT <- if {ctx} |- (Expr e1) : TNBool, {ctx} |- (Expr e2) : TNBool then (Expr EOr e1 e2) : TNBool return {Succ ()};
typerule EqT <- if {ctx} |- (Expr e1) : t, {ctx} |- (Expr e2) : t then (Expr EEq e1 e2) : TNBool return {Succ ()};
typerule LTT <- if {ctx} |- (Expr e1) : TNInt, {ctx} |- (Expr e2) : TNInt then (Expr ELT e1 e2) : TNBool return {Succ ()};
typerule LVVarT <- if {t <- lookupContext (Var name) ctx} then (LValue LVVar name) : t return {Succ ()};
typerule LVArrT <- if {ctx} |- (LValue name) : TNArr t, {ctx} |- (Expr e) : TNInt then (LValue LVArr name e) : t return {Succ ()};

typerule LValueSngle <- if {ctx} |- (LValue name) : t then [LValue name] : t return {Succ ()};
typerule LValueCons <- if {let Context (tret, _, _, _, tdecls) = ctx; ctx' = Context (tret, emptyContext, emptyContext, emptyContext, tdecls)}, 
        {ctx} |- (LValue nhead) : tmp, {ctx} |- (Type tmp) : TNRec decls, {ctx'} |- [VarDecl decls], {var1} |- [LValue ntail] : t then 
    [LValue nhead ntail] : t return {Succ ()};

typerule ExprEmpty <- if {let ts = []} then [Expr] : ts return {Succ ()};
typerule ExprCons <- if {ctx} |- (Expr e) : t, {ctx} |- [Expr es] : ts, {let ts' = t : ts} then [Expr e es] : ts' return {Succ ()};

typerule SynT <- if {t' <- lookupContext (Typ t) ctx} then (Type TNSyn t) : t' return {Succ ()};
typerule TypeT <- (Type t) : t return {Succ ()};
\end{lstlisting}

\section{Simply Typed Lambda Calculus}
$\Gamma$ in the Simply Typed Lambda Calculus is simply just a single name-space containing variables.
\subsection{AST}
\begin{lstlisting}[language=Haskell]
data Id = Id String

data Literal 
    = LitInt Integer
    | LitBool Bool

data Expr 
    = Var Id
    | Lit Literal
    | App Expr Expr -- e1 e2
    | Lam Id Type Expr -- \ name : t . e
    | If Expr Expr Expr -- \ if e1 then e2 else e3
    | Fix Expr -- fix e
    | Pred Expr -- pred e
    | ESucc Expr -- succ e
    | Iszero Expr -- iszero e

data Type
    = TInt - Int
    | TBool - Bool
    | TFun Type Type - t1 -> t2
\end{lstlisting}

\subsection{Type Rules}

\begin{figure}[H]
    \begin{center}
        \LeftLabel{STLCBool: }
        \RightLabel{$[b \in \{true, false\}]$}
        \AxiomC{}
        \UnaryInfC{$\Gamma \vdash b : Bool$}
        \DisplayProof{}
        \ \\\ \\
        \LeftLabel{STLCInt: }
        \RightLabel{$[n \in \mathbb{N}]$}
        \AxiomC{}
        \UnaryInfC{$\Gamma \vdash n : Int$}
        \DisplayProof{}
        \ \\\ \\
        \LeftLabel{STLCVar: }
        \RightLabel{$[\tau = lookup(x, \Gamma)]]$}
        \AxiomC{}
        \UnaryInfC{$\Gamma \vdash n : \tau$}
        \DisplayProof{}
        \ \\\ \\
        \LeftLabel{STLCLam: }
        \AxiomC{$expandContext(x, \sigma, newblock(\Gamma)) \vdash e : \tau$}
        \UnaryInfC{$\Gamma \vdash (\lambda x : \sigma\ .\ e) : (\sigma \rightarrow \tau)$}
        \DisplayProof{}

        \ \\\ \\
        \LeftLabel{STLCApp: }
        \AxiomC{$\Gamma \vdash e_1 : \sigma \rightarrow \tau$}
        \AxiomC{$\Gamma \vdash e_2 : \sigma$}
        \BinaryInfC{$\Gamma \vdash e_1\ e_2 : \tau$}
        \DisplayProof{}

        \ \\\ \\
        \LeftLabel{STLCPred: }
        \AxiomC{$\Gamma \vdash e : Int$}
        \UnaryInfC{$\Gamma \vdash pred\ e : Int$}
        \DisplayProof{}
        \ 
        \LeftLabel{STLCSucc: }
        \AxiomC{$\Gamma \vdash e : Int$}
        \UnaryInfC{$\Gamma \vdash succ\ e : Int$}
        \DisplayProof{}

        \ \\\ \\
        \LeftLabel{STLCIszero: }
        \AxiomC{$\Gamma \vdash e : Int$}
        \UnaryInfC{$\Gamma \vdash iszero\ e : Bool$}
        \DisplayProof{}
        \ 
        \LeftLabel{STLCFix: }
        \AxiomC{$\Gamma \vdash e : \tau \rightarrow \tau$}
        \UnaryInfC{$\Gamma \vdash fix\ e : \tau$}
        \DisplayProof{}

        \ \\\ \\
        \LeftLabel{STLCIf: }
        \AxiomC{$\Gamma \vdash e_1 : Bool$}
        \AxiomC{$\Gamma \vdash e_2 : \tau$}
        \AxiomC{$\Gamma \vdash e_3 : \tau$}
        \TrinaryInfC{$\Gamma \vdash if\ e_1\ then\ e_2\ else\ e_3 : \tau$}
        \DisplayProof{}
    \end{center}
    \label{fig:stlcTyperules}
    \caption{Type rules for the simply typed lambda calculus}
\end{figure}

\subsection{JET Specification}
\begin{lstlisting}[language=Haskell]
{
import Lang.Ast

type Context = JetContextMap Id Type
makeCheckError _ _ _ = "Type Error"
makeInferError _ _ = "Type Error"
}

typerule TLitBool <- (Literal LitBool b) : TBool return {return ()};
typerule TLitInt <- (Literal LitInt n) : TInt return {return ()};

typerule TExprVar <- if {t <- lookupContext var ctx} then (Expr Var loc var) : t return {return ()};
typerule TExprLit <- if {ctx} |- (Literal k) : t then (Expr Lit pos k) : t return {return ()} ;
typerule TExprIf <- if {ctx} |- (Expr ep) : TBool, {ctx} |- (Expr et) : t, {ctx} |- (Expr ef) : t then (Expr If ep et ef) : t return {return ()};
typerule TExprApp <- if {ctx} |- (Expr e1) : TFun t1 t2, {ctx} |- (Expr e2) : t1 then (Expr App e1 e2) : t2 return {return ()};
typerule TExprLam <- if {=<< expandContext var t1 (newBlock ctx)} |- (Expr e) : t2 then (Expr Lam var t1 e) : TFun t1 t2 return {return ()};
typerule TExprFix <- if {ctx} |- (Expr e) : TFun t1 t2, {if t1 == t2 then return () else fail "Type error"} then (Expr Fix e) : t1 return {return ()};
typerule TExprPred <- if {ctx} |- (Expr e) : TInt then (Expr Pred e) : TInt return {return ()};
typerule TExprSucc <- if {ctx} |- (Expr e) : TInt then (Expr ESucc e) : TInt return {return ()};
typerule TExprIszero <- if {ctx} |- (Expr e) : TInt then (Expr Iszero e) : TBool return {return ()};
\end{lstlisting}

\chapter{Haskell Modules}
\label{appendix:modules}
\section{Jet Context Base}
\begin{lstlisting}[language=Haskell]
class JetContextBase t where
    emptyContext :: t
    newBlock :: t -> t
\end{lstlisting}
\section{Jet Context Relational}
\begin{lstlisting}[language=Haskell]
class JetContextBase r => JetContext r k t where
    lookupContext :: Ord k => k -> r -> JetError t
    expandContext :: Ord k => k -> t -> r -> JetError r
    expandContextIf :: Ord k => (k -> r -> Bool) -> k -> t -> r -> JetError r
\end{lstlisting}
\subsection{Jet Context Map}
\begin{lstlisting}[language=Haskell]
instance JetContextBase (JetContextMap a b) where
    emptyContext = JetContextMap [M.empty]
    newBlock (JetContextMap blocks) = JetContextMap (M.empty : blocks)

instance JetContext (JetContextMap a b) a b where
    lookupContext key (JetContextMap blocks) = maybe2Error "Lookup error" $ M.lookup key (M.unions blocks)
    expandContext key t (JetContextMap (block:blocks)) = do
        block' <- insertUnique key t block
        return (JetContextMap (block':blocks))
    expandContextIf predicate key t ctx = 
        if predicate key ctx then expandContext key t ctx
        else fail "Predicate check failed"
\end{lstlisting}
\section{Jet Error Monad}
\begin{lstlisting}[language=Haskell]
module JetErrorM where

import Control.Monad
import Control.Monad.Fail

data JetError a = Succ a | Fail String deriving (Show, Read, Eq, Ord)

instance Functor JetError where
    fmap = liftM

instance Applicative JetError where
    pure = Succ
    (<*>) = ap

instance Monad JetError where
    return = pure
    fail = Fail
    Succ a >>= f = f a
    Fail s >>= _ = Fail s

instance MonadFail JetError where
    fail = Fail

maybe2Error :: String -> Maybe a -> JetError a
maybe2Error s Nothing = Fail s
maybe2Error _ (Just a) = Succ a
\end{lstlisting}