\chapter{State of the Art}

The idea of generating code to be a part of the compiler pipeline is not a new idea.
For the first two stages of the frontend of a compiler, the lexical analyser and the syntax analyser, more commonly known as the lexer and the parser, there have been general tools that have existed for defining them using Domain Specific Languages\cite{Bentley:1986:PPL:6424.315691,van2000domain}.
Two such of these languages are Lex, as the lexer, and YACC, as the parser generator, and these would then generate C code to be used to build the rest of a compiler, possibly including a type checker.
These programs have been around for many years, and many derivatives of them have occurred to be used with different general purpose programming languages, such as Alex and Happy for haskell or Jlex and Cup for Java\cite{ranta2012implementing}.
Many languages have been made using these such programs and their respective DSL's, for example the parser for haskell it's self generated by a grammar defined in a piece of Happy code.
How ever not all parts of the compiler pipeline have general tools that are widely used, however some such tools do exist\cite{grimm2007typical,ruler:10.1007/11737414_4}.
\section{Type Systems}

Many languages have a type system in one form or another with varying degrees of complexity.
Ranging from relatively simple type systems in languages such as C to fairly complex and expressive type systems such as ones in Haskell and Agda.

The concept and implementation of type systems has also existed for almost as long as the concept of higher level language to perform computation.

What are Type Systems?



\begin{figure}
    \begin{prooftree}
        \AxiomC{$J_1\ J_2\ \ ..\ \ J_n$}
        \UnaryInfC{$C$}
    \end{prooftree}
    \caption{Example of a natural deduction rule}
\end{figure}
