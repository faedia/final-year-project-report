\chapter{Evaluation and Conclusion}
\label{chap:eval}
Given all of the features and type rule patterns we have discussed in \autoref{chap:Method} we can successfully type check both of our example languages.
As per the aims given in \autoref{itms:aims}, we have discussed how to type check all but user defined types.
The full type specification in JET, along with the natural deduction rules that we are implementing in our JET specification, can be seen in \autoref{appendix:witnessLanguages}.
Whilst we have not discussed how to type check user defined types and record types, the JET specification for the Ad language does support these language features and does successfully type check them.
We did not discuss these type rules because there are no other features in the language that allow for a better representation of user defined types and record types than the representation that can be seen in the JET specification of the Ad language.

Both of our example languages have differing paradigms and language structure, meeting one of our first main criteria of the project which was to be able type check simple programming languages from different paradigms.
The first of our languages, Ad was a small imperative block structured language which had variable, function and procedures along with a few primitive statements and expressions.
Whereas STLC was a functional programming language that allowed for the use of functions as first class citizens, along with higher order functions.
These two languages represent two main paradigms in programming languages and provide a base from which to implement other features in the future that would allow the extension of the type systems in these languages.
The type system features that could be added to these languages are, in brief, for Ad, the use of extending record type to have the ability of subtyping and also implementing full object orientations; and in STLC, the addition of recursive types\cite{pierce2002types,cardelli1996type} and polymorphic types\cite{Cardelli:1985:UTD:6041.6042}.
Whilst these features are not a requirement, they were mentioned as possible end goals that a full tool (JET is currently still a prototype) would need to support.
They would almost certainly be required for a complete tool to support a large variety of complex language.

We have discussed briefly whether JET met our criteria for success.
However, now we will discuss in detail each required feature in our list of aims given in \autoref{itms:aims}.
The notation is simple, there are a few more advanced features for type rules, but, the type rules are similar to if statements in general purpose programming language and should therefore feel familiar to programmers.
In its current state the notation is not as easy to understand or use as it could be.
In order to type check an imperative language that has a relatively complicated context type associated with it the user may have to write a large amount of inline Haskell to get the generated code to function as required.
This can make the given JET specification hard to read in places, an example of this is when trying to type check user defined types in Ad.
We have demonstrated that we can generate code for simple imperative and functional programming languages, I would, therefore, state we have succeeded in type checking languages of varying paradigms.
We have also succeeded in generating code for simple arithmetic expressions and simple statements as we can type check Ad which has both of these language constructs.
Both of our example languages have variables, and we can type check both functions and procedures in Ad.
Ad has complex types as part of its language and we can successfully type check these.
The slight exception is that in order to type check record types correctly we do require a small amount of inline Haskell, however we can type check array types perfectly.
STLC has higher order functions since it is a variation on the Simply Typed Lambda Calculus.
Therefore we can successfully type check function types and lambda expressions.

I conclude that our notation (JET) has been a success as we can type check two complete simple languages. 
The notation is in most places. fairly simple.
It's only in more complicated language constructs that the JET begins to lose its simplicity.
Finally, we can type check all of our required language features and, with some work, we can type check user types, although not as succinctly as we would like.
However, the further work we have mentioned would resolve many of the issues that JET currently has, therefore, making the tool far more usable.

We have very few ethical considerations to make.
We have not used any people or animals as part of the testing process for the tool.
There have been no experiments performed during the development of the tool or in any of its subsequent testing.
Any software that was used during the development of the tool or used as part of the discussing during this report has been credited to the relevant people such as the use of BNFC and Haskell, we have not claimed any of the software as our own.

\section{Further work}
\label{sec:furtherwork}

\subsection{Generated Code Structure}
Whilst we are able to generate correct code that meets our objectives, we are not generating particularly efficient code.
Since JET is still a prototype tool the efficiency of the code generated is not a concern at this time.
However, that is not to say that we should completely ignore the idea of generating more efficient code.
The biggest issue with efficiency is the order of generated code in our generated check functions with a given type.
For example, the code we generate for our if-expressions in STLC is:
\begin{lstlisting}
checkExpr (If e1 e2 e3) jetCheckType ctx = do
    var1 <- checkExpr e1 TBool ctx
    t <- inferExpr e2 ctx
    var2 <- checkExpr e3 t ctx
    if jetCheckType == t then Succ () else Fail "Type error"
\end{lstlisting}
In the event that our type \texttt{t} is not the same the type we wish to check, \texttt{jetCheckType}, then we will end up evaluating all the way to the if condition.
This is not the most efficient code we can generate because we know \texttt{jetCheckType} is going to have to be the same as \texttt{t}, therefore we can say \texttt{jetCheckType} is \texttt{t} and turn the \texttt{inferExpr} into a \texttt{checkExpr}, so we would instead be generating the following code:
\begin{lstlisting}
checkExpr (If e1 e2 e3) t ctx = do
    var1 <- checkExpr e1 TBool ctx
    var2 <- checkExpr e2 t ctx
    var3 <- checkExpr e3 t ctx
    Succ ()
\end{lstlisting}
In this case we will fail as soon as possible if \texttt{e2} does not have the correct type.
Whereas before, we would not fail until after everything else had been evaluated.

\subsection{Language Structure and Extensions}
Syntactically, the language is similar to natural deduction rules, which sets it apart from a tool such as Typical\cite{grimm2007typical}.
This is the main advantage of this tool.
It is designed to represent the natural deduction rules that it seeks to implement, however, it also allows the user to write their own Haskell to add functionality that, otherwise, JET would not be able to support.  

One of the major short comings of JET is the representation of the context, both how it is defined and used throughout a given type system.
As mentioned in \autoref{sec:typical}, I think the representation of the context in Typical is one of the best.
It gives the user the ability to customise the type system as they wish but also provides a notation that is intuitive and easy to use.
We tried, as much as possible, to make the definition of the type classes as easy to use as possible and to help give the user an idea of what information may be required to build a specification in JET.
However, a better notation for defining namespaces would most likely make the tool easier to use than it is in its current form.
Currently you have to write a fair amount of Haskell in order to get the context to act as required.
Whereas, the notation given in \textcite{grimm2007typical} is a much nicer way of representing namespaces.
The inclusion of the predefined instance of the type class for \texttt{JetContextMap} does help provide a base for how the user will define their namespaces.
The lack of language support for functions defined by the Haskell type class reduces the usability of the tool.
This could be fixed by adding extra statements in the language.
For example, if we were to attempt to lookup an item in the context in STLC, instead of writing the premise using inline haskell as \texttt{\{t <- lookupContext var ctx\}}, we could instead have a piece of syntax in JET that encodes this meaning.
For example we could have a grammar rule that has a keyword lookup and takes an item to lookup and a context to lookup within (an example BNF grammar rule can be seen in \autoref{fig:LookupBNF}).
Using this grammar rule the type premise would be rewritten as \texttt{var lookup context : t}.
This would then be generated into the code that appeared in the initial type premise.

\begin{figure}[]
    \centering
    \begin{grammar}
        <TypePremise> ::= \dots \alt <Identifier> `lookup' <Identifier> `:' <Identifier>
    \end{grammar}
    \caption{Possible grammar rule for an extension to JET type premise's}
    \label{fig:LookupBNF}
\end{figure}

I have mentioned that we can type check user defined types.
This is true, but, it requires excessive use of user defined Haskell functions and also takes advantages of how the language is structured and how we generate the output code.
Using odd side effects of the language design to get the required behaviour obviously leads to unintuitive and hard to read type rules.
Therefore, an area of future work to be performed, for JET specifically, would be to design and implement a notation that provides a more intuitive method of performing type checking for user defined types along with record types etc.
This would greatly improve the usability of the tool, making it easier to implement a type checker for a language with a type system involving user defined types.
It ay possibly make it quite simple to type check object oriented language which is class of language that we have not discussed but in the current state would require similar type rules to the ones in Ad for record types.

The type inference algorithm that JET generates is very simple, and is used to retrieve the type of expressions within a given language.
Therefore, JET has no built-in support for type reconstruction or unification\cite{pierce2002types,cardelli1996type}.
This because there is no built-in way of specifying constraint typing relation\cite{pierce2002types}.
This would require a notation in JET that allowed for the definition and use of type constraints.
We will then be required to generate some type unification algorithm such that we can find a valid substitution that satisfies the constraints given. 
If we were able to add these features, then we would be able to start type checking languages of a similar class to standard ML\cite{milner1997definition}, which as of at the moment would either require vast amounts of inline Haskell or cannot be done with JET.

\subsection{Operational Semantics and Big Step Notation}
The notation for the semantics of the evaluation and code generation of languages is very similar to the notation for type systems.
They use the same natural deduction rules, instead of expressing types, it expresses evaluation or the resulting code.
\textcite{pierce2002types} use natural deduction rules to describe the evaluation of their Simply Typed Lambda Calculus.
It is foreseeable then, since we have shown we can generate type checking code, that one could generate code for rules that express the operational semantics of a language.

\section{Conclusion}
It has been shown that it is possible to, at least partially, generate code for type checkers.
It has also been show that code can be generated for languages of different programming paradigms, such as imperative and functional programming languages.
For simpler languages generating type checking code is some what trivial.
However, as the languages gain complexity, generating code becomes more difficult due to the corner cases of the languages semantics.
This is especially prevalent in languages that have complicated structures for their context.
We briefly began to touch on this issue in Ad, however not as much as we would if we attempted to type check languages such as Ada or C++.
This is due to the nature of such languages, their type systems have grown in complexity and do not have form definition for their semantics and as such are open to interpretation.
Therefore, more work will be required, to both extend JET with new features to support more complicated type systems (the ones discussed in \autoref{sec:furtherwork} are but a few), and also produce and test type checkers for more complicated languages using JET with possible extensions added to make this possible.
It may be that some type systems will be able to be generated completely due to the "ad-hoc" nature of compilers and because many languages lack formal specifications for their type systems (along with other parts of the compiler).
There is also the possibility of generating code for evaluation and code generation for languages which could make process of producing whole compilers and interprets much easier and quicker.
