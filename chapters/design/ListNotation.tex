\section{List Notation}
Many languages, in their abstract syntax, have nodes that take a list of nodes to be one of their children, an example of this can be seen as a sequence of statements as part of the Ad language in \autoref{appendix:witnessLanguages}.
It is currently possible to type check these sorts of type rules with the current set of features we have discussed, unfortunately, it requires the use of side conditions to define variables to allow us to define more conventional premises to do the rest of the type checking.
However, since dealing with lists of statements, or expressions, etc. are very common then it is worthwhile to add a set of features to the language that allows the user to define the type rules for lists of nodes.
The set of features should allow the user to create type rules handle that the different patterns of lists that the user may want to have type rules for: such as the empty list; singleton list and the cons of an item and the tail of a list.
The most useful pattern is probably the cons of an item and the rest of the list, represented in Haskell as \texttt{(x:xs)}.
This is because it allows the user to recursively type check an item and then the end of the list.
The other two patterns are useful when defining the base case of the recursive type rule, such as if the list is empty or if the list is the singleton list.
Being able to pattern match on a singleton list allows the user to assert within the type rule that the list cannot be empty.
The list notation can be used to both just perform a check, but also is able to return a type or a list of types.
Returning a list of types is useful when type checking a function or procedure call, as seen in the Ad language, to check that the types of expressions (given as an expression list) matches the expected types of the procedure or function.

\subsection{List Notation in Type Premises}
Using list notation in type premises is similar to how normal type premises are used.
The one difference is instead of encasing the premise with parentheses, we instead encase it within square brackets.
This so that it matches how lists are defined in Haskell, therefore, the users should be familiar with square brackets being associated with lists.
The example for the list of statements is given as \texttt{\{ctx\} |- [Stmnt stmnts]}.
As can be seen it is exactly the same is a normal type premise, with only the type of brackets used being different so that both the user and the tool can tell the difference.
The code that is produced from this type of premise is also very similar to the code produced from a normal premise: \texttt{var1 <- checkStmntList stmnts ctx}.
How the parameters are passed, along with how the result of the premise is bound to a variable, is exactly the same.
The only difference being is the function name has \texttt{List} as a suffix.
If we did not have this suffix then we would call \texttt{checkStmnt} which would lead to a type error as \texttt{checkStmnt} has the type: \texttt{Stmnt -> Context -> JetError a}, whereas the function we are trying to call would have to have type: \texttt{[Stmnt] -> Context -> JetError a}.
The difference being, one expects a statement, the other expects a list of statements.

\subsection{List Notation Type Rules}
We have discusses how list notation is used in premises, however, that is not useful unless code can be generated for type rules using list notation.
Type rules for list notation bear a resemblance to normal type rules, however, they do vary in sum fundamental ways.
They have the same style representing premises in an if-then block, however the consequent is expressed differently.
Similarly to premises, instead of being encased in parentheses they are encased in square brackets.
The major differences occur in the variables that are defined within the rule and what semantic meaning they have.
The simplest form is to just define the Node type and nothing else, for example: \texttt{[Stmnt]}.
This will pattern match on the empty list, in this case, when all statements have been successfully type checked.
The next pattern has the Node type and a variable name, given as: \texttt{[Stmnt s]}.
The pattern matched on the singleton list, where the element with in the list is given the variable name given in the type rule.
The final pattern that exists is the list constructor pattern, this, similar the the last two,  is given as the Node type but now followed by two variable names.
The first variable represents the head of the list to be type checked (the first item), the second variable represents the tail of the list to be type checked (The rest of the list after the head has been removed).
The variables in these type rules can be used the in the premises the same as any other variables.

\begin{figure}
    \begin{prooftree}
        \LeftLabel{StmntList: }
        \AxiomC{$\Gamma \vdash \lfloor stmnt \rfloor\ valid $}
        \AxiomC{$\Gamma \vdash \lfloor stmnts \rfloor \ valid$}
        \BinaryInfC{$\Gamma \vdash \lfloor stmnt;\ stmnts \rfloor\ valid$}
    \end{prooftree}
    \caption{Type rule for a sequence of statements in the Ad language}
    \label{fig:StmntListTypeRule}
\end{figure}

An example of how to define a type rule in list notation is when type checking a list of statements.
The type rule in natural deduction logic can be seen in \autoref{fig:StmntListTypeRule}.
As can be seen the premises in this type rule make a recursive type rule, and our the type rule in JET will have to reflect this meaning.
In the scenario given, the list statements must non-empty, we can create type rules that assert that the non-empty property must hold.

\begin{lstlisting}[caption = Type rule in JET for a seqeunce of statements, label=lst:jetStmntList]
typerule StmntListCons <- if {ctx} |- (Stmnt stmnt), 
        {ctx} |- [Stmnt stmnts] then 
    [Stmnt stmnt stmnts] return {Succ ()};
typerule StmntListSngltn <- if {ctx} |- (Stmnt stmnt),
    then [Stmnt stmnt] return {Succ ()};
\end{lstlisting}

We therefore have the the type rule in JET that can be seen in \autoref{lst:jetStmntList}.
As can be seen to create the behavior we desire need two type rules, one for when we have a list arbitrary length, and another for when the list contains a single item.
This structure forces the list to be nonempty and the type checker will error if the list happens to be empty.
Another important aspect to note is that we can recursively call out same type rule in order to type check the while list.
This can be seen in the list constructor type rule where we type check the current statement but then also go on to type check the rest of the list.
These patterns should be able to handle any usual type rule that a user is going to write.
However they do not cover every possible pattern that could occur as they do not replicate how Haskell does pattern matching on lists but instead uses a subset of the patterns that Haskell has available.
It could be possible in future to expand these patterns so that they do match all of the patterns that Haskell can handle, however we did not require this feature to type check the witness languages given.