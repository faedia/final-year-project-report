\section{Simple Type Rules}
\subsection{Null Statement Type Rule}
The simplest type rule that we can represent is similar to the one shown in \autoref{fig:adnullt}.
This is a type rule with no premises to check, along with no types to check for the consequent.
Simply, if we have done a successfully parsed this statement, then it will always pass its type check.
Therefore the only code we need to generate is whatever needs to be returned back to the parent.
In this case we don't need to return anything, however there maybe cases in other languages where we would want to return the context back to the parent.

An example of rule such as this written in JET can be seen in \autoref{lst:jetAdNullT}
\begin{lstlisting}[caption = Jet type rule for null statement, label=lst:jetAdNullT]
typerule AdNullT <- (Stmnt SNull) return {return ()}
\end{lstlisting}