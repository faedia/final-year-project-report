\section{The JET Language}

JET allows the user to write type rules that will be generated into Haskell code.
In order to do that the user may have to write some Haskell code in an initial section.
For this reason JET consists of two parts: some Haskell code in an initial section, followed by a list of type rules.

In order to write a type system in JET, the user will also be proved with a Haskell type class that defines a number of functions that will be useful when interacting with the context, such as looking up a variable in the context and adding an item to the context.
This type class has been written in such a way to support as many contexts as possible as to not limit what type of languages JET supports.
Along with the type class, there will also be an example of how to implement an instance of the type class using a Haskell map as an example.

\subsection{Initial Inline Haskell}