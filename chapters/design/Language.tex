\section{The JET Language}

JET allows the user to write type rules that will be generated into Haskell code.
In order to do that the user may have to write some Haskell code in an initial section.
For this reason JET consists of two parts: some Haskell code in an initial section, followed by a list of type rules.

In order to write a type system in JET, the user will also be proved with a Haskell type class that defines a number of functions that will be useful when interacting with the context, such as looking up a variable in the context and adding an item to the context.
This type class has been written in such a way to support as many contexts as possible as to not limit what type of languages JET supports.
Along with the type class, there will also be an example of how to implement an instance of the type class using a Haskell map as an example.

\subsection{Initial Inline Haskell}
The code that comes at the top of a JET type specification is a block of inline Haskell code.
This will be put inline at the top of the generated output code.
This is where a user can define useful functions and types that will be needed later.
For example this is where a user can define the type for the context along with defining the instance of the type class JetContext for the users custom context type.
It also allows for the ability to import any other haskell modules, this means that the user can import their Abstract Syntax Definition for that language that may be defined in a separate Haskell module.
The inline Haskell takes the form of $\{Code\}$ and a user can write any Haskell code they like here.
An example use of of this will be to define the module header, along with the context type declaration, and a instance of the JetContext type class using the context the user has declared.
One such example can be seen in \autoref{lst:inlineHaskellCode}, the listing is an abbreviated example of what appears in the JET type specification for the Ad language which can be seen in \autoref{appendix:AdLanguage}.
Here we define the context to be a 4-tuple which contains the current return type of the current subprogram, along with three namespaces, one for procedures; one for functions; and one for variables.
These namespaces utilise the JetContextMap provided to the user.
JetContextMap is a list of Maps, where each Map is a scope in the code.
The instance of JetContext for JetContextMap is already defined for the user an example of how to define an instance of JetContext.
The example does not include full definitions for the functions and type classes, full definitions for JetContext can be seen in \autoref{appendix:jetLanguage}

\begin{lstlisting}[caption = Example of initial inline haskell code, label=lst:inlineHaskellCode]
{
module TypeAd where

Import AbsAd
newtype Context = Context (Type, 
    JetContextMap (Ident, [Type]) Type, 
    JetContextMap (Ident, [Type]) Type, 
    JetContextMap Ident Type)
instance JetContextBase Context where
    emptyContext = 
        Context (TNNone, emptyContext, ...)
    ...
instance JetContext Context ContextKey Type where
    lookupContext (P ident ts) (Context (_, p, _, _)) 
        = lookupContext (ident, ts) p
    ...
updateRetType :: Context -> Type -> Context
-- Other useful functions and definitions
}
\end{lstlisting}

Having this inline code allows the user to specify exactly how they wish the context to function.
It also allows for a set of standard functions that the user can use to interact with the context, along with adding their own custom functions if and when they need them.

\subsection{Type Rule Format}
\todo{The grammar is subject to change, this may be in state of flux for a while}

The format of the type rules is extremely similar to the way we represent type rules in the natural deduction logic.
This allows for simple translation of a natural deduction type rule into one of the type rules written in JET.
We will delve into more detail about the format of the type rules, but the basic structure of the type rules is as follows

\begin{grammar}
    <Rule> ::= `typerule' <Identifier> `<-' ( `if' <TypePremiseList> `then' ) <TypePremise> `return' <InlineHaskell>
\end{grammar}

The type premise list encased within the if-then is the list of judgments to prove the consequent.
It is possible for a type rule to require no judgments to prove the consequent so the if-then is optional, therefore, the type premise list can be forced to be nonempty.
The type premise after the if-then is the consequent we are trying to prove.
We then have a return keyword followed by some inline Haskell.
The reason for this will become more apparent later, but simply the type rule may want to return some more information back up the tree.
For example, if we were type checking a list of statements and we have just seen a declaration for a variable, we will then want to add that variable to the context so that in later statements we have access to the variable's type information.
In this situation we would want to return the new context to the parent so that it can give the new context to the next statement in the list of statements.
The grammar of JET is designed to replicate as much as possible from the natural deduction rules as they are an already accepted way of formally representing type rules.
The formal grammar can be seen in its entirety in \autoref{appendix:jetLanguage}.