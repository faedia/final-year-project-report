\chapter{Method}
Before we can generate type checking code we need to have a way of specifying the type system in such that a program can read the given specification and generate the required code.
The way we will be specifying the the type system is through a domain specified language I have designed called JET Expresses Types (JET).
The way type rules are written in JET are analogous to the natural deduction rules to define the type system.
This appeared to make the most sense as other parts of the compiler pipeline already have domain specific languages that are used to generate code.

In JET, rather than using the concrete syntax that is used in in the natural deduction rules, uses the abstract syntax provided by the user, to define the type rules.
Type rules in JET have a have a similar form to the natural deduction rules, such that there is a rule name followed by some judgments and then the consequent.
This provides a common way of representing type rules even if the way of writing the rules in JET use an ascii representation of the natural deduction logic.

The code we are generating is based off of the type checking and type inference algorithm described in Implementing Programming Languages\cite{ranta2012implementing}.
This is because the the algorithms described are simple yet powerful enough to express common type rules, such as arithmetic expressions and also allow for the use of functions and variables.
It also makes the code generation quite simple as there is a, more or less, direct translation of type rules to a function in the algorithm.

The code we will be generating will be Haskell code, since we can use the features in the Haskell language such as pattern matching and the use of monads to help produce relatively simple code from a given type specification.

Below is explained how the language is used, why certain design decisions were made and what code is generated from the input source. 
This will be demonstrated using a small number of  witness languages defined in \autoref{appendix:witnessLanguages}.
The first language is a simple imperative language. 
The language consists of single procedure in which you can define more procedures and subprograms along with define variables. In the body of the procedure is a series of statements such as a block statement, if-else statement, and for subprograms a return statement. The language is similar to Ada but much simpler.
The next language is the simply typed lambda calculus, this will allow the demonstration of type checking higher order functions.
The lambda calculus is based off of the definition that can be found in Pierce's book, Types and Programming Languages\cite{pierce2002types}.

\section{The JET Language}

JET allows the user to write type rules that will be generated into Haskell code.
In order to do that the user may have to write some Haskell code in an initial section.
For this reason JET consists of two parts: some Haskell code in an initial section, followed by a list of type rules.

In order to write a type system in JET, the user will also be proved with a Haskell type class that defines a number of functions that will be useful when interacting with the context, such as looking up a variable in the context and adding an item to the context.
This type class has been written in such a way to support as many contexts as possible as to not limit what type of languages JET supports.
Along with the type class, there will also be an example of how to implement an instance of the type class using a Haskell map as an example.

\subsection{Initial Inline Haskell}
\section{Simple Type Rules}
\subsection{Null Statement Type Rule}
The simplest type rule that we can represent is similar to the one shown in \autoref{fig:adnullt} for our example of an imperative language.
This is a type rule with no premises to check, along with no types to check for the consequent.
Simply, if we have done a successfully parsed this statement, then it will always pass its type check.
Therefore the only code we need to generate is whatever needs to be returned back to the parent.
In this case we don't need to return anything, however there maybe cases in other languages where we would want to return the context back to the parent.

An example of rule such as this written in JET can be seen in \autoref{lst:jetAdNullT}
\begin{lstlisting}[caption = JET type rule for null statement, label=lst:jetAdNullT]
typerule AdNullT <- (Stmnt SNull) return {return ()}
\end{lstlisting}

The initial part \texttt{type AdNullT} has no affect on the rule is just a name of the rule for a use in future documentation.
The following part after the left facing arrow is the piece of abstract syntax that we wish to pattern match against.
This is analogous to the concrete syntax used to define the consequent in the natural deduction logic.
The abstract syntax definition takes the form of the type of abstract syntax we wish to check followed by the constructor we want to pattern match against, hence why for this example it is \texttt{Stmnt SNull} since we wish to check a statement and the specific statement is the null statement.
We need the AST node type name so that we can form the generated code correctly, we also need the constructor so that we can use Haskell's features of pattern matching to match against the correct piece of syntax.

From this type rule we produce the code that can be seen in \autoref{lst:codeAdNullT}.
we generate the name of the function based off of the AST node type name given.
Also notice that since the type rule contained no information about resulting type of the rule no code was produced to do any type checking.
This also means that there is no need to generate a function to infer the type given by this rule as there is no type associated with the rule, therefore, we do not generate the function to do such a type inference.
Therefore, the only thing the code does is pattern match against the piece of specified syntax and return an empty tuple since Statement rules in the language do not affect the context in any way.

\begin{lstlisting}[caption = Code generated from AdNullT, label=lst:codeAdNullT]
checkStmnt SNull ctx = do
    return ()
\end{lstlisting}

One possible improvement to be made in the code is instead of generating code of the form \texttt{foo astNode ctx} is to instead generate code to be of the form \texttt{foo ctx astNode}.
That way if we have multiple multiple instances of \texttt{foo ctx}, then we can give that expression a name and only evaluate it once.
However, there is one possible instance where having \texttt{ctx} be last argument has an advantage which can be seen at \todo{autoref to monad context binding}.

\subsection{Literal Expression Type Rule}
Both of our test languages have literal expressions.
Literal expressions are similar to our null statements example, in that there is not much that they type checker algorithm has to do.
This is because, similar to null statements, they have no premises, and do not affect the context in anyway.
The only difference between literal expressions and null statements is that literal expressions have types associated with them where as statements do not.

\begin{lstlisting}[caption = JET type rule for simple literal expressions, label=lst:jetAdLitExpr]
typerule AdTrueT <- (Expr ETrue) : TNBool return {return ()};
\end{lstlisting}

As can be seen in \autoref{lst:jetAdLitExpr}, we have specified something that was not specified in \autoref{lst:jetAdNullT}.
That is we have annotated the type rule with the type that is associated with the code we are checking.
In the case of a boolean \texttt{true} it has a type of boolean, where as a number has a type of integer.

Since this rule has a type associated, the code we must generate is now more complicated.
For example, we did not have to check the type of the null statement because it did not have a type.
We also did not have to generate a function to infer the type of the null statement because again, there is no type to infer.
However, since literal expressions do have types, we do have to check the type and provide an infer function to get the type of a literal expression.

The check function we generate, which can be seen in \autoref{lst:codeAdTrueTCheck}, should assert that the incoming type is, in this case, the \texttt{TNBool}, otherwise we should produce some form of error.
Errors take the form of a monad, in particular they are of type \texttt{JetError}, this monad is given to the user in a haskell module JetErrorM when they generate their code.
This allows us to propagate the error, along with the error message back out to the user without having to call the error function explicitly\todo{Cite Functional programming book}.
To produce the error message the generated code will call \texttt{makeCheckError} with the data that caused the error.
The user must then define the function \texttt{makeCheckError}, to keep things simple in the example cases \texttt{makeCheckError} is defined to return "Type Error".
However the user could write any code they liked as long it returned a String.

\begin{lstlisting}[caption = Code generated for checkExpr from AdTrueT, label=lst:codeAdTrueTCheck]
checkExpr ETrue jetCheckType ctx = do
    if jetCheckType == TNBool then return () 
    else fail (makeCheckError ETrue jetCheckType TNBool)
\end{lstlisting}

In \autoref{lst:altCodeAdTrueTCheck} I have given another piece of code that I could have generated that would have done the same thing.
It uses pattern matching on the input type as well as the expression, this means we do not require if expression within the function.
However we still do need the more generic function in order for us to be able to correct fail and call the \texttt{makeCheckError} function.

\begin{lstlisting}[caption = Alternate Code for checkExpr from AdTrueT, label=lst:altCodeAdTrueTCheck]
checkExpr ETrue TNBool ctx = then return () 
checkExpr ETrue jetCheckType ctx = 
    fail (makeCheckError ETrue jetCheckType TNBool)
\end{lstlisting}

In hindsight, this is a better to structure the code.
This is because under certain circumstances, such as the ones seen in\todo{autoref to premises}, it can reduce the amount of unnecessary evaluation that may be required due to generating the code with the if expression being the last monadic statement.

Infer functions return the type given by the type rule where as check functions assert that the type we are expecting is the same as the one given by the type rule.
From this, for example given in \autoref{lst:jetAdLitExpr}, we generate the code that can be seen in \autoref{lst:codeAdTrueTInfer}.
The code generated is very similar to the code seen in \autoref{lst:codeAdNullT}.
This is because there are no premises to check and we do not need to check the type specified by the type rule, we just need to return the type.
Therefore, the only thing this function does is return the type associated with the literal value $true$.

\begin{lstlisting}[caption = Code generated for inferExpr from AdTrueT, label=lst:codeAdTrueTInfer]
inferExpr ETrue ctx = do
    return TNBool
\end{lstlisting}
\section{Type Rules with Premises}
The type rules we have discussed so far have been relatively uninteresting and extremely simply.
While these sorts of type rules are simple they will be required in every language at some point.
The simpleness of the type rules comes from the fact that none of them required any premises, this means that as long as the consequent is true then the evaluation of the type rule will succeed.
The type rules we are going to discuss in this section will have premises of varying complexity, this will demonstrate what features are implemented in JET that allow the user to define type rules like these and how these type rules are then turned into code.

\subsection{Simply Premise}
The first type rule that we will look at is the rule STLCPred in \autoref{fig:stlcTyperules}.
This is a rule for an expression in the Simply Typed Lambda Calculus that takes the predecessor of a number.
Hence it is a rule that requires that the type of the expression it is acting on is also a number.
Therefore, the premise for this rule is that the expression is of type $Int$ which means that $pred\ e$ is of type $Int$.
A representation of this in JET is given in \autoref{lst:jetSTLCPredExpr}

\begin{lstlisting}[caption = Type rule for pred expression involving one premise, label=lst:jetSTLCPredExpr]
typerule TExprPred <- if {ctx} |- (Expr e) : TInt then 
    (Expr Pred e) : TInt return {return ()};
\end{lstlisting}

As can now be seen, the type rule takes the appearance of an if-then statement in traditional imperative languages.
It can be thought of that the premise is a guard to the consequent, such that the consequent can only happen if we can prove the premise.
The type premise takes a similar form as the consequent of the type rule, however, there is once difference.
Just as a type rule may need to return some information about a change in the context, such as in the case of declarations, a type premise will need to be told what context we need to give to the type judgement such that we can prove it.
Similarly to the consequent, we can also specify that the type of the premise must match some expected type.
In this instance we are checking to make sure that the premise is of type integer such that we can say the consequent is of type integer.
Another piece of syntax to note is we have also given a parameter to the constructor that we wish to pattern match in the consequent.
This is so we can produce a valid constructor, but also so that we can perform premises on the paramters of the constructor, such as making sure that the expression for $pred$ is of type integer.

When generating code for a check function involving a premise we want to make sure that we evaluate the premise before we return from the check function.
We do this by making the premise a monadic statement that is executed before the return of the check function.
Another reason why the premise is a monad is so that we can propagate the error back to the user in the event that the premise fails.
This means that no other code is required when handling the error case of the premise due to Haskell's do notation.
From these requirements we produce the code that can be seen in \autoref{lst:codeTExprPredCheck}.

\begin{lstlisting}[caption = Code generated for checkExpr from TExprPred, label=lst:codeTExprPredCheck, language=Haskell]
checkExpr (Pred e) jetCheckType ctx = do
    var1 <- checkExpr e TInt ctx
    if jetCheckType == TInt then return () 
    else fail (makeCheckError (Pred e) jetCheckType TInt)
\end{lstlisting}

One thing to note in the code that wasn't already stated is that we store the result of the successful return of the premise in a variable so that it can be used later if necessary.
We now also see the advantage that the style of code in \autoref{lst:altCodeAdTrueTCheck} would have provided.
If had produced code like that then we would not have to unnecessarily evaluate \texttt{checkExpr e TInt ctx}.
We would only want to evaluate that premise if the type to be checked was integer.
If it was not of type integer then the evaluation of the premise was unnecessary because we could have already known that the type check was going to fail.

Similarly to the check function, when trying to infer the type we also want to run the premises and fail if the premises fail.
Otherwise if we succeed when the premises fail then we have falsely provided a proof of the type.

The code for the infer function, see in \autoref{lst:codeTExprPredInfer}, is very similar to what is seen in the code for the check function for the type rule.
An infer function must perform the same premises as the check function.
Where they differ is in what they return, a check function returns what ever changes it made to the context or anything else of the sort, where as infer function return the type that they have inferred from the given input and context.
This function is so simple that there is not much that could be changed in the code it generates.
It does what is required of it to perform the proof of the type and then return the type.
There is no way to not perform the premises, unlike the optimisation of the check function, as the premise is what proves the type to return and are therefore required to be evaluated.

\begin{lstlisting}[caption = Code generated for inferExpr from TExprPred, label=lst:codeTExprPredInfer, language=Haskell]
    var1 <- checkExpr e TInt ctx
    return TInt
\end{lstlisting}

\subsection{Premises with Type Variables}
There is more we can do with premises other than just assert, for example, something is of type integer or boolean.
So far we have only covered type rules with specified typed.
However, in natural deduction logic we can make statements about arbitrary types such as the if expression inference rule seen in \autoref{fig:ifExprRule}.
Here we have the expression $e_1$ with the type $Bool$, where as $e_2$ and $e_3$ have arbitrary types $\tau$.
Along with the whole expression having type $\tau$.
In inference rules all occurrences of a variable have the same value, therefore, all $\tau$'s are the same.

In Jet, similarly to natural deduction logic, we can not only specify types but also say that something is of an arbitrary type variable.
How the distinction is handled is the same way Haskell handles the distinction between types and type variables: types start with an upper case letter and type variables start with a lower case letter.

\begin{figure}[tbp]
    \begin{prooftree}
        \LeftLabel{IfExpr: }
        \AxiomC{$\Gamma \vdash e_1 : Bool$}
        \AxiomC{$\Gamma \vdash e_2 : \tau$}
        \AxiomC{$\Gamma \vdash e_3 : \tau$}
        \TrinaryInfC{$\Gamma \vdash if\ e_1\ then\ e_2\ else\ e_3 : \tau$}
    \end{prooftree}
    \label{fig:ifExprRule}
    \caption{Example of If-Expression inference rule}
\end{figure}

From these constraints we can specify a type rule to type check and type infer the inference expression given in \autoref{fig:ifExprRule}.
It is very similar to type rules that we have seen before, just instead of giving a type for $e_2$ and $e_3$ we will be giving them a type variable.
We will also need three type premises, we can define any number of type premises, where the premises are comma separated.
The last thing we need to do is to make sure the consequent is of the same type as $e_2$ and $e_3$, we do this by saying it is the same type variable as them.
The whole type rule can be seen in \autoref{lst:jetSTLCIfExpr}

\begin{lstlisting}[caption = Type rule for if expression expression involving type variables, label=lst:jetSTLCIfExpr]
typerule TExprIf <- if 
        {ctx} |- (Expr e1) : TBool, 
        {ctx} |- (Expr e2) : t, 
        {ctx} |- (Expr e3) : t then 
    (Expr If e1 e2 e3) : t return {return ()};
\end{lstlisting}

Generating code for premises where we know the type of the premise is relatively simply.
However, since we now have to deal with type variables, we need to figure out the type of the premises. 
This is why we have been generating infer functions.
Check functions check that the rule is the same as the incoming type.
Whereas infer function return the type so that it can be used in other type rules.
However if we have the same type rule in multiple premises then we do not want to figure out the same type twice for a number of reasons.
The first reason is that there is no point to do so, we in already know the type, we only need to make sure that this premise is of the same type as the previous premise with that type variable.
The other reason is the types for the same type variable could be different and we would not know, this means that the type checker could say that an input was type correct when in reality it should have been a type error.
Therefore we need to keep track of what type variables we have seen when generating the code for the premises.

We will not discuss the check function for this example because the code generated for the premises in both the check functions and infer functions will always be the same.

The code we produce for the infer function can be see in \autoref{lst:codeTExprIfInfer}.
The first premise is as we have seen previously, it is a know type so we are able to perform the check function on it.
The next statement relates to the second premise in the type rule.
This premise refers to a type variable that we currently know nothing about, therefore, the code we generate will want to be the infer function so that we can figure out the type that the type variable, t, represents.
We always bind the result of the infer function to a variable of the same name as the type variable in the specification.
Then we move onto the final premise, which also refers to the type variable.
At this point we have already figured out the type that t represents.
This is why instead of calling inferExpr again, we can now call checkExpr where the type to check for is now the variable t that was assigned in the previous premise.
Finally, we then wish to return the type t rather than some known type.

\begin{lstlisting}[caption = Code generated for type variables in inferExpr from TExprIf, label=lst:codeTExprIfInfer, language=Haskell]
    t <- inferExpr e2 ctx
    var2 <- checkExpr e3 t ctx
    return t
\end{lstlisting}

We can also type variables with the type constructors.
This allows us to extract the parameters to some type constructor.
For example, if there was an array access expression, we may want to know the type that the elements of the array hold.
The natrual deduction rule for this kind of type rule can be seen in \autoref{fig:typeRuleArr} \todo{cite TAPL book where the example comes from}.
In this example the definition of the array type in the AST would be \texttt{Type = TArr Type | ...}.
As can be seen the array type takes in a type as a parameter which is the type of the elements of the array.
Therefore, when type checking an array access, to retrieve the type of the element of the array you would want extract that parameter from the array.
This can be achieved through the use of Haskell's pattern matching features.
We want to generate some code of the form \texttt{TArr t <- inferExpr e ctx}, because this code asserts that the type returned from infer function must be an array type, we then get the parameter of the array type and store it in some variable t.
The generator will also have to add the variable t to the list of known type variables so that if another premise required the use of this type variable then we can generate a check function instead of an infer function.
We express this premise in the type rule in a very similar manner to what code is generated:
\texttt{\{ctx\} |- (Expr e) : TArr t}.
There is one issue with the code that we generate at the moment.
If the return of the infer function does not match the pattern function then the type checker will error, but not with our defined error function since the error was an internal Haskell error.
Instead the right hand side of the binding will want to call our error function in case of a bad patter match, from this criteria we produce the follow code: \texttt{case inferExpr e ctx of Succ jet0@(TArr t) -> Succ jet0; Succ t -> Fail (makeInferError t t); x -> x}.

\begin{figure}[t]
    \centering
    \begin{prooftree}
        \LeftLabel{ArrAccess: }
        \AxiomC{$\Gamma \vdash e : Arr\ t$}
        \AxiomC{$\Gamma \vdash i : Int$}
        \BinaryInfC{$\Gamma \vdash e[i] : t$}
    \end{prooftree}
    \caption{Example type rule for array access}
    \label{fig:typeRuleArr}
\end{figure}

This kind of type rule is used in both witness languages.
The Simply Typed Lambda Calculus required it so that lambda abstract and lambda application can be type checked.
This feature is also required so that the fix point operator can be type checked.
It is required in the Ad language so that array types and record types can be type checked correctly.


\subsection{Side Conditions}
There is another way to represent type premises, and that is through the use of side conditions.
Side conditions are extra predicates along side normal premises that must also be true for the consequent to be true.
A user may wish to represent these side conditions with in their JET specification.
Currently, from all the features of JET that we have discusses, there is no way to easily evaluate arbitrary code such that it can match any possible arbitrary predicate that could appear in a type specification in natural deduction logic.
This is why there is feature that allows the user to write any arbitrary Haskell code and it be evaluated as a premise.
For example is a user want to have a language that allowed the assignment of arbitrary number of variables in a single statement i.e. \texttt{a, b, c, d := e$_1$, e$_2$, e$_3$, e$_4$}, where a is assigned to e$_1$ and b is assigned to e$_2$ etc.
In a statement like this it would be reasonable to expect that the list of variables to assign to is the same as the list of expressions.
We could use the side conditions, it would be written in place of a normal type premise and would instead be a piece of inline Haskell code, where inline Haskell code is surrounded by curly braces (\{\}).
There fore this could be written as: \texttt{\{if length vars == length exprs then return () else fail "Type error"\}}

\subsection{Limitations}
\todo{discuss limitations on the use of type variables.}
\section{List Notation}
\section{Utilising the Context}