\chapter{Method}
\label{chap:Method}
Before we can generate type checking code we need to have a way of specifying the type system in such that a program can read the given specification and generate the required code.
The way we will be specifying the the type system is through a domain specified language I have designed called JET Expresses Types (JET).
The way type rules are written in JET are analogous to the natural deduction rules to define the type system.
This appeared to make the most sense as other parts of the compiler pipeline already have domain specific languages that are used to generate code.

In JET, rather than using the concrete syntax that is used in in the natural deduction rules, uses the abstract syntax provided by the user, to define the type rules.
Type rules in JET have a have a similar form to the natural deduction rules, such that there is a rule name followed by some judgments and then the consequent.
This provides a common way of representing type rules even if the way of writing the rules in JET use an ascii representation of the natural deduction logic.

The code we are generating is based off of the type checking and type inference algorithm described in Implementing Programming Languages\cite{ranta2012implementing}.
This is because the the algorithms described are simple yet powerful enough to express common type rules, such as arithmetic expressions and also allow for the use of functions and variables.
It also makes the code generation quite simple as there is a, more or less, direct translation of type rules to a function in the algorithm.

The code we will be generating will be Haskell code, since we can use the features in the Haskell language such as pattern matching and the use of monads to help produce relatively simple code from a given type specification.

Below is explained how the language is used, why certain design decisions were made and what code is generated from the input source. 
This will be demonstrated using a small number of  witness languages defined in \autoref{appendix:witnessLanguages}.
The first language is a simple imperative language. 
The language consists of single procedure in which you can define more procedures and subprograms along with define variables. In the body of the procedure is a series of statements such as a block statement, if-else statement, and for subprograms a return statement. The language is similar to Ada but much simpler.
The next language is the simply typed lambda calculus, this will allow the demonstration of type checking higher order functions; lambda abstraction; and let expressions.
The lambda calculus is based off of the definition that can be found in Pierce's book, Types and Programming Languages\cite{pierce2002types}.

\section{The JET Language}

JET allows the user to write type rules that will be generated into Haskell code.
In order to do that the user may have to write some Haskell code in an initial section.
For this reason JET consists of two parts: some Haskell code in an initial section, followed by a list of type rules.

In order to write a type system in JET, the user will also be proved with a Haskell type class that defines a number of functions that will be useful when interacting with the context, such as looking up a variable in the context and adding an item to the context.
This type class has been written in such a way to support as many contexts as possible as to not limit what type of languages JET supports.
Along with the type class, there will also be an example of how to implement an instance of the type class using a Haskell map as an example.

\subsection{Initial Inline Haskell}
The code that comes at the top of a JET type specification is a block of inline Haskell code.
This will be put inline at the top of the generated output code.
This is where a user can define useful functions and types that will be needed later.
For example this is where a user can define the type for the context along with defining the instance of the type class JetContext for the users custom context type.
It also allows for the ability to import any other haskell modules, this means that the user can import their Abstract Syntax Definition for that language that may be defined in a separate Haskell module.
The inline Haskell takes the form of $\{Code\}$ and a user can write any Haskell code they like here.
An example use of of this will be to define the module header, along with the context type declaration, and a instance of the JetContext type class using the context the user has declared.
One such example can be seen in \autoref{lst:inlineHaskellCode}, the listing is an abbreviated example of what appears in the JET type specification for the Ad language which can be seen in \autoref{appendix:AdLanguage}.
Here we define the context to be a 4-tuple which contains the current return type of the current subprogram, along with three namespaces, one for procedures; one for functions; and one for variables.
These namespaces utilise the JetContextMap provided to the user.
JetContextMap is a list of Maps, where each Map is a scope in the code.
The instance of JetContext for JetContextMap is already defined for the user an example of how to define an instance of JetContext.
The example does not include full definitions for the functions and type classes, full definitions for JetContext can be seen in \autoref{appendix:jetLanguage}

\begin{lstlisting}[caption = Example of initial inline haskell code, label=lst:inlineHaskellCode, language=Haskell]
{
module TypeAd where

Import AbsAd
newtype Context = Context (Type, 
    JetContextMap (Ident, [Type]) Type, ...
instance JetContextBase Context where
    emptyContext = 
        Context (TNNone, emptyContext, ...)
    ...
instance JetContext Context ContextKey Type where
    lookupContext (P ident ts) (Context (_, p, _, _)) 
        = lookupContext (ident, ts) p
    ...
updateRetType :: Context -> Type -> Context
-- Other useful functions and definitions
}
\end{lstlisting}

Having this inline code allows the user to specify exactly how they wish the context to function.
It also allows for a set of standard functions that the user can use to interact with the context, along with adding their own custom functions if and when they need them.

\subsection{Type Rule Format}
\todo{The grammar is subject to change, this may be in state of flux for a while}

The format of the type rules is extremely similar to the way we represent type rules in the natural deduction logic.
This allows for simple translation of a natural deduction type rule into one of the type rules written in JET.
We will delve into more detail about the format of the type rules, but the basic structure of the type rules is as follows

\begin{grammar}
    <Rule> ::= `typerule' <Identifier> `<-' ( `if' <TypePremiseList> `then' ) <TypePremise> `return' <InlineHaskell>
\end{grammar}

The type premise list encased within the if-then is the list of judgments to prove the consequent.
It is possible for a type rule to require no judgments to prove the consequent so the if-then is optional, therefore, the type premise list can be forced to be nonempty.
The type premise after the if-then is the consequent we are trying to prove.
We then have a return keyword followed by some inline Haskell.
The reason for this will become more apparent later, but simply the type rule may want to return some more information back up the tree.
For example, if we were type checking a list of statements and we have just seen a declaration for a variable, we will then want to add that variable to the context so that in later statements we have access to the variable's type information.
In this situation we would want to return the new context to the parent so that it can give the new context to the next statement in the list of statements.
The grammar of JET is designed to replicate as much as possible from the natural deduction rules as they are an already accepted way of formally representing type rules.
The formal grammar can be seen in its entirety in \autoref{appendix:jetLanguage}.
\section{Simple Type Rules}
\subsection{Null Statement Type Rule}
The simplest type rule that we can represent is similar to the one shown in \autoref{fig:adnullt} for our example of an imperative language.
This is a type rule with no premises to check, along with no types to check for the consequent.
Simply, if we have done a successfully parsed this statement, then it will always pass its type check.
Therefore the only code we need to generate is whatever needs to be returned back to the parent.
In this case we don't need to return anything, however there maybe cases in other languages where we would want to return the context back to the parent.

An example of rule such as this written in JET can be seen in \autoref{lst:jetAdNullT}
\begin{lstlisting}[caption = JET type rule for null statement, label=lst:jetAdNullT]
typerule AdNullT <- (Stmnt SNull) return {return ()}
\end{lstlisting}

The initial part \texttt{type AdNullT} has no affect on the rule is just a name of the rule for a use in future documentation.
The following part after the left facing arrow is the piece of abstract syntax that we wish to pattern match against.
This is analogous to the concrete syntax used to define the consequent in the natural deduction logic.
The abstract syntax definition takes the form of the type of abstract syntax we wish to check followed by the constructor we want to pattern match against, hence why for this example it is \texttt{Stmnt SNull} since we wish to check a statement and the specific statement is the null statement.
We need the AST node type name so that we can form the generated code correctly, we also need the constructor so that we can use Haskell's features of pattern matching to match against the correct piece of syntax.

From this type rule we produce the code that can be seen in \autoref{lst:codeAdNullT}.
we generate the name of the function based off of the AST node type name given.
Also notice that since the type rule contained no information about resulting type of the rule no code was produced to do any type checking.
This also means that there is no need to generate a function to infer the type given by this rule as there is no type associated with the rule, therefore, we do not generate the function to do such a type inference.
Therefore, the only thing the code does is pattern match against the piece of specified syntax and return an empty tuple since Statement rules in the language do not affect the context in any way.

\begin{lstlisting}[caption = Code generated from AdNullT, label=lst:codeAdNullT]
checkStmnt SNull ctx = do
    return ()
\end{lstlisting}

One possible improvement to be made in the code is instead of generating code of the form \texttt{foo astNode ctx} is to instead generate code to be of the form \texttt{foo ctx astNode}.
That way if we have multiple multiple instances of \texttt{foo ctx}, then we can give that expression a name and only evaluate it once.
However, there is one possible instance where having \texttt{ctx} be last argument has an advantage which can be seen at \todo{autoref to monad context binding}.

\subsection{Literal Expression Type Rule}
Both of our test languages have literal expressions.
Literal expressions are similar to our null statements example, in that there is not much that they type checker algorithm has to do.
This is because, similar to null statements, they have no premises, and do not affect the context in anyway.
The only difference between literal expressions and null statements is that literal expressions have types associated with them where as statements do not.

\begin{lstlisting}[caption = JET type rule for simple literal expressions, label=lst:jetAdLitExpr]
typerule AdTrueT <- (Expr ETrue) : TNBool return {return ()};
\end{lstlisting}

As can be seen in \autoref{lst:jetAdLitExpr}, we have specified something that was not specified in \autoref{lst:jetAdNullT}.
That is we have annotated the type rule with the type that is associated with the code we are checking.
In the case of a boolean \texttt{true} it has a type of boolean, where as a number has a type of integer.

Since this rule has a type associated, the code we must generate is now more complicated.
For example, we did not have to check the type of the null statement because it did not have a type.
We also did not have to generate a function to infer the type of the null statement because again, there is no type to infer.
However, since literal expressions do have types, we do have to check the type and provide an infer function to get the type of a literal expression.

The check function we generate, which can be seen in \autoref{lst:codeAdTrueTCheck}, should assert that the incoming type is, in this case, the \texttt{TNBool}, otherwise we should produce some form of error.
Errors take the form of a monad, in particular they are of type \texttt{JetError}, this monad is given to the user in a haskell module JetErrorM when they generate their code.
This allows us to propagate the error, along with the error message back out to the user without having to call the error function explicitly\todo{Cite Functional programming book}.
To produce the error message the generated code will call \texttt{makeCheckError} with the data that caused the error.
The user must then define the function \texttt{makeCheckError}, to keep things simple in the example cases \texttt{makeCheckError} is defined to return "Type Error".
However the user could write any code they liked as long it returned a String.

\begin{lstlisting}[caption = Code generated for checkExpr from AdTrueT, label=lst:codeAdTrueTCheck]
checkExpr ETrue jetCheckType ctx = do
    if jetCheckType == TNBool then return () 
    else fail (makeCheckError ETrue jetCheckType TNBool)
\end{lstlisting}

In \autoref{lst:altCodeAdTrueTCheck} I have given another piece of code that I could have generated that would have done the same thing.
It uses pattern matching on the input type as well as the expression, this means we do not require if expression within the function.
However we still do need the more generic function in order for us to be able to correct fail and call the \texttt{makeCheckError} function.

\begin{lstlisting}[caption = Alternate Code for checkExpr from AdTrueT, label=lst:altCodeAdTrueTCheck]
checkExpr ETrue TNBool ctx = then return () 
checkExpr ETrue jetCheckType ctx = 
    fail (makeCheckError ETrue jetCheckType TNBool)
\end{lstlisting}

In hindsight, this is a better to structure the code.
This is because under certain circumstances, such as the ones seen in\todo{autoref to premises}, it can reduce the amount of unnecessary evaluation that may be required due to generating the code with the if expression being the last monadic statement.

Infer functions return the type given by the type rule where as check functions assert that the type we are expecting is the same as the one given by the type rule.
From this, for example given in \autoref{lst:jetAdLitExpr}, we generate the code that can be seen in \autoref{lst:codeAdTrueTInfer}.
The code generated is very similar to the code seen in \autoref{lst:codeAdNullT}.
This is because there are no premises to check and we do not need to check the type specified by the type rule, we just need to return the type.
Therefore, the only thing this function does is return the type associated with the literal value $true$.

\begin{lstlisting}[caption = Code generated for inferExpr from AdTrueT, label=lst:codeAdTrueTInfer]
inferExpr ETrue ctx = do
    return TNBool
\end{lstlisting}
\section{Type Rules with Premises}
The type rules we have discussed so far have been relatively uninteresting and extremely simply.
While these sorts of type rules are simple they will be required in every language at some point.
The simpleness of the type rules comes from the fact that none of them required any premises, this means that as long as the consequent is true then the evaluation of the type rule will succeed.
The type rules we are going to discuss in this section will have premises of varying complexity, this will demonstrate what features are implemented in JET that allow the user to define type rules like these and how these type rules are then turned into code.

\subsection{Simply Premise}
The first type rule that we will look at is the rule STLCPred in \autoref{fig:stlcTyperules}.
This is a rule for an expression in the Simply Typed Lambda Calculus that takes the predecessor of a number.
Hence it is a rule that requires that the type of the expression it is acting on is also a number.
Therefore, the premise for this rule is that the expression is of type $Int$ which means that $pred\ e$ is of type $Int$.
A representation of this in JET is given in \autoref{lst:jetAdPredExpr}

\begin{lstlisting}[caption = Type rule for pred expression involving one premise, label=lst:jetAdPredExpr]
typerule TExprPred <- if {ctx} |- (Expr e) : TInt then 
    (Expr Pred e) : TInt return {return ()};
\end{lstlisting}

As can now be seen the type rule takes the appearance of an if-then statement in traditional imperative languages.
It can be thought of that the premise is a guard to the consequent, such that the consequent can only happen if we can prove the premise.
The type premise takes a similar form as the consequent of the type rule, however, there is once difference.
Just as a type rule may need to return some information about a change in the context, such as in the case of declarations, a type premise will need to be told what context we need to give to the type judgement such that we can prove it.
Similarly to the consequent, we can also specify that the type of the premise must match some expected type.
In this instance we are checking to make sure that the premise is of type integer such that we can say the consequent is of type integer.
Another piece of syntax to note is we have also given a parameter to the constructor that we wish to pattern match in the consequent.
This is so we can produce a valid constructor, but also so that we can perform premises on the paramters of the constructor, such as making sure that the expression for $pred$ is of type integer.

\subsubsection{Check Function}
When generating code for a check function involving a premise we want to make sure that we evaluate the premise before we return from the check function.
We do this by making the premise a monadic statement that is executed before the return of the check function.
Another reason why the premise is a monad is so that we can propagate the error back to the user in the event that the premise fails.
This means that no other code is required when handling the error case of the premise due to Haskell's do notation.
From these requirements we produce the code that can be seen in \autoref{lst:codeTExprPredCheck}.

\begin{lstlisting}[caption = Code generated for checkExpr from TExprPred, label=lst:codeTExprPredCheck]
checkExpr (Pred e) jetCheckType ctx = do
    var1 <- checkExpr e TInt ctx
    if jetCheckType == TInt then return () 
    else fail (makeCheckError (Pred e) jetCheckType TInt)
\end{lstlisting}

One thing to note in the code that wasn't already stated is that we store the result of the successful return of the premise in a variable so that it can be used later if necessary.
\todo{Rest of simply premise}

\subsection{More Complicated Premises}
\subsection{Side Conditions}
\subsection{Limitations}
\section{List Notation}
Many languages, in their abstract syntax, have nodes that take a list of nodes to be one of their children, an example of this can be seen as a sequence of statements as part of the Ad language in \autoref{appendix:witnessLanguages}.
It is currently possible to type check these sorts of type rules with the current set of features we have discussed, unfortunately, it requires the use of side conditions to define variables to allow us to define more conventional premises to do the rest of the type checking.
However, since dealing with lists of statements, or expressions, etc. are very common then it is worthwhile to add a set of features to the language that allows the user to define the type rules for lists of nodes.
The set of features should allow the user to create type rules handle that the different patterns of lists that the user may want to have type rules for: such as the empty list; singleton list and the cons of an item and the tail of a list.
The most useful pattern is probably the cons of an item and the rest of the list, represented in Haskell as \texttt{(x:xs)}.
This is because it allows the user to recursively type check an item and then the end of the list.
The other two patterns are useful when defining the base case of the recursive type rule, such as if the list is empty or if the list is the singleton list.
Being able to pattern match on a singleton list allows the user to assert within the type rule that the list cannot be empty.
The list notation can be used to both just perform a check, but also is able to return a type or a list of types.
Returning a list of types is useful when type checking a function or procedure call, as seen in the Ad language, to check that the types of expressions (given as an expression list) matches the expected types of the procedure or function.

\subsection{List Notation in Type Premises}
Using list notation in type premises is similar to how normal type premises are used.
The one difference is instead of encasing the premise with parentheses, we instead encase it within square brackets.
This so that it matches how lists are defined in Haskell, therefore, the users should be familiar with square brackets being associated with lists.
The example for the list of statements is given as \texttt{\{ctx\} |- [Stmnt stmnts] }.
As can be seen it is exactly the same is a normal type premise, with only the type of brackets used being different so that both the user and the tool can tell the difference.
The code that is produced from this type of premise is also very similar to the code produced from a normal premise: \texttt{var1 <- checkStmntList stmnts ctx}.
How the parameters are passed, along with how the result of the premise is bound to a variable, is exactly the same.
The only difference being is the function name has \texttt{List} as a suffix.
If we did not have this suffix then we would call \texttt{checkStmnt} which would lead to a type error as \texttt{checkStmnt} has the type: \texttt{Stmnt -> Context -> JetError a}, whereas the function we are trying to call would have to have type: \texttt{[Stmnt] -> Context -> JetError a}.
The difference being, one expects a statement, the other expects a list of statements.

\subsection{List Notation Type Rules}
We have discusses how list notation is used in premises, however, that is not useful unless code can be generated for type rules using list notation.
Type rules for list notation bear a resemblance to normal type rules, however, they do vary in sum fundamental ways.
They have the same style representing premises in an if-then block, however the consequent is expressed differently.
Similarly to premises, instead of being encased in parentheses they are encased in square brackets.
The major differences occur in the variables that are defined within the rule and what semantic meaning they have.
The simplest form is to just define the Node type and nothing else, for example: \texttt{[Stmnt]}.
This will pattern match on the empty list, in this case, when all statements have been successfully type checked.
The next pattern has the Node type and a variable name, given as: \texttt{[Stmnt s]}.
The pattern matched on the singleton list, where the element with in the list is given the variable name given in the type rule.
The final pattern that exists is the list constructor pattern, this, similar the the last two,  is given as the Node type but now followed by two variable names.
The first variable represents the head of the list to be type checked (the first item), the second variable represents the tail of the list to be type checked (The rest of the list after the head has been removed).
The variables in these type rules can be used the in the premises the same as any other variables.

\begin{figure}
    \begin{prooftree}
        \LeftLabel{StmntList: }
        \AxiomC{$\Gamma \vdash \lfloor stmnt \rfloor\ valid $}
        \AxiomC{$\Gamma \vdash \lfloor stmnts \rfloor \ valid$}
        \BinaryInfC{$\Gamma \vdash \lfloor stmnt;\ stmnts \rfloor\ valid$}
    \end{prooftree}
    \caption{Type rule for a sequence of statements in the Ad language}
    \label{fig:StmntListTypeRule}
\end{figure}

An example of how to define a type rule in list notation is when type checking a list of statements.
The type rule in natural deduction logic can be seen in \autoref{fig:StmntListTypeRule}.
As can be seen the premises in this type rule make a recursive type rule, and our the type rule in JET will have to reflect this meaning.
In the scenario given, the list statements must non-empty, we can create type rules that assert that the non-empty property must hold.
We therefore have the the type rule in JET that can be seen in \autoref{lst:jetStmntList}.

\begin{lstlisting}[caption = Type rule in JET for a seqeunce of statements, label=lst:jetStmntList]
typerule StmntListCons <- if {ctx} |- (Stmnt stmnt), 
        {ctx} |- [Stmnt stmnts] then 
    [Stmnt stmnt stmnts] return {Succ ()};
typerule StmntListSngltn <- if {ctx} |- (Stmnt stmnt),
    then [Stmnt stmnt] return {Succ ()};
\end{lstlisting}


\section{Utilising the Context}
Handling the context in JET is done through Haskell side condition to call functions that will get data from or insert data to the context.
As the user can write any code they want it may seem initiative as to what is the best way to interact with the context.
This is why the tool will provide the modules that can be seen in\todo{autoref to haskell modules appendix}, which will allow the user to instantiate a small number of Haskell type classes\todo{cite type classes} that will give the user a common interface with their defined context.
It is important to note that these modules do not define the context for the user, the user will have to define their own context type and instantiate the Haskell type classes for their own context type.
There is an example provided to the user of how to do this used Haskell Data Maps\todo{cite data maps} so that the user is not on their own in this regard.
\subsection{Empty Contexts and Blocks}
The most basic things you can do with a context is create an empty context or add a new block.
An empty context should do exactly what the name suggests, it should create a new context that has no items in it.
Since the user implements all the functions they should ensure that their function meets these semantic requirements.
We will assume that all languages will have a block structure, and if a language does not that this can be modelled by having a context which will only ever have one block.
How the context represents blocks is up to the user, however in our witness languages a single block is a map and our context is made up of a list of blocks.
So when we implement out new block function for our context we will prepend to the list a new empty map, this is to meet the requirement of the new block function that each new block should have no items in it.
Adding a new block is fairly simple with in the type rule, within the context definition of a type premise within a type rule you call the new block function on the context that you wish to add a new block in, for example:\\\texttt{{(newBlock ctx)} |- ($TypePremise$)}
this also produces very simply code where we would usually have the variable \texttt{ctx} we now have the code \texttt{(newBlock ctx)} giving us the whole type premise of\\\texttt{$variable$ <- $TypePremiseFunc$ $TypePremiseParams$ (newBlock ctx)}
\subsection{Variables}
The most common use of the context is when attempting to type check the use of variables within a program.
We will want to do two main actions with variables, the first is to retrieve the type of a variable from the context, the second is to add a new variable with a given type to the context.
Retrieving the type of a variable from the context is given by the natural deduction rule in \autoref{fig:varTypeRule}, as can be seen the only thing we need to do is to perform an existence check that the variable given is in the context, we also need to get the type of that variable from the context.
There is no way to express this meaning using normal type premises, instead we will have to use a Haskell side condition (similar to the side condition seen in the natural deduction rule).
However the consequent is very trivial and is the same as the ones we have seen for other expression which have arbitrary types associated with them, the consequent in \author{lst:jetSTLCIfExpr} is an example of the consequent required here.
We still need to retrieve the arbitrary type $\tau$ from the context.
We can use do this by using the function, given by the \texttt{JetContext} type class, \texttt{lookupContext}.
This function, given some a parameter to identify the item in the context (such as the variable name), will return the type associated with the input, or will return some fail message within the \texttt{JetError} monad.
When using this in a haskell side condition we will want to bind the resulting type from the context to the variable that we are going to use as the return type variable of the consequent.
If we say that the return variable of the consequent is \texttt{t}, then we will want to write the side condition as \texttt{\{t <- lookupContext var ctx\}}, where \texttt{var} is the variable to lookup and \texttt{ctx} is the given context to perform the lookup in.
We do not do anything special when generating the code from a Haskell side condition, we simply put it directly into the generated code in the position it occurs in the premise list (In this example it is the only premise so that does not matter just as long as it appears before the return monadic statement).

\begin{figure}
    \begin{prooftree}
        \LeftLabel{VarT: }
        \RightLabel{$[v : \tau \in \Gamma]$}
        \AxiomC{}
        \UnaryInfC{$\Gamma \vdash v : \tau$}
    \end{prooftree}
    \label{fig:varTypeRule}
    \caption{Type rule for using a variable in the Simply Typed Lambda Calculus and Ad languages}
\end{figure}

We can now retrieve types of variables from the context, however, that is not useful unless we can add variables to the context.
We will be using the example that can be seen in \autoref{fig:varDeclTypeRule} taken from the Ad language.
In this case we don't have any side conditions or premises, we only have the consequent which has the context $\Gamma$ but also has the variable and type added to the context.
We therefore need a way to represent this behavior in the return part of our type rule so that we can give this new context to the rest of the program that requires it.
We do this by writing the required Haskell at the end of the type rule after the return keyword.
The code written here is the code that will be return to the calling function so that it has all the required information from its child node, this is required when type checking a list of declarations.
In order to add the variable to the context will want to use the function, given by the \texttt{JetContext} type class, \texttt{expandContext} which takes in something to identify the item (the variable), the type to be associated and the context to add the item to.
This function may fail depending upon what the required semantics the user wants, for example a user may not want to be able to declare the same variable twice, and in other languages this may be allowed.
The user is able to define this functionality when defining the instance of the type class.
To use this function in the type rule we will want write something similar that was seen in the use of a variable but is slightly more complicated.
We will want to write \texttt{\{expandContext var t ctx\}}, this will add (if it is possible) the variable \texttt{var} with type \texttt{t} to the context given by \texttt{ctx}.
this gives us the whole type rule \texttt{(VarDecl var t) return \{expandContext var t ctx\}}.
where the \texttt{expandContext} function call replaces the \texttt{()} that we have used previously.

\begin{figure}
    \begin{prooftree}
        \LeftLabel{AdVarDecl: }
        \AxiomC{}
        \UnaryInfC{$\Gamma, v : \tau \vdash \lfloor v : \tau \rfloor\ valid$}
    \end{prooftree}
    \label{fig:varDeclTypeRule}
    \caption{Type rule for variable declaration in the Ad language}
\end{figure}

\subsection{Procedures and Functions}
Type checking functions in the Simply Typed Lambda Calculus is exactly the same as type checking variables as functions are first class citizens in the Simply Typed Lambda Calculus.
However, this is not the same in the Ad language (or in many other similar imperative languages such as C or Ada).

In order to type check procedures and functions we will want to discuss the notion of namespaces within JET.
Namespaces is a way that we can write our context in such a way that we can type check functions procedures and variables and be able to tell the difference between them based upon how they are used.
For example, in our Ad language we have to ways to call a subprogram (function or procedure), one is an expression therefore it requires a return type, the other is a statement and therefore requires not to have a return type.
From this we can identify in the expression call we need to check for a function, and in our statement call we need to check for a procedure.
We can also check for the difference between a function call and the use of a variable.
These will therefore all come under different "contexts" within our actual context, we will refer to these as namespaces to avoid the use of conflicting names.
We can interact with different namespaces by defined our new type, which will be our context identifier type.
In this type we will have multiple constructors, one for variables, one for function, and one for procedures, this allows us to tell the difference between the variables procedures and function within our JetContext instance and perform different actions accordingly.

Now we can discus how to type check functions and procedures.
They are very similar in structure and semantics, they both have a list of variable declaration to type check, they also both have a block structure as their body.
These consist of the premises.
We then want to add the whole subprogram declaration to the outer scope that it is contained in so that we can be used in the block it is declared in.
We will also want to add the subprogram declaration to the inner scope so that we can use the subprogram recursively.
So if we have a consequent \texttt{(ProcDecl PDecl ident vdecls decls stmnts)}, where ident is the name of the procedure, vdecls are the variable declaration for the parameters of the procedure, decls is the initial declaration block of the procedure, and stmnts is the statement block of the procedure.
We will want the type rule given in \autoref{lst:jetProcPremises}.

\begin{lstlisting}[caption = Jet premise list for procedures., label=lst:jetProcPremises]
If {let ts = map (\(VDecl _ t) -> t) vdecls},
    {ctx' <- expandContext (Proc ident ts) TNNone (updateRetType ctx TNNone)}
    {(newBlock ctx')} |- [VarDecl vdecls], 
    {var1} |- [Decl decls], 
    {var2} |- [Stmnt stmnts]
then (ProcDecl PDecl ident vdecls decls stmnt) return {return ctx'}
\end{lstlisting}

The first 2 premises are side conditions to define variables that can be used later on in the premise list or in the consequent.
The first simply gets the list of types from the list of variable declarations.
Next we add the procedure to the given context as this will be required in another type premise and in the consequent.
We then are able to type check the rest of the procedure declaration, the first thing we should do is type check the variable declaration to make sure that they are well formed before we can type check anything else, it is worth to note that here we add a new block to the context so as the parameters should belong to a new block.
We can then type check the list of normal declarations, with the new context returned by the variable declarations (this makes sure that we do not redeclare a parameter in the declaration list).
Finally we can check the body of the function, we give this the context given by the previous type premise, this is so it has visibility of all of the declarations and the parameters.
It will also have the procedure in the context so that we can recursively call the procedure.
We have already covered the consequent, the next thing to cover then is the return statement.
This is simply just to return the context that has the procedure declaration added to it.
We should not need to cover the generated code for this as we have discussed all the features that we have used to write this type rule previously, we are simply demonstrating how you would use those features to type check a procedure.]

Functions are more or less identical expect we have the return type of the function that goes in place of the \texttt{TNNone} placeholder.
We will also want to set the current return type so that we are type checking a return statement we can successfully check the type of the expression to return and the expected return type.

\subsection{Record Types and Type Synonyms}
\todo{I might just say that this is possible to do, but that the notation is not nice (Might therefore be better to go in evaluation.)}