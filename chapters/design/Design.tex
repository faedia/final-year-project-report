\chapter{Method}
Before we can generate type checking code we need to have a way of specifying the type system in such that a program can read the given specification and generate the required code.
The way we will be specifying the the type system is through a domain specified language I have designed called JET Expresses Types (JET).
The way type rules are written in JET are analogous to the natural deduction rules to define the type system.
This appeared to make the most sense as other parts of the compiler pipeline already have domain specific languages that are used to generate code.

In JET, rather than using the concrete syntax that is used in in the natural deduction rules, uses the abstract syntax provided by the user, to define the type rules.
Type rules in JET have a have a similar form to the natural deduction rules, such that there is a rule name followed by some judgments and then the consequent.
This provides a common way of representing type rules even if the way of writing the rules in JET use an ascii representation of the natural deduction logic.

The code we are generating is based off of the type checking and type inference algorithm described in Implementing Programming Languages\cite{ranta2012implementing}.
This is because the the algorithms described are simple yet powerful enough to express common type rules, such as arithmetic expressions and also allow for the use of functions and variables.
It also makes the code generation quite simple as there is a, more or less, direct translation of type rules to a function in the algorithm.

The code we will be generating will be Haskell code, since we can use the features in the Haskell language such as pattern matching and the use of monads to help produce relatively simple code from a given type specification.

Below is explained how the language is used, why certain design decisions were made and what code is generated from the input source. 
This will be demonstrated using a small number of  witness languages defined in \autoref{appendix:witnessLanguages} \todo{Simply Typed Lambda Calculus}.
The first language is a simple imperative language. 
The language consists of single procedure in which you can define more procedures and subprograms along with define variables. In the body of the procedure is a series of statements such as a block statement, if-else statement, and for subprograms a return statement. The language is similar to Ada but much simpler.
The next language is the simply typed lambda calculus\todo{Cite}, this will allow the demonstration of type checking higher order functions.

\section{The JET Language}

JET allows the user to write type rules that will be generated into Haskell code.
In order to do that the user may have to write some Haskell code in an initial section.
For this reason JET consists of two parts: some Haskell code in an initial section, followed by a list of type rules.

In order to write a type system in JET, the user will also be proved with a Haskell type class that defines a number of functions that will be useful when interacting with the context, such as looking up a variable in the context and adding an item to the context.
This type class has been written in such a way to support as many contexts as possible as to not limit what type of languages JET supports.
Along with the type class, there will also be an example of how to implement an instance of the type class using a Haskell map as an example.

\subsection{Initial Inline Haskell}
\section{Simple Type Rules}
\subsection{Null Statement Type Rule}
The simplest type rule that we can represent is similar to the one shown in \autoref{fig:adnullt} for our example of an imperative language.
This is a type rule with no premises to check, along with no types to check for the consequent.
Simply, if we have done a successfully parsed this statement, then it will always pass its type check.
Therefore the only code we need to generate is whatever needs to be returned back to the parent.
In this case we don't need to return anything, however there maybe cases in other languages where we would want to return the context back to the parent.

An example of rule such as this written in JET can be seen in \autoref{lst:jetAdNullT}
\begin{lstlisting}[caption = JET type rule for null statement, label=lst:jetAdNullT]
typerule AdNullT <- (Stmnt SNull) return {return ()}
\end{lstlisting}

The initial part \texttt{type AdNullT} has no affect on the rule is just a name of the rule for a use in future documentation.
The following part after the left facing arrow is the piece of abstract syntax that we wish to pattern match against.
This is analogous to the concrete syntax used to define the consequent in the natural deduction logic.
The abstract syntax definition takes the form of the type of abstract syntax we wish to check followed by the constructor we want to pattern match against, hence why for this example it is \texttt{Stmnt SNull} since we wish to check a statement and the specific statement is the null statement.
We need the AST node type name so that we can form the generated code correctly, we also need the constructor so that we can use Haskell's features of pattern matching to match against the correct piece of syntax.

From this type rule we produce the code that can be seen in \autoref{lst:codeAdNullT}.
we generate the name of the function based off of the AST node type name given.
Also notice that since the type rule contained no information about resulting type of the rule no code was produced to do any type checking.
This also means that there is no need to generate a function to infer the type given by this rule as there is no type associated with the rule, therefore, we do not generate the function to do such a type inference.
Therefore, the only thing the code does is pattern match against the piece of specified syntax and return an empty tuple since Statement rules in the language do not affect the context in any way.

\begin{lstlisting}[caption = Code generated from AdNullT, label=lst:codeAdNullT]
checkStmnt SNull ctx = do
    return ()
\end{lstlisting}

One possible improvement to be made in the code is instead of generating code of the form \texttt{foo astNode ctx} is to instead generate code to be of the form \texttt{foo ctx astNode}.
That way if we have multiple multiple instances of \texttt{foo ctx}, then we can give that expression a name and only evaluate it once.
However, there is one possible instance where having \texttt{ctx} be last argument has an advantage which can be seen at \todo{autoref to monad context binding}.

\subsection{Literal Expression Type Rule}
Both of our test languages have literal expressions.
Literal expressions are similar to our null statements example, in that there is not much that they type checker algorithm has to do.
This is because, similar to null statements, they have no premises, and do not affect the context in anyway.
The only difference between literal expressions and null statements is that literal expressions have types associated with them where as statements do not.

\begin{lstlisting}[caption = JET type rule for simple literal expressions, label=lst:jetAdLitExpr]
typerule AdTrueT <- (Expr ETrue) : TNBool return {return ()};
\end{lstlisting}

As can be seen in \autoref{lst:jetAdLitExpr}, we have specified something that was not specified in \autoref{lst:jetAdNullT}.
That is we have annotated the type rule with the type that is associated with the code we are checking.
In the case of a boolean \texttt{true} it has a type of boolean, where as a number has a type of integer.

Since this rule has a type associated, the code we must generate is now more complicated.
For example, we did not have to check the type of the null statement because it did not have a type.
We also did not have to generate a function to infer the type of the null statement because again, there is no type to infer.
However, since literal expressions do have types, we do have to check the type and provide an infer function to get the type of a literal expression.

The check function we generate, which can be seen in \autoref{lst:codeAdTrueTCheck}, should assert that the incoming type is, in this case, the \texttt{TNBool}, otherwise we should produce some form of error.
Errors take the form of a monad, in particular they are of type \texttt{JetError}, this monad is given to the user in a haskell module JetErrorM when they generate their code.
This allows us to propagate the error, along with the error message back out to the user without having to call the error function explicitly\todo{Cite Functional programming book}.
To produce the error message the generated code will call \texttt{makeCheckError} with the data that caused the error.
The user must then define the function \texttt{makeCheckError}, to keep things simple in the example cases \texttt{makeCheckError} is defined to return "Type Error".
However the user could write any code they liked as long it returned a String.

\begin{lstlisting}[caption = Code generated for checkExpr from AdTrueT, label=lst:codeAdTrueTCheck]
checkExpr ETrue jetCheckType ctx = do
    if jetCheckType == TNBool then return () 
    else fail (makeCheckError ETrue jetCheckType TNBool)
\end{lstlisting}

In \autoref{lst:altCodeAdTrueTCheck} I have given another piece of code that I could have generated that would have done the same thing.
It uses pattern matching on the input type as well as the expression, this means we do not require if expression within the function.
However we still do need the more generic function in order for us to be able to correct fail and call the \texttt{makeCheckError} function.

\begin{lstlisting}[caption = Alternate Code for checkExpr from AdTrueT, label=lst:altCodeAdTrueTCheck]
checkExpr ETrue TNBool ctx = then return () 
checkExpr ETrue jetCheckType ctx = 
    fail (makeCheckError ETrue jetCheckType TNBool)
\end{lstlisting}

In hindsight, this is a better to structure the code.
This is because under certain circumstances, such as the ones seen in\todo{autoref to premises}, it can reduce the amount of unnecessary evaluation that may be required due to generating the code with the if expression being the last monadic statement.

Infer functions return the type given by the type rule where as check functions assert that the type we are expecting is the same as the one given by the type rule.
From this, for example given in \autoref{lst:jetAdLitExpr}, we generate the code that can be seen in \autoref{lst:codeAdTrueTInfer}.
The code generated is very similar to the code seen in \autoref{lst:codeAdNullT}.
This is because there are no premises to check and we do not need to check the type specified by the type rule, we just need to return the type.
Therefore, the only thing this function does is return the type associated with the literal value $true$.

\begin{lstlisting}[caption = Code generated for inferExpr from AdTrueT, label=lst:codeAdTrueTInfer]
inferExpr ETrue ctx = do
    return TNBool
\end{lstlisting}