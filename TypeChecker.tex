\documentclass{UoYCSproject}

\usepackage{bussproofs}
\usepackage{float}
\usepackage{todonotes}
\usepackage{syntax}
\usepackage{listings}
\usepackage{amssymb}
\usepackage{caption}
\usepackage{subcaption}

\addbibresource{ref.bib}

\author{Steven M. Tomlinson}
\title{Generation of Type Checking Code}
\date{2018-September-21}
\supervisor{Jeremy L. Jacob}
\BEng

\dedication{For my mum}

\acknowledgements{
  I give thanks to my supervisor Jeremy Jacob for his continued help during the process of writing this project.
  I would also like to give thanks to my friends and family who have supported (and coped) throughout the project.
}

\lstset{
  aboveskip=0mm,
  belowskip=0mm,
  showstringspaces=true,
  columns=flexible,
  basicstyle={\footnotesize\ttfamily},
  breaklines=true,
  breakatwhitespace=true,
  tabsize=3,
  numbers=left
}
  % More definitions & declarations in example.ldf
  \begin{document}
  \pagenumbering{roman}
  \maketitle
  \renewcommand*{\lstlistlistingname}{List of Listings}
  \lstlistoflistings

  \listoffigures
  %\lstlistoflistings

  \begin{summary}
    Type systems are used to reduce the number of incorrect programs that a compiler will accept and generate code for.
    They are often the first line in defence against stopping bugs from being introduced into programs.
    For many years type systems have had a formal notation to specify them.
    There are a small number of tools that will generate code for type checkers, some of which have designed their notation such that it is akin to the formal notation.
    However, the feature sets that these tools support vary by a large amount, and many of these tools are can be complicated and not beginner friendly.
    We are therefore going to devising a notation that can describe type systems in a machine readable format.
    The notation should also be intuitive such that compiler writers not familiar with the formal notation type systems will still be able to understand the type systems written in our notation.
  
    The notation should be able to support simple languages from different programming paradigms, such as imperative and function paradigms.
    We should be able to support arithmetic and Boolean expressions, such as addition, equality, etc.
    There should be the facility to support the use and declaration of variables and subprograms.
    We should be able to support more complicated types than just integer and Boolean, such as array types and record types.
    However, given the time scale for the project we are not expected to be able to type check object orientated languages, such as Java,  and other languages of similar complexity, such as Haskell so only the language features discussed are required and any extra languages features are to be implemented if time permits.
    
    The notation we have devised is similar to the formal notation of type systems but its syntax is devised in such a way that it used programming constructs that developers should be familiar with, such as an if-then statement to assert properties within the type system and a return statement that can be used to return information.
    The notation differs from the notation used in other tools in a number of ways.
    Many other notations force the user to use a particular set of tools to perform lexical analysis and syntax analysis.
    The notation we have devised, and subsequent tool, do not have this issue because we use the abstract syntax definition of the language to both represent the type system in our notation and to perform the type checking in the generated code.
    The provides the user with the ability to use any parser and lexer they wish.
    The one stipulation we have, is the parser and lexer must be written in Haskell, or a tool that generates Haskell, this is because the target language (the output language of the generated code) of the code we generate is Haskell.
    We generate Haskell code so that we make use of some of the high level language features in Haskell which makes traversing abstract syntax trees trivial.
    The algorithms that we generate are based off of the algorithms discussed in \textcite{ranta2012implementing}.
    Our notation allows the use to write their own Haskell code so that they interact with the context.
    Interacting with the context is important because the context stores all the information about variables and subprograms, it may also contain user defined type information.
    We let the user define their own context so that we do not limit what kinds of languages we support.
    However, to help the user to define their context we have provided a Haskell type class with function signatures for common actions that would be performed on the context.
    The functions in these type classes are functions to make an empty context, add a new block to the context, lookup an item and retrieve its type from the context, and add a new item to the context with a given type.

    Our notation and tool is able to represent and generate code for two example languages that we have used as test cases to prove that we can type check all of the features that we have specified we should be able to type check.
    These two languages are very different, one is a simple imperative language, the other is a simple functional programming language.
    This demonstrates that the kinds of languages that our tool can represent and generate code for is diverse.
    However, our notation is far from perfect, it cannot type check every type of language, in particular it is unable to type check languages similar to ML and Haskell that have polymorphic type systems.
    The code we generate is also not as efficient as it could be, especially in an error case.
    There are also an number extensions we could make for the representation of the context and its given functions, especially in the use of its given functions.

    \subsubsection{Statement of Ethics}
    There are very few ethical considerations to make.
    We have not used any people or animals as part of the testing process for the tool.
    There have been no experiments performed during the development of the tool or in any of its subsequent testing.
    Any software that was used during the development of the tool or used as part of the discussing during this report has been credited to the relevant people, we have not claimed any of the software as our own.

    \end{summary}
\chapter{Introduction}
\todo{I will want to set the aims and objects of the project, along with the motivation}
\pagebreak
Test page
\chapter{State of the Art}
\label{chap:sota}
The idea of generating code to be a part of the compiler pipeline is not new.
The first two stages of the front-end of a compiler, the lexical analyser and the syntax analyser (more commonly known as the lexer and the parser), have general tools to generate their code using \textit{Domain Specific Languages} (DSL)\cite{Bentley:1986:PPL:6424.315691,van2000domain}.
Two of these languages are Flex\cite{Levine:2009:FB:1696439}, as the lexer generator, and Bison\cite{Levine:2009:FB:1696439}, as the parser generator.
These tools generate C code to be used to build the rest of a compiler, possibly including a type checker and machine code generation.
These programs have been around for many years, and many derivatives of them have been made to be used with different general purpose programming languages, such as Alex and Happy for Haskell or Jlex and Cup for Java\cite{ranta2012implementing}.
Many languages have been made using these programs and their respective DSL's, for example the parser for Haskell is self-generated by a grammar defined in a piece of Happy code.
Although, not all parts of the compiler pipeline have general tools that are widely used, some such tools do exists\cite{grimm2007typical,dijkstra2006ruler,Gray:1992:ECF:129630.129637}.
\section{Type Systems}
Before we can start discussing type checkers, we need to discuss the type systems which the checkers implement.

The concept and implementation of type systems has existed for almost as long as the concept of higher level language to perform computation\cite{Backus:1978:HFI:960118.808380}.
Many languages have a type system in one form or another with varying degrees of complexity.
Ranging from relatively simple type systems in languages such as C to fairly complex and expressive type systems such as ones in Haskell and Agda.

A type systems is a specification of a given programming languages type rules\cite{cardelli1996type}.
So from this we know that a type system is a specification and we can say that a type checker is an implementation of the specification.
Cardelli goes on to say that, we should separate the two, and that a type system is part of the language specification and the type checker is part of the compilers implementation of the language\cite{cardelli1996type}.
This is similar to the how we view the formal grammar is defined by the language, and the parser is how the compiler implements that formal grammar\cite{cardelli1996type}.

\subsection{Usefulness of Type Systems}
\todo{Need to research}


\subsection{Type Rules}

Much like how we have a formal representation for context free grammars as Backus-Naur form\cite{Backus1960,aho2003compilers,ranta2012implementing}, we also have a formal way of representing type systems and type rules.
In the literature, the standard method we have of representing type systems and type rules is through the use of natural deduction rules\cite{cardelli1996type,ranta2012implementing}.

Natural deduction rules have a form of: given some premises we conclude the consequent\cite{prawitz2006natural,ranta2012implementing}, for example in figure \ref{fig:generalNatDectRule} the premises are denoted as $J_k$ and the consequent is denoted as $C$.

\begin{figure}[H]
    \begin{prooftree}
        \LeftLabel{Rule Label:\quad}
        \RightLabel{$[Side\ Condition]$}
        \AxiomC{$J_1\ J_2\ \ ..\ \ J_n$}
        \UnaryInfC{$C$}
    \end{prooftree}
    \caption{General form of a natural deduction rule}
    \label{fig:generalNatDectRule}
\end{figure}

We can then use these Natural Deduction rules to specify the type rules in a given type system\cite{ranta2012implementing,cardelli1996type,}.
Figure \ref{fig:simpleTypeRule} is an example of how we would write a simple type rule for an arithmetic expression using the previously discussed natural deduction logic.
It is important to break this rule into its constituent parts to understand what it is trying to specify.
Starting from the consequent $\Gamma \vdash e_1 + e_2 : Int$, $\Gamma$ is the current context which we are applying the rule to.
The $\vdash$ is piece of syntax to say given the left hand side we can prove the right hand side.
Finally for the consequent $e_1 + e_2 : Int$ means that the piece of concrete syntax $e_1 + e_2$ must have type $Int$.
The top half of the rule are the things that must hold for the bottom half to be valid.\
So simply $e_1$ must be of type $Int$ given the context $\Gamma$, and the same is for the second premise\cite{cardelli1996type,ranta2012implementing}.

\begin{figure}[H]
    \begin{prooftree}
        \LeftLabel{Add Expression:\quad}
        \AxiomC{$\Gamma \vdash e_1 : Int$}
        \AxiomC{$\Gamma \vdash e_2 : Int$}
        \BinaryInfC{$\Gamma \vdash e_1 + e_2 : Int$}
    \end{prooftree}
    \caption{Simple arithmetic expression type rule}
    \label{fig:simpleTypeRule}
\end{figure}
%\section{Type Checkers}
\section{Type Checker Generators}
There have been multiple attempts to make the writing of type checkers easier on the compiler writers by creating new tools or notations.

\subsection{Attribute Grammars}
\subsubsection{Eli}

\subsection{Domain Specific Languages}
\subsubsection{Ruler}
\subsubsection{Typical}

\chapter{Design of the Notation}
\label{chap:Method}
Before we can generate type checking code we need to have a way of specifying the type system such that a program can read the given specification and generate the required code.
The way we will be specifying the the type system is through a domain specified language I have designed called \textit{JET Expresses Types} (JET).
The way type rules are written in JET are analogous to the natural deduction rules to define the type system.
This made the most sense as other parts of the compiler pipeline already have domain specific languages that are used to generate code.

In JET, rather than using the concrete syntax that is used in the natural deduction rules, we use the abstract syntax provided by the user to define the type rules.
We are using the abstract syntax for a number of reasons; the first reason is that we are going to be type checking input that has already been parsed therefore we are going to be type checking the abstract syntax tree and not the input of the concrete syntax.
The second reason, it is possible to type check an intermediate representation of a language.
Intermediate representations may only be represented through abstract syntax and have no concrete syntax, for example, Haskell's intermediate representation (known as core) is type checked during Haskell's compilation process\cite{marlow2004glasgow}.
Type rules in JET have a similar form to the natural deduction rules, such that there is a rule name followed by some judgments and then the consequent.
This provides a common way of representing type rules even if the way of writing the rules in JET uses an ASCII representation of the natural deduction rules.

The code we are generating is based off of the type checking and type inference algorithm described in \textcite{ranta2012implementing}.
This is because the algorithms described are simple, yet powerful enough to express common type rules, such as arithmetic expressions and also allow for the use of functions and variables.
It also makes the code generation quite simple as there is a direct translation of type rules to a function in the algorithm.
We will be generating Haskell code, since we can use the features in the Haskell language such as pattern matching and the use of monads to help produce relatively simple code from a given type specification.

\subsection{Example Languages}
\label{sec:exampleLanguages}
When discussing the features in JET, we will be discussing them in relation to two example languages.
These languages fulfil our aims for what language features we want JET to be able to type check given in \autoref{itms:aims}.
They also exhibit wildly different language features and paradigms so that we can demonstrate the diversity of languages that JET can type check.

The first of these languages is a language I have called Ad.
The language is a small imperative block structured language similar to Ada.
An Ad program is a series of type definitions followed by a procedure.
Procedures have a list of variable declarations as parameters.
Procedures consist of a list of declarations followed by a list of statements.
Declarations can introduce new variables or subprograms (collective term for a function or procedure).
A variable declaration consists of an identifier annotated with the type of the new variable, i.e. \texttt{x : Int}.
A function differs from a procedure in one way, a function returns a value whereas a procedure does not.
There are a number of built-in statements in the language, consisting of control flow statements; a statement to introduce a new block; an assignment statement; a procedure call statement and a print statement. 
The language also has a small number of built-in expressions: literal, arithmetic, Boolean, variable, and function call expressions.
Ad has two primitive types: integer and Boolean.
We are also able to express more complex types such as record and array types.
We can define type synonyms in Ad, type synonyms are similar to typedef's in C such that we can give a more complex type a name to reduce repeated type definitions in variable declarations.

Our second language is an implementation of the \textit{Simply Typed Lambda Calculus} (STLC).
STLC is a simple functional programming language.
A program in STLC consists of a single expression.
In the language we have access to simple arithmetic expressions; we then have more complicated expressions such as an if-expression and fix-expressions (this allows for recursive functions), we also have lambda abstraction (this creates an anonymous function that takes an expression as a parameter and has a single expression as a body), and function application that applies a given function with a given parameter.
The type system, while having the same primitive types as Ad, is quite different to the type system in Ad.
The major difference is that we have a function type which is a type that takes an input to an output, often represented as $\tau \rightarrow \sigma$ where $\tau$ is the type of the parameter and $\sigma$ is the type returned by the function.
This also means that the type system in STLC treats functions as "first-class citizens" as we can give functions as parameters to other functions and return new functions from a given function.
STLC is based off of the lambda calculus that is given in \textcite{pierce2002types}, they go into much more detail and have more extensions to STLC than we are going to consider here.

Full definitions of the abstract syntax and formal type systems for both of our example languages can be seen in \autoref{appendix:witnessLanguages}.
JET specifications for both of these languages are also provided.

\section{Overview of the JET Language}
The JET language has two main structural components: some Haskell code in an initial section\footnote{This is similar Flex and Bison's C code blocks to provide data structures and functions for the scanner and parser}, followed by a list of type rules.
The initial Haskell block is to allow the user to define functions and types that are to be used in the following type rules.

In order to write a type system in JET, the user will also have to provide a type which is an instance of a Haskell type class that defines a number of functions that will be useful when interacting with the context, such as looking up a variable in the context and adding an item to the context.
This type class has been written in such a way to support as many contexts as possible and not limit what kinds of languages JET can support.
Along with the type class, there is also be an example of how to implement an instance of the type class using a Haskell map as an example.

\subsection{Initial Inline Haskell}
The block of initial Haskell code, during code generation, will be placed just under the module definition but before the code for the type rules.
The inline Haskell code blocks take the form of $\{Code\}$, we chose a similar syntactic structure to code blocks in Flex and Bison (and their variants) so that the format will feel familiar to compiler writers.
As previously mentioned this is where a user can define functions and types required for the generated type checking code.
For example this is where a user can define the type for the context along with defining the instance of the type class \texttt{JetContext} for the users custom context type.
It also allows for the ability to import any other Haskell modules, this means that the user can import their Abstract Syntax Definition for that language that may be defined in a separate Haskell module.
An complete example for a block structured imperative language, similar to Add but with less features, can be seen in \autoref{lst:inlineHaskellCode}.
Here we define the context to be a 3-tuple which contains return type of the current subprogram, along with two namespaces, one for functions; and one for variables.
Namespaces, and their uses are discussed in detail \autoref{sec:context}.
These namespaces utilise the JetContextMap provided to the user.
JetContextMap is a list of Maps, where each Map is a scope in the code.
The instance of JetContext for JetContextMap is already defined for the user an example of how to define an instance of JetContext.

Having this inline code allows the user to specify exactly how they wish the context to function.
It also allows for a set of standard functions that the user can use to interact with the context, along with adding their own custom functions if and when they need them.

\subsection{Type Rule Format Overview}
\label{sec:typeRuleOverview}
The format of the type rules is extremely similar to the way we represent type rules in the natural deduction logic.
This allows for simple translation of a natural deduction type rule into one of the type rules written in JET.
We will delve into more detail about the format of the type rules, but the basic structure of the type rules is as follows

\begin{figure}[]
    \begin{grammar}
        <Rule> ::= `typerule' <Identifier> `<-' [ `if' <TypePremiseList> `then' ] <TypeConsequent> `return' <InlineHaskell>
    \end{grammar}
    \caption{BNF Grammar rule for defining a JET type rule}
    \label{fig:bnfTypeRuleBasic}
\end{figure}


The type premise list encased within the if-then is the list of judgments to prove the consequent.
It is possible for a type rule to require no judgments to prove the consequent so the if-then is optional, therefore, the type premise list can be forced to be nonempty.
The type premise after the if-then is the consequent we are trying to prove.
We then have a return keyword followed by some inline Haskell.
This is to allow the user to return any data that is required for type checking future AST Nodes.
For example, in our block structured language Ad, blocks are list of declarations followed by a list of statements.
When checking the declarations we will be adding new variables and subprograms to the context and when we finish checking the declarations we will want to return the context so that those variables and subprograms can be used in the statements.
If the context was not returned then those declared variables and subprograms would not be found in the context when they should have been.
The grammar of JET is designed to replicate as much as possible from the natural deduction rules as they are an already accepted way of formally representing type rules.
The formal grammar can be seen in its entirety in \autoref{appendix:jetLanguage}.

\section{Type Rules}
Type rules are how we express the type system for which we wish to generate the type checking code for.
Therefore, the structure and semantics of type rules will determine what language features JET can express and successfully generate valid code for the type checker.
We are going to be using the example languages seen in \autoref{sec:exampleLanguages} as motivation for the existence of features and their use.

\subsection{Basic Type Rules}
\label{sec:basisTypeRules}
\begin{figure}[]
    \centering
    \begin{subfigure}[b]{0.5\textwidth}
        \begin{prooftree}
            \LeftLabel{STLCIszero: }
            \AxiomC{$\Gamma \vdash e : Int$}
            \UnaryInfC{$\Gamma \vdash iszero\ e : Bool$}
        \end{prooftree}
        \caption{Type rule for iszero expression in STLC}
        \label{fig:iszeroTypeRule}
    \end{subfigure}
    ~
    \begin{subfigure}[b]{0.4\textwidth}
        \begin{prooftree}
            \LeftLabel{STLCInt: }
            \AxiomC{}
            \UnaryInfC{$\Gamma \vdash n : Int$}
        \end{prooftree}
        \caption{Type rule for integer literal expression in STLC}
        \label{fig:intTypeRule}
    \end{subfigure}
    ~
    \begin{subfigure}[b]{0.4\textwidth}
        \begin{prooftree}
            \LeftLabel{AdEqT: }
            \AxiomC{$\Gamma \vdash e_1 : \tau$}
            \AxiomC{$\Gamma \vdash e_2 : \tau$}
            \BinaryInfC{$\Gamma \vdash e_1 =  e_2 : Bool$}
        \end{prooftree}
        \caption{Type rule for equality expression in Ad}
        \label{fig:eqTypeRule}
    \end{subfigure}

    \caption{Examples of simple type rules}
    \label{fig:simpleTypeRulesExample}
\end{figure}

The type consequent is the most important part of the type rule, this is because the type consequent tells us exactly what part of the abstract syntax is being type checked along with the type associated with that piece of abstract syntax.
For example the type consequent for the type rule given in \autoref{fig:iszeroTypeRule} for STLC is:
\begin{lstlisting}[numbers=none]
(Expr Iszero e) : TBool
\end{lstlisting}
The consequent given represents exactly the same thing as the consequent in the natural deduction rule it simply just uses the abstract syntax that we have specified in \autoref{appendix:witnessLanguages}.
It is also worth to note that we associate the type consequent by writing a colon and the Haskell constructor or variable of the type we wish to associate with the consequent.
In this case it is the Haskell that identifies the Boolean type which is \texttt{TBool}.

Using the natural deduction rule in \autoref{fig:intTypeRule} we can create an example of our JET type rules and code that we generate from such an example.
The type rule is very simple, having no premises or side conditions, meaning that the type rule in our language should be just as simple:
\begin{lstlisting}[numbers=none]
typerule STLCInt <- (Literal LitInt n) : TInt 
    return {Succ ()};
\end{lstlisting}
There are three constituent parts to this type rule.
The first (\texttt{typerule STLCInt}) has no semantic meaning and is instead just an identifier for the type rule to be used for documentation purposes of the type system.
The second is the consequent of the type rule, taking a form similar to the previously discussed example, using the abstract syntax to pattern match on the literal expression for integer.
It is also annotated with the representation, in the abstract syntax, for integer in STLC.
The final part is the return statement.
This where any extra data, if necessary, is returned so that is can be used in the rest of the type checking, since this is just checking a literal integer there is no extra information that is useful from this type rule so we just return an empty tuple.
The type checking function we generate has to assert than the expected type is the same as the type associated with out type rule.
There are a number of way to do this in Haskell, the way in which we do it is within an if-expression in Haskell:
\begin{lstlisting}
checkLiteral (LitInt n) jetCheckType ctx = do
    if jetCheckType == TInt then Succ () 
    else Fail (makeCheckError (LitInt n) jetCheckType TInt)
\end{lstlisting}
The first thing you may notice is the name of the function, its suffix is the same as the first identifier in the consequent.
This is because the first identifier in the consequent must always be the type name of AST type we are type checking, in this case we are checking a \texttt{Literal}, in other cases we could be checking an \texttt{Expr} in which case we would replace that first identifier in the consequent.
This allows us to pattern match against any constructor of a given AST node. For example if we needed to type check against an arbitrary expression in our Ad language, that expression could be a literal expression or an addition expression or any other expression within the language.
If we were to have unique function for each possible kind of expressions, i.e. \texttt{checkEAdd}, then we could not match against any arbitrary expression, we would need to know the kind of expression before we can call a function.
However, if each expression is within the same function, then we can use Haskell's pattern matching features to identify what kind of expression it is and then execute the relevant type checking code for that kind of expression.
From this we can generalise the type consequent so that it has the format of \texttt{HaskellType HaskellTypeConstructor}.

The first parameter (\texttt{LitInt n}) of the function is the Haskell type constructor specified in the consequent, this allows for the required pattern matching so that we can call the correct type checking code.
The second parameter (\texttt{jetCheckType}) is the expected type to be associated with this type rule, this is the parameter that we will perform the equality check with when asserting whether or not the actual and the expected type match.
Finally, the last parameter it the current context which we are type checking in.
In this case the type rule does not interact with the context, however, more complicated type rules may interact with the context so it is required to be a parameter to every type checking function so that is never lost when performing the type checking.

Since this type rule has no premises the body of the function only has to check that the expected and actual type match.
If they do then we succeed and return whatever the inline Haskell was in the return statement of the type rule.
If not then we fail and generate some error message from a user defined error function.

We have discussed how to generate a type check function, however that is not the only function that is required. 
The infer function is used to return the type associated with a type rule so that the type can be used in other areas of the type checking.
For example, in the type rule show in \autoref{fig:eqTypeRule} we have two type premises both with type $\tau$, in this case $\tau$ is an arbitrary type but both instances of $\tau$ are the same.
Therefore we need a function that we can call so we can the type of piece of syntax, more commonly this is known as type inference\cite{cardelli1996type}.
Going back to our previous example (to see examples of type rules with premises go to \autoref{sec:premiseTypeRules}), we wish to generate code that will return the type associated with our type rule.
In this case we wish to return the type TInt, in this case we produce the following code:
\begin{lstlisting}[numbers=none]
inferLiteral (LitInt n) ctx = do Succ TInt
\end{lstlisting}
The function signature for our infer function is similar to the signature for our check function.
The difference is that we no longer require the \texttt{jetCheckType} parameter as we are no longer checking the type associate with our type rule matches some expected type, instead we are returning the type associated with our type rule.\todo{This sentence needs a rewrite}
Therefore, our function does not require the if-expression, instead we just wish to return the type \texttt{TInt}.

\subsection{Typeless Type Rules}
\label{sec:tltr}
\begin{figure}[]
    \begin{prooftree}
        \LeftLabel{AdNullT: }
        \AxiomC{}
        \UnaryInfC{$\Gamma \vdash \lfloor null \rfloor\ valid$}
    \end{prooftree}
    \caption{Type rule for a null statement in the Ad language}
    \label{fig:nullStmntTR}
\end{figure}

Some languages require that every type rule must have a type associated with it, STLC is an example of such a language; however, this is not the case for all languages.
For example, in our Ad language only expressions have types associated with them, every other syntactic part of the language is simply referred to as $valid$ if all of the premises of the type rule are true.
The simplest of these type rules is the null statement in the Ad language seen in \autoref{fig:nullStmntTR}, this is a simple type rule where if it is successfully parsed it will always be true as it has no premises.
This type rule given in JET is represented as:
\begin{lstlisting}[numbers=none]
typerule AdNullT <- (Stmnt SNull) return {Succ ()}
\end{lstlisting}

Here, unlike in previous type rules we have seen in \autoref{sec:basisTypeRules}, we have not specified the type associated with the type rules at all, it has been completely omitted.
This represents the concept of marking something as $valid$ in the natural deduction rules.
The code we generate is also substantially different (and much simpler) as we no longer need to check the type as there is no type, we also no longer need to generate an infer function to return a type as again there is no type to return:
\begin{lstlisting}[language=Haskell, numbers=none]
checkStmnt SNull ctx = do Succ ()
\end{lstlisting}
You may also notice that the parameter \texttt{jetCheckType} has also been omitted as there is no need for it since there is no type to check.

We could get the same functionality without this feature by having an internal type that represents $valid$.
In the Ad language we could \texttt{TNNone} for this purpose giving us the type rule:
\begin{lstlisting}[escapechar=\#, numbers=none]
typerule AdNullT <- (Stmnt SNull) #\textbf{: TNNone} return {Succ ()}
\end{lstlisting}
However, this is redundant as we are generating more code than necessary because we now have a type to check, where as before we did not.
This could potentially lead to a loss in performance since we are now having to perform an equality check and also have to give the \texttt{jetCheckType} parameter to the check function.
We will also generate an infer function since there is now the possibility to infer the type of such a statement where as in our original type rule we did not.

\subsection{Type Premises}
\label{sec:premiseTypeRules}

\begin{figure}[]
    \centering    
    \begin{subfigure}{0.6\textwidth}
        \begin{prooftree}
            \AxiomC{$\Gamma \vdash e_1 : Bool$}
            \AxiomC{$\Gamma \vdash e_2 : \tau$}
            \AxiomC{$\Gamma \vdash e_3 : \tau$}
            \TrinaryInfC{$\Gamma \vdash if\ e_1\ then\ e_2\ else\ e_3 : \tau$}
        \end{prooftree}
        \caption{Type rule for STLC if-expression}
        \label{fig:ifExprTypeRule}
    \end{subfigure}
    ~
    \begin{subfigure}{1\textwidth}
        \begin{prooftree}
            \AxiomC{}
            \RightLabel{Int}
            \UnaryInfC{$\Gamma \vdash 0 : Int$}
            \RightLabel{Iszero}
            \UnaryInfC{$\Gamma \vdash iszero\ 0 : Bool$}
            \AxiomC{}
            \RightLabel{Int}
            \UnaryInfC{$\Gamma \vdash 0 : Int$}
            \RightLabel{Pred}
            \UnaryInfC{$\Gamma \vdash pred\ 0 : Int$}
            \AxiomC{}
            \RightLabel{Int}
            \UnaryInfC{$\Gamma \vdash 0 : Int$}
            \RightLabel{Pred}
            \UnaryInfC{$\Gamma \vdash pred\ 0 : Int$}
            \RightLabel{Pred}
            \UnaryInfC{$\Gamma \vdash pred(pred\ 0) : Int$}
            \TrinaryInfC{$\Gamma \vdash if\ iszero\ 0\ then\ pred\ 0\ else\ pred\ (pred\ 0)$}
        \end{prooftree}
        \caption{Proof tree example for a STLC expression (Type rule names have been abbreviated)}
        \label{fig:proofTree}
    \end{subfigure}
    \caption{Type rule for if-expression and corresponding proof tree example}
\end{figure}

So far we have only discussed in detail type rules without premises.
However, for us to be able to type check and traverse the whole abstract syntax tree given to us by a parser we need to generate code that uses premises.
For example, if we use the natural deduction rule given in \autoref{fig:ifExprTypeRule}, the type rule we write in JET will need to represent the premises given in this type rule.
The code we generate will also want to call the code that checks all of the premises given for the type rule.
Considering the code snipped for STLC \texttt{if iszero 0 then pred 0 else pred (pred 0)} we get the proof tree seen in \autoref{fig:proofTree}.
We can see that the method of proving a sentence in a language is through recursive application of type rules until we reach a type rule that has no premises causing the proof to terminate, in this case the literal integer type rule acts as our base case for all of our premises.
The type rules we have already discussed are able to act as base cases to the recursive nature of type checking that we wish to generate code for.
Therefore, the code we wish to generate for type premises will need to replicate this recursive nature we desire.

Type premises have a similar syntax to type consequents.
More information that is required for a type premise, namely the context which the type premise is meant to be proved with respect to, for now we will be looking at type rules that do not interact with the context (go to \autoref{sec:context} to see type rules that interact with the context).
This gives means the general form of the type premise is:
\begin{grammar}
<TypePremise> ::= <InlineHaskell> `|-' `(' <Identifier> <Identifier> `)' [`:' <Type>]
\end{grammar}
Here, the first $Identifier$ is the type of the AST node for the premise; the second $Identifier$ is the variable that the premise is meant to check; $:\ Type$ is the optional type that is associated with the type premise.
We are now in a position to write a JET type rule for our type rule given in \autoref{fig:ifExprTypeRule}:
\begin{lstlisting}
typerule IfExpr <- if {ctx} |- (Expr e1) : TBool, 
        {ctx} |- (Expr e2) : t, 
        {ctx} |- (Expr e3) : t then 
    (Expr If e1 e2 e3) : t return {Succ ()}
\end{lstlisting}
We can see that a type premise list is a comma separated sequence of type premises.
This also the first time we have seen a JET type variable being used in a type rule.
There are a few sematic rules about type variables.
If we have said that the type of a premise is a type variable and this is the first occurrence of the type variable then we will be generating an infer function to get the value that the type variable is to represent.
Again, if we have said that the type of a premise is a type variable, but this time we have seen the type variable we will generate a check function where the type to check is the already defined type variable, this is to assert the two type variables are consistent.
Finally, if we use a type variable in the consequent then the variable should have been defined some where in the type premises.
From this set of rule we end up generating the following code:
\begin{lstlisting}
checkExpr (If e1 e2 e3) jetCheckType ctx = do
    var1 <- checkExpr e1 TBool ctx
    t <- inferExpr e2 ctx
    var2 <- checkExpr e3 ctx
    if jetCheckType == t then Succ () else Fail (makeCheckError (If e1 e2 e3) jetCheckType t)
inferExpr (If e1 e2 e3) jetCheckType ctx = do
    var1 <- checkExpr e1 TBool ctx
    t <- inferExpr e2 ctx
    var2 <- checkExpr e3 ctx
    Succ t
\end{lstlisting}
As can be seen, the final line of both functions are similar to the bodies of functions that we have seen that did not have type premises.
The other thing to notice, is that the code generated for the type premises appears in both of our generated functions, this is because in both instances we need to prove premises otherwise anything we state about the consequent could be incorrect.
Premises are evaluated in the order of left to right in the premise list, and the return value from check functions is stored in a variable with name $var_n$ where n is the next free natural number such that the variable name is unique.

\subsubsection{Type Variables as Parameters to Type Constructors}
\begin{figure}[]
    \begin{subfigure}{0.5\textwidth}
        \begin{prooftree}
            \LeftLabel{ArrAcc: }
            \AxiomC{$\Gamma \vdash e : Arr~\tau$}
            \AxiomC{$\Gamma \vdash n : Int$}
            \BinaryInfC{$\Gamma \vdash e[n] : \tau$}
        \end{prooftree}
    
        \caption{Type rule for array access in Ad}
        \label{fig:typeRuleArrAccess}
    \end{subfigure}
    ~
    \begin{subfigure}{0.4\textwidth}
        \begin{prooftree}
            \LeftLabel{Fix: }
            \AxiomC{$\Gamma \vdash e : \tau \rightarrow \tau$}
            \UnaryInfC{$\Gamma \vdash fix~e : \tau$}
        \end{prooftree}

        \caption{Type rule for fix point operator in STLC}
        \label{fig:typeRuleFix}
    \end{subfigure}

    \caption{Type rules with type variables as part of type constructors}
    \label{fig:complicatedTypeRules}
\end{figure}

We can also use type variables as parameters to the constructor of a full type definition.
Such examples of these types are array types in the Ad language, and a function type for STLC.
All elements in an array, in Ad, are all of the same type, therefore the type rule getting an element of the array will need to extract the type of the elements of the array.
In the natural deduction rules this is trivial, you represent arrays as $Arr~\tau$ where $\tau$ is the type of the elements of the array.
We can use a similar form of representing such types in JET.
For example, the type rule for array access in Ad is seen in \autoref{fig:typeRuleArrAccess}.
We can write the first type premise in JET in almost exactly the same way:
\begin{lstlisting}[numbers=none]
{ctx} |- (Expr e) : TArr t 
\end{lstlisting}
From this we can to generate some code of the form:
\begin{lstlisting}[numbers=none]
TArr t <- inferExpr e ctx
\end{lstlisting}
This code asserts that the type returned from the infer function must be an array type, we then get the parameter of the array type and store it in some variable t.
The generator will also have to add the variable t to the list of known type variables so that if another premise required the use of this type variable then we can generate a check function instead of an infer function.
There is one issue with the code that we generate at the moment.
If the return of the infer function does not match the pattern function then the type checker will error, but not with our defined error function since the error was an internal Haskell error.
Instead the right hand side of the binding will want to call our error function in case of a bad pattern match, from this criteria we produce the follow code:
\begin{lstlisting}[numbers=none]
TArr t <- case inferExpr e ctx of Succ jet0@(TArr t) -> Succ jet0; Succ t -> Fail (makeInferError t t); x -> x
\end{lstlisting}

\subsubsection{Side Conditions as Premises'}
\label{sec:scond}
There is another example of such a type rule, but instead this is part of STLC, can be seen in \autoref{fig:typeRuleFix}.
As can be seen there is the type $\tau \rightarrow \tau$ where both instances of $\tau$ are the same.
We would like to write the type premise as:
\begin{lstlisting}[numbers=none]
{ctx} |- (Expr e) : TFun t t
\end{lstlisting}
Where we assert that both instances of \texttt{t} are the same.
However the code we generate for this is not correct as we do not assert that both instances of \texttt{t} are the same, instead \texttt{t} is always the value of the second parameter to the type constructor.
Therefore, we write the type premise as:
\begin{lstlisting}[numbers=none]
{ctx} |- (Expr e) : TFun t1 t2
\end{lstlisting}
We also write a side condition premise such that asserts that \texttt{t1} and \texttt{t2} are the same and creates a new type variable \texttt{t} which is the same as \texttt{t1} and \texttt{t2}.
A side condition premise is piece of inline Haskell that is inserted into the generated code as a monadic statement, the used of the side conditions can be extremely varied in functionality and can do anything that the user requires that cannot be done as part of normal type premises.
One of the major uses of Haskell side conditions is for retrieving or expanding the context that we will see in \autoref{sec:context}.
Giving us the full type rule as follows:
\begin{lstlisting}
typerule Fix <- if {ctx} |- (Expr e) : TFun t1 t2, 
        {if t1 == t2 then Succ t1 else Fail "type error"} then 
    (Expr Fix e) : t1 return {Succ ()};
\end{lstlisting}

\section{List Notation Type Rules}
Many languages, in their abstract syntax, have nodes that take a list of nodes to be one of their children, an example of this can be seen as a sequence of statements that make up the body of a procedure or function as part of the Ad language in \autoref{appendix:witnessLanguages}.
In the set of features discussed there is no simple way to represent type rules that act on these kinds of nodes.

\subsection{List Notation in Type Premises}
Using list notation in type premises is similar to how normal type premises are used.
The one difference is instead of encasing the premise with parentheses, we instead encase it within square brackets.
This so that it matches how lists are defined in Haskell, therefore, the users should be familiar with square brackets being associated with lists.
The example for the list of statements is given as \texttt{\{ctx\} |- [Stmnt stmnts]}.
As can be seen it is exactly the same is a normal type premise, with only the type of brackets used being different so that both the user and the tool can tell the difference.
The code that is produced from this type of premise is also very similar to the code produced from a normal premise: \texttt{var1 <- checkStmntList stmnts ctx}.
How the parameters are passed, along with how the result of the premise is bound to a variable, is exactly the same.
The only difference being is the function name has \texttt{List} as a suffix.
If we did not have this suffix then we would call \texttt{checkStmnt} which would lead to a type error as \texttt{checkStmnt} has the type: \texttt{Stmnt -> Context -> JetError a}, whereas the function we are trying to call would have to have type: \texttt{[Stmnt] -> Context -> JetError a}.
The difference being, one expects a statement, the other expects a list of statements.

\subsection{List Notation Type Rules}
We have discusses how list notation is used in premises, however, that is not useful unless code can be generated for type rules using list notation.
Type rules for list notation bear a resemblance to normal type rules.
They have different semantics for the consequent but the type premises have the same semantics we have previously discussed.
Similarly to premises, instead of being encased in parentheses they are encased in square brackets.
The major differences occur in the variables that are defined within the rule and what semantic meaning they have.
The simplest form is to just define the Node type and nothing else, for example: \texttt{[Stmnt]}.
This will pattern match on the empty list, in this case, when all statements have been successfully type checked.
The next pattern has the Node type and a variable name, given as: \texttt{[Stmnt s]}.
The pattern matched on the singleton list, where the element with in the list is given the variable name given in the type rule.
The final pattern that exists is the list constructor pattern, this, similar the the last two,  is given as the Node type but now followed by two variable names.
The first variable represents the head of the list to be type checked (the first item), the second variable represents the tail of the list to be type checked (The rest of the list after the head has been removed).
The variables in these type rules can be used the in the premises the same as any other variables.

\begin{figure}
    \begin{prooftree}
        \LeftLabel{StmntList: }
        \AxiomC{$\Gamma \vdash \lfloor stmnt \rfloor\ valid $}
        \AxiomC{$\Gamma \vdash \lfloor stmnts \rfloor \ valid$}
        \BinaryInfC{$\Gamma \vdash \lfloor stmnt;\ stmnts \rfloor\ valid$}
    \end{prooftree}
    \caption{Type rule for a sequence of statements in the Ad language}
    \label{fig:StmntListTypeRule}
\end{figure}

An example of how to define a type rule in list notation is when type checking a list of statements.
The type rule in natural deduction logic can be seen in \autoref{fig:StmntListTypeRule}.
The premises in this type rule make a recursive type rule, and our the type rule in JET will have to reflect this meaning.
In the scenario given, the list statements must non-empty, we can create type rules that assert that the non-empty property must hold.

\begin{lstlisting}[caption = Type rule in JET for a seqeunce of statements, label=lst:jetStmntList]
typerule StmntListCons <- if {ctx} |- (Stmnt stmnt), 
        {ctx} |- [Stmnt stmnts] then 
    [Stmnt stmnt stmnts] return {Succ ()};
typerule StmntListSngltn <- if {ctx} |- (Stmnt stmnt),
    then [Stmnt stmnt] return {Succ ()};
\end{lstlisting}

We therefore have the the type rule in JET that can be seen in \autoref{lst:jetStmntList}.
As can be seen to create the behavior we desire need two type rules, one for when we have a list arbitrary length, and another for when the list contains a single item.
This structure forces the list to be nonempty and the type checker will error if the list happens to be empty.
Another important aspect to note is that we can recursively call out same type rule in order to type check the while list.
This can be seen in the list constructor type rule where we type check the current statement but then also go on to type check the rest of the list.
These patterns should be able to handle any usual type rule that a user is going to write.
However they do not cover every possible pattern that could occur as they do not replicate how Haskell does pattern matching on lists but instead uses a subset of the patterns that Haskell has available.
It could be possible in future to expand these patterns so that they do match all of the patterns that Haskell can handle, however we did not require this feature to type check the witness languages given.

\section{Interacting with the Context}
\label{sec:context}
Using side conditions to update and lookup items within the context is very common when specifying the natural deduction rules.
We have discussed in \autoref{sec:scond} how side conditions work in JET and how the code is generated for them.
To make interacting with the context through Haskell side conditions easier and more consistent, there is a Haskell type class that has been provided to the user to give them a common interface to interact with the context\footnote{The definition of this type class (called \texttt{JetContext}) is provided in ref to haskell modules!!!!!!!!}.
It is important to note that these modules do not define the context for the user, the user will have to define their own context type and instantiate the Haskell type class for their own context type.
This is because many different languages can have contexts that work differently and letting the user define their own context gives them greater control on the semantics of the context.
There is an example provided to the user of how to do this using Haskell Data Maps\todo{cite data maps} so that the user is not on their own in this regard, the name of the type for this example is \texttt{JetContextMap}.
The semantics of the each function will be explained using the provided example.

\subsection{Empty Contexts and Blocks}
\begin{lstlisting}[language=ada, caption=Example of declarations in a block structured language such as Ad, label=lst:blockDecl]
declare
    x : int; -- This will succeed
begin
    declare
        x : bool; -- This will also succeed because the variable is re-declared in a new block
        x : int; -- This will fail because the variable was re-declared in the same block as the previous declaration
    begin
    end;
end;
\end{lstlisting}

The basic thing you can do with a context is create an empty context or add a new block to the context.
An empty context should do exactly what the name suggests, it should be a new context that has no items in it.
Since the user implements all the functions they should ensure that their function meets these semantic requirements.
There is an assumption that all languages will have a block structure even if that block structure is trivial, and if a language is not a block structured language it can be modelled by having a context which will only ever have one block.
How the context represents blocks is up to the user, however in our \texttt{JetContextMap} example a single block is a map and our context is made up of a list of blocks.
The reason for this is simple, consider the Ad program given in \autoref{lst:blockDecl}.
We want to be able to re-declare variables as long as the variable to re-declare does not exist in the current block.
If the context was just a single map then there would be no way of knowing whether or not a variable was declared in the current block, we would only be able to know if a variable was declared in any block.
This gives us the formal definition of the context to be a sequence of functions which maps names to types given as $\langle Name \rightarrow Type \rangle$

\begin{figure}[]
    \centering
    \begin{prooftree}
        \AxiomC{$\langle \emptyset \rangle ^\frown \Gamma \vdash \lfloor decls \rfloor valid$}
        \AxiomC{$\langle \{decls\} \rangle ^\frown \Gamma \vdash \lfloor stmnts \rfloor~valid$}
        \BinaryInfC{$\Gamma \vdash \lfloor$ declare $decls$ begin $stmnts$ end $\rfloor~valid$}
    \end{prooftree}
    \caption{Type rule for a new block in the Ad language}
    \label{fig:typeRuleBlock}
\end{figure}

These functions can be used anywhere inline Haskell can be used.
However, they will most likely be used in the context definition for type premise's.
An example of natural deduction rule defining a new block for the Ad language can be seen in \autoref{fig:typeRuleBlock}.
Our first premise states, given a sequence containing an empty set (conceptual a mapping of names to types containing nothing), concatenated with our current context check our declarations.
The second premise states check the statements of the block given a new block containing all of our declarations concatenated with the current context.
The following type rule in JET represents this natural deduction rule:
\begin{lstlisting}
typerule BlockT <- if {(newBlock ctx)} |- [Decl decls],
        {var1} |- [Stmnt stmnts] then
    (Stmnt SBlock decls stmnts) return {Succ ()};
\end{lstlisting}
We use \texttt{var1} for the context to the second premise because \texttt{var1} is the context returned from the first premise which has all of the declarations added to the context.
This is required so that those declarations are visible to the statements in the block.

\subsection{Lookup, Expand}
\begin{figure}[]
    \centering
    \begin{subfigure}{1\textwidth}
        \begin{prooftree}
            \LeftLabel{VarT: }
            \RightLabel{$[\tau = lookup(v, \Gamma)]$}
            \AxiomC{}
            \UnaryInfC{$\Gamma \vdash v : \tau$}
        \end{prooftree}
        \caption{Type rule for using a variable in the Simply Typed Lambda Calculus and Ad languages}
        \label{fig:varTypeRule}
    \end{subfigure}
    ~
    \begin{subfigure}{1\textwidth}
        \begin{prooftree}
            \LeftLabel{AdVarDecl: }
            \AxiomC{$(head (\Gamma) \oplus {v : \tau}) ^\frown (tail (\Gamma)) \vdash \lfloor decl \rfloor$}
            \RightLabel{$[v \notin dom((head(\Gamma)))]$}
            \UnaryInfC{$\Gamma \vdash \lfloor v : \tau; decls \rfloor\ valid$}
        \end{prooftree}
        \caption{Type rule for variable declaration in the Ad language}
        \label{fig:varDeclTypeRule}
    \end{subfigure}
    ~
    \begin{subfigure}{1\textwidth}
        \begin{prooftree}
            \LeftLabel{AdVarDecl: }
            \AxiomC{$expandContext(v,\tau,\Gamma) \vdash \lfloor decls \rfloor$}
            \UnaryInfC{$\Gamma \vdash \lfloor v : \tau; decls \rfloor\ valid$}
        \end{prooftree}
        \caption{Type rule for variable declaration in the Ad language using functions from the Haskell type class}
        \label{fig:modvarDeclTypeRule}
    \end{subfigure}
    \caption{}
    \label{}
\end{figure}

The most common use of the context is when attempting to retrieve types of objects from the context.
We will want to do two main actions with variables, the first is to retrieve the type of a variable from the context, the second is to add a new variable with a given type to the context.
Retrieving the type of a variable from the context is given by the natural deduction rule in \autoref{fig:varTypeRule}.
The consequent is trivial and is the same as the ones we have seen throughout \autoref{sec:basisTypeRules}.
We still need to retrieve the arbitrary type $\tau$ from the context.
this can be done by using the function, given by the \texttt{JetContext} type class, \texttt{lookupContext}.
This function, given some a parameter to identify the item in the context (such as the variable name), will return the type associated with the input, or will return some fail message within the \texttt{JetError} monad.
When using this in a Haskell side condition we will want to bind the resulting type from the context to the variable that we are going to use as the type for the consequent.
If we say that the return variable of the consequent is \texttt{t}, then we will want to write the side condition as \texttt{\{t <- lookupContext var ctx\}}, where \texttt{var} is the variable to lookup and \texttt{ctx} is the given context to perform the lookup in.
We do not do anything special when generating the code from a Haskell side condition, we simply put it directly into the generated code in the position it occurs in the premise list (In this example it is the only premise so that does not matter just as long as it appears before the return monadic statement).

We can now retrieve types from the context, however, that is not useful unless we can add objects to the context.
We will be using the example that can be seen in \autoref{fig:varDeclTypeRule} taken from the Ad language.
The type rule has a very complicated predicates for the new context and also for the side condition for the type rule.
However, the functionality can be encoded into the \texttt{expandContext} function.
The \texttt{expandContext}, takes as parameters the object, the type to associate with the object, and the context to add the object to.
The function will return an error monad, therefore if the function is successful it will return the new context otherwise it will return an error.
If we were to then use this function to write the natural deduction rule again we would get the rule given in \autoref{fig:modvarDeclTypeRule}.

The new version of the natural deduction rule is now much easier to write in JET because we can use the function from our type class.
We write the type rule in JET as the following:
\begin{lstlisting}[numbers=none]
typerule VDeclT <- (VarDecl VDecl var t) return {expandContext var t ctx};
\end{lstlisting}
There is no need for the premise in our JET type rules as we write that using the list notations, this type rule is simply to add the variable to the context therefore we need to return the new context which has been expanded with the variable rather than returning the Unit type that we have seen in every other type rule so far.
The code we produce for this is similar to the typeless type rules seen in \autoref{sec:tltr} but with the expand function in place of \texttt{Succ ()}:
\begin{lstlisting}
checkVarDecl (VDecl var t) ctx = do
    expandContext var t ctx
\end{lstlisting}

\subsection{Name-Spaces}
Type checking functions in the Simply Typed Lambda Calculus is exactly the same as type checking variables as functions are first class citizens in the Simply Typed Lambda Calculus.
However, this is not the same in the Ad language (or in many other similar imperative languages such as C or Ada).

In order to type check procedures and functions we have used the concept of namespaces from \textcite{grimm2007typical}.
Namespaces are a way of segregating identifiers based upon their declaration and use.
For example, in our Ad language we have to ways to call a subprogram (function or procedure), one is an expression therefore it requires a return type, the other is a statement and therefore requires not to have a return type.
From this we can identify in the expression call we need to check for a function, and in our statement call we need to check for a procedure.
We can also check for the difference between a function call and the use of a variable.
These will therefore all come under different "contexts" within our actual context, we will refer to these as namespaces to avoid the use of conflicting names.
We can interact with different namespaces by defined our new type, which will be our context identifier type.
In this type we will have multiple constructors, one for variables, one for functions, and one for procedures, this allows us to tell the difference between the variables procedures and functions within our JetContext instance and perform different actions accordingly.

Now we can discus how to type check functions and procedures.
They are very similar in structure and semantics, they both have a list of variable declaration to type check, they also both have a block structure as their body.
These consist of the premises.
We then want to add the whole subprogram declaration to the outer scope that it is contained in so that we can be used in the block it is declared in.
We will also want to add the subprogram declaration to the inner scope so that we can use the subprogram recursively.
So if we have a consequent \texttt{(ProcDecl PDecl ident vdecls decls stmnts)}, where ident is the name of the procedure, vdecls are the variable declaration for the parameters of the procedure, decls is the initial declaration block of the procedure, and stmnts is the statement block of the procedure.
We will want the type rule given in \autoref{lst:jetProcPremises}.
\begin{lstlisting}[caption = Jet premise list for procedures., label=lst:jetProcPremises]
If {let ts = map (\(VDecl _ t) -> t) vdecls},
    {ctx' <- expandContext (Proc ident ts) TNNone (updateRetType ctx TNNone)}
    {(newBlock ctx')} |- [VarDecl vdecls], 
    {var1} |- [Decl decls], 
    {var2} |- [Stmnt stmnts]
then (ProcDecl PDecl ident vdecls decls stmnt) return {return ctx'}
\end{lstlisting}

\todo{May have to rewrite}
The first 2 premises are side conditions to define variables that can be used later on in the premise list or in the consequent.
The first simply gets the list of types from the list of variable declarations.
Next we add the procedure to the given context as this will be required in another type premise and in the consequent.
We then are able to type check the rest of the procedure declaration, the first thing we should do is type check the variable declaration to make sure that they are well formed before we can type check anything else, it is worth to note that here we add a new block to the context so as the parameters should belong to a new block.
We can then type check the list of normal declarations, with the new context returned by the variable declarations (this makes sure that we do not redeclare a parameter in the declaration list).
Finally we can check the body of the function, we give this the context given by the previous type premise, this is so it has visibility of all of the declarations and the parameters.
It will also have the procedure in the context so that we can recursively call the procedure.
We have already covered the consequent, the next thing to cover then is the return statement.
This is simply just to return the context that has the procedure declaration added to it.
We should not need to cover the generated code for this as we have discussed all the features that we have used to write this type rule previously, we are simply demonstrating how you would use those features to type check a procedure.
%\section{The JET Language}

JET allows the user to write type rules that will be generated into Haskell code.
In order to do that the user may have to write some Haskell code in an initial section.
For this reason JET consists of two parts: some Haskell code in an initial section, followed by a list of type rules.

In order to write a type system in JET, the user will also be proved with a Haskell type class that defines a number of functions that will be useful when interacting with the context, such as looking up a variable in the context and adding an item to the context.
This type class has been written in such a way to support as many contexts as possible as to not limit what type of languages JET supports.
Along with the type class, there will also be an example of how to implement an instance of the type class using a Haskell map as an example.

\subsection{Initial Inline Haskell}
%\section{Simple Type Rules}
\subsection{Null Statement Type Rule}
The simplest type rule that we can represent is similar to the one shown in \autoref{fig:adnullt} for our example of an imperative language.
This is a type rule with no premises to check, along with no types to check for the consequent.
Simply, if we have done a successfully parsed this statement, then it will always pass its type check.
Therefore the only code we need to generate is whatever needs to be returned back to the parent.
In this case we don't need to return anything, however there maybe cases in other languages where we would want to return the context back to the parent.

An example of rule such as this written in JET can be seen in \autoref{lst:jetAdNullT}
\begin{lstlisting}[caption = JET type rule for null statement, label=lst:jetAdNullT]
typerule AdNullT <- (Stmnt SNull) return {return ()}
\end{lstlisting}

The initial part \texttt{type AdNullT} has no affect on the rule is just a name of the rule for a use in future documentation.
The following part after the left facing arrow is the piece of abstract syntax that we wish to pattern match against.
This is analogous to the concrete syntax used to define the consequent in the natural deduction logic.
The abstract syntax definition takes the form of the type of abstract syntax we wish to check followed by the constructor we want to pattern match against, hence why for this example it is \texttt{Stmnt SNull} since we wish to check a statement and the specific statement is the null statement.
We need the AST node type name so that we can form the generated code correctly, we also need the constructor so that we can use Haskell's features of pattern matching to match against the correct piece of syntax.

From this type rule we produce the code that can be seen in \autoref{lst:codeAdNullT}.
we generate the name of the function based off of the AST node type name given.
Also notice that since the type rule contained no information about resulting type of the rule no code was produced to do any type checking.
This also means that there is no need to generate a function to infer the type given by this rule as there is no type associated with the rule, therefore, we do not generate the function to do such a type inference.
Therefore, the only thing the code does is pattern match against the piece of specified syntax and return an empty tuple since Statement rules in the language do not affect the context in any way.

\begin{lstlisting}[caption = Code generated from AdNullT, label=lst:codeAdNullT]
checkStmnt SNull ctx = do
    return ()
\end{lstlisting}

One possible improvement to be made in the code is instead of generating code of the form \texttt{foo astNode ctx} is to instead generate code to be of the form \texttt{foo ctx astNode}.
That way if we have multiple multiple instances of \texttt{foo ctx}, then we can give that expression a name and only evaluate it once.
However, there is one possible instance where having \texttt{ctx} be last argument has an advantage which can be seen at \todo{autoref to monad context binding}.

\subsection{Literal Expression Type Rule}
Both of our test languages have literal expressions.
Literal expressions are similar to our null statements example, in that there is not much that they type checker algorithm has to do.
This is because, similar to null statements, they have no premises, and do not affect the context in anyway.
The only difference between literal expressions and null statements is that literal expressions have types associated with them where as statements do not.

\begin{lstlisting}[caption = JET type rule for simple literal expressions, label=lst:jetAdLitExpr]
typerule AdTrueT <- (Expr ETrue) : TNBool return {return ()};
\end{lstlisting}

As can be seen in \autoref{lst:jetAdLitExpr}, we have specified something that was not specified in \autoref{lst:jetAdNullT}.
That is we have annotated the type rule with the type that is associated with the code we are checking.
In the case of a boolean \texttt{true} it has a type of boolean, where as a number has a type of integer.

Since this rule has a type associated, the code we must generate is now more complicated.
For example, we did not have to check the type of the null statement because it did not have a type.
We also did not have to generate a function to infer the type of the null statement because again, there is no type to infer.
However, since literal expressions do have types, we do have to check the type and provide an infer function to get the type of a literal expression.

The check function we generate, which can be seen in \autoref{lst:codeAdTrueTCheck}, should assert that the incoming type is, in this case, the \texttt{TNBool}, otherwise we should produce some form of error.
Errors take the form of a monad, in particular they are of type \texttt{JetError}, this monad is given to the user in a haskell module JetErrorM when they generate their code.
This allows us to propagate the error, along with the error message back out to the user without having to call the error function explicitly\todo{Cite Functional programming book}.
To produce the error message the generated code will call \texttt{makeCheckError} with the data that caused the error.
The user must then define the function \texttt{makeCheckError}, to keep things simple in the example cases \texttt{makeCheckError} is defined to return "Type Error".
However the user could write any code they liked as long it returned a String.

\begin{lstlisting}[caption = Code generated for checkExpr from AdTrueT, label=lst:codeAdTrueTCheck]
checkExpr ETrue jetCheckType ctx = do
    if jetCheckType == TNBool then return () 
    else fail (makeCheckError ETrue jetCheckType TNBool)
\end{lstlisting}

In \autoref{lst:altCodeAdTrueTCheck} I have given another piece of code that I could have generated that would have done the same thing.
It uses pattern matching on the input type as well as the expression, this means we do not require if expression within the function.
However we still do need the more generic function in order for us to be able to correct fail and call the \texttt{makeCheckError} function.

\begin{lstlisting}[caption = Alternate Code for checkExpr from AdTrueT, label=lst:altCodeAdTrueTCheck]
checkExpr ETrue TNBool ctx = then return () 
checkExpr ETrue jetCheckType ctx = 
    fail (makeCheckError ETrue jetCheckType TNBool)
\end{lstlisting}

In hindsight, this is a better to structure the code.
This is because under certain circumstances, such as the ones seen in\todo{autoref to premises}, it can reduce the amount of unnecessary evaluation that may be required due to generating the code with the if expression being the last monadic statement.

Infer functions return the type given by the type rule where as check functions assert that the type we are expecting is the same as the one given by the type rule.
From this, for example given in \autoref{lst:jetAdLitExpr}, we generate the code that can be seen in \autoref{lst:codeAdTrueTInfer}.
The code generated is very similar to the code seen in \autoref{lst:codeAdNullT}.
This is because there are no premises to check and we do not need to check the type specified by the type rule, we just need to return the type.
Therefore, the only thing this function does is return the type associated with the literal value $true$.

\begin{lstlisting}[caption = Code generated for inferExpr from AdTrueT, label=lst:codeAdTrueTInfer]
inferExpr ETrue ctx = do
    return TNBool
\end{lstlisting}
%\section{Type Rules with Premises}
The type rules we have discussed so far have been relatively uninteresting and extremely simply.
While these sorts of type rules are simple they will be required in every language at some point.
The simpleness of the type rules comes from the fact that none of them required any premises, this means that as long as the consequent is true then the evaluation of the type rule will succeed.
The type rules we are going to discuss in this section will have premises of varying complexity, this will demonstrate what features are implemented in JET that allow the user to define type rules like these and how these type rules are then turned into code.

\subsection{Simply Premise}
The first type rule that we will look at is the rule STLCPred in \autoref{fig:stlcTyperules}.
This is a rule for an expression in the Simply Typed Lambda Calculus that takes the predecessor of a number.
Hence it is a rule that requires that the type of the expression it is acting on is also a number.
Therefore, the premise for this rule is that the expression is of type $Int$ which means that $pred\ e$ is of type $Int$.
A representation of this in JET is given in \autoref{lst:jetSTLCPredExpr}

\begin{lstlisting}[caption = Type rule for pred expression involving one premise, label=lst:jetSTLCPredExpr]
typerule TExprPred <- if {ctx} |- (Expr e) : TInt then 
    (Expr Pred e) : TInt return {return ()};
\end{lstlisting}

As can now be seen, the type rule takes the appearance of an if-then statement in traditional imperative languages.
It can be thought of that the premise is a guard to the consequent, such that the consequent can only happen if we can prove the premise.
The type premise takes a similar form as the consequent of the type rule, however, there is once difference.
Just as a type rule may need to return some information about a change in the context, such as in the case of declarations, a type premise will need to be told what context we need to give to the type judgement such that we can prove it.
Similarly to the consequent, we can also specify that the type of the premise must match some expected type.
In this instance we are checking to make sure that the premise is of type integer such that we can say the consequent is of type integer.
Another piece of syntax to note is we have also given a parameter to the constructor that we wish to pattern match in the consequent.
This is so we can produce a valid constructor, but also so that we can perform premises on the paramters of the constructor, such as making sure that the expression for $pred$ is of type integer.

When generating code for a check function involving a premise we want to make sure that we evaluate the premise before we return from the check function.
We do this by making the premise a monadic statement that is executed before the return of the check function.
Another reason why the premise is a monad is so that we can propagate the error back to the user in the event that the premise fails.
This means that no other code is required when handling the error case of the premise due to Haskell's do notation.
From these requirements we produce the code that can be seen in \autoref{lst:codeTExprPredCheck}.

\begin{lstlisting}[caption = Code generated for checkExpr from TExprPred, label=lst:codeTExprPredCheck, language=Haskell]
checkExpr (Pred e) jetCheckType ctx = do
    var1 <- checkExpr e TInt ctx
    if jetCheckType == TInt then return () 
    else fail (makeCheckError (Pred e) jetCheckType TInt)
\end{lstlisting}

One thing to note in the code that wasn't already stated is that we store the result of the successful return of the premise in a variable so that it can be used later if necessary.
We now also see the advantage that the style of code in \autoref{lst:altCodeAdTrueTCheck} would have provided.
If had produced code like that then we would not have to unnecessarily evaluate \texttt{checkExpr e TInt ctx}.
We would only want to evaluate that premise if the type to be checked was integer.
If it was not of type integer then the evaluation of the premise was unnecessary because we could have already known that the type check was going to fail.

Similarly to the check function, when trying to infer the type we also want to run the premises and fail if the premises fail.
Otherwise if we succeed when the premises fail then we have falsely provided a proof of the type.

The code for the infer function, see in \autoref{lst:codeTExprPredInfer}, is very similar to what is seen in the code for the check function for the type rule.
An infer function must perform the same premises as the check function.
Where they differ is in what they return, a check function returns what ever changes it made to the context or anything else of the sort, where as infer function return the type that they have inferred from the given input and context.
This function is so simple that there is not much that could be changed in the code it generates.
It does what is required of it to perform the proof of the type and then return the type.
There is no way to not perform the premises, unlike the optimisation of the check function, as the premise is what proves the type to return and are therefore required to be evaluated.

\begin{lstlisting}[caption = Code generated for inferExpr from TExprPred, label=lst:codeTExprPredInfer, language=Haskell]
    var1 <- checkExpr e TInt ctx
    return TInt
\end{lstlisting}

\subsection{Premises with Type Variables}
There is more we can do with premises other than just assert, for example, something is of type integer or boolean.
So far we have only covered type rules with specified typed.
However, in natural deduction logic we can make statements about arbitrary types such as the if expression inference rule seen in \autoref{fig:ifExprRule}.
Here we have the expression $e_1$ with the type $Bool$, where as $e_2$ and $e_3$ have arbitrary types $\tau$.
Along with the whole expression having type $\tau$.
In inference rules all occurrences of a variable have the same value, therefore, all $\tau$'s are the same.

In Jet, similarly to natural deduction logic, we can not only specify types but also say that something is of an arbitrary type variable.
How the distinction is handled is the same way Haskell handles the distinction between types and type variables: types start with an upper case letter and type variables start with a lower case letter.

\begin{figure}[tbp]
    \begin{prooftree}
        \LeftLabel{IfExpr: }
        \AxiomC{$\Gamma \vdash e_1 : Bool$}
        \AxiomC{$\Gamma \vdash e_2 : \tau$}
        \AxiomC{$\Gamma \vdash e_3 : \tau$}
        \TrinaryInfC{$\Gamma \vdash if\ e_1\ then\ e_2\ else\ e_3 : \tau$}
    \end{prooftree}
    \label{fig:ifExprRule}
    \caption{Example of If-Expression inference rule}
\end{figure}

From these constraints we can specify a type rule to type check and type infer the inference expression given in \autoref{fig:ifExprRule}.
It is very similar to type rules that we have seen before, just instead of giving a type for $e_2$ and $e_3$ we will be giving them a type variable.
We will also need three type premises, we can define any number of type premises, where the premises are comma separated.
The last thing we need to do is to make sure the consequent is of the same type as $e_2$ and $e_3$, we do this by saying it is the same type variable as them.
The whole type rule can be seen in \autoref{lst:jetSTLCIfExpr}

\begin{lstlisting}[caption = Type rule for if expression expression involving type variables, label=lst:jetSTLCIfExpr]
typerule TExprIf <- if 
        {ctx} |- (Expr e1) : TBool, 
        {ctx} |- (Expr e2) : t, 
        {ctx} |- (Expr e3) : t then 
    (Expr If e1 e2 e3) : t return {return ()};
\end{lstlisting}

Generating code for premises where we know the type of the premise is relatively simply.
However, since we now have to deal with type variables, we need to figure out the type of the premises. 
This is why we have been generating infer functions.
Check functions check that the rule is the same as the incoming type.
Whereas infer function return the type so that it can be used in other type rules.
However if we have the same type rule in multiple premises then we do not want to figure out the same type twice for a number of reasons.
The first reason is that there is no point to do so, we in already know the type, we only need to make sure that this premise is of the same type as the previous premise with that type variable.
The other reason is the types for the same type variable could be different and we would not know, this means that the type checker could say that an input was type correct when in reality it should have been a type error.
Therefore we need to keep track of what type variables we have seen when generating the code for the premises.

We will not discuss the check function for this example because the code generated for the premises in both the check functions and infer functions will always be the same.

The code we produce for the infer function can be see in \autoref{lst:codeTExprIfInfer}.
The first premise is as we have seen previously, it is a know type so we are able to perform the check function on it.
The next statement relates to the second premise in the type rule.
This premise refers to a type variable that we currently know nothing about, therefore, the code we generate will want to be the infer function so that we can figure out the type that the type variable, t, represents.
We always bind the result of the infer function to a variable of the same name as the type variable in the specification.
Then we move onto the final premise, which also refers to the type variable.
At this point we have already figured out the type that t represents.
This is why instead of calling inferExpr again, we can now call checkExpr where the type to check for is now the variable t that was assigned in the previous premise.
Finally, we then wish to return the type t rather than some known type.

\begin{lstlisting}[caption = Code generated for type variables in inferExpr from TExprIf, label=lst:codeTExprIfInfer, language=Haskell]
    t <- inferExpr e2 ctx
    var2 <- checkExpr e3 t ctx
    return t
\end{lstlisting}

We can also type variables with the type constructors.
This allows us to extract the parameters to some type constructor.
For example, if there was an array access expression, we may want to know the type that the elements of the array hold.
The natrual deduction rule for this kind of type rule can be seen in \autoref{fig:typeRuleArr} \todo{cite TAPL book where the example comes from}.
In this example the definition of the array type in the AST would be \texttt{Type = TArr Type | ...}.
As can be seen the array type takes in a type as a parameter which is the type of the elements of the array.
Therefore, when type checking an array access, to retrieve the type of the element of the array you would want extract that parameter from the array.
This can be achieved through the use of Haskell's pattern matching features.
We want to generate some code of the form \texttt{TArr t <- inferExpr e ctx}, because this code asserts that the type returned from infer function must be an array type, we then get the parameter of the array type and store it in some variable t.
The generator will also have to add the variable t to the list of known type variables so that if another premise required the use of this type variable then we can generate a check function instead of an infer function.
We express this premise in the type rule in a very similar manner to what code is generated:
\texttt{\{ctx\} |- (Expr e) : TArr t}.
There is one issue with the code that we generate at the moment.
If the return of the infer function does not match the pattern function then the type checker will error, but not with our defined error function since the error was an internal Haskell error.
Instead the right hand side of the binding will want to call our error function in case of a bad patter match, from this criteria we produce the follow code: \texttt{case inferExpr e ctx of Succ jet0@(TArr t) -> Succ jet0; Succ t -> Fail (makeInferError t t); x -> x}.

\begin{figure}[t]
    \centering
    \begin{prooftree}
        \LeftLabel{ArrAccess: }
        \AxiomC{$\Gamma \vdash e : Arr\ t$}
        \AxiomC{$\Gamma \vdash i : Int$}
        \BinaryInfC{$\Gamma \vdash e[i] : t$}
    \end{prooftree}
    \caption{Example type rule for array access}
    \label{fig:typeRuleArr}
\end{figure}

This kind of type rule is used in both witness languages.
The Simply Typed Lambda Calculus required it so that lambda abstract and lambda application can be type checked.
This feature is also required so that the fix point operator can be type checked.
It is required in the Ad language so that array types and record types can be type checked correctly.


\subsection{Side Conditions}
There is another way to represent type premises, and that is through the use of side conditions.
Side conditions are extra predicates along side normal premises that must also be true for the consequent to be true.
A user may wish to represent these side conditions with in their JET specification.
Currently, from all the features of JET that we have discusses, there is no way to easily evaluate arbitrary code such that it can match any possible arbitrary predicate that could appear in a type specification in natural deduction logic.
This is why there is feature that allows the user to write any arbitrary Haskell code and it be evaluated as a premise.
For example is a user want to have a language that allowed the assignment of arbitrary number of variables in a single statement i.e. \texttt{a, b, c, d := e$_1$, e$_2$, e$_3$, e$_4$}, where a is assigned to e$_1$ and b is assigned to e$_2$ etc.
In a statement like this it would be reasonable to expect that the list of variables to assign to is the same as the list of expressions.
We could use the side conditions, it would be written in place of a normal type premise and would instead be a piece of inline Haskell code, where inline Haskell code is surrounded by curly braces (\{\}).
There fore this could be written as: \texttt{\{if length vars == length exprs then return () else fail "Type error"\}}

\subsection{Limitations}
\todo{discuss limitations on the use of type variables.}
%\section{List Notation}
Many languages, in their abstract syntax, have nodes that take a list of nodes to be one of their children, an example of this can be seen as a sequence of statements as part of the Ad language in \autoref{appendix:witnessLanguages}.
It is currently possible to type check these sorts of type rules with the current set of features we have discussed, unfortunately, it requires the use of side conditions to define variables to allow us to define more conventional premises to do the rest of the type checking.
However, since dealing with lists of statements, or expressions, etc. are very common then it is worthwhile to add a set of features to the language that allows the user to define the type rules for lists of nodes.
The set of features should allow the user to create type rules handle that the different patterns of lists that the user may want to have type rules for: such as the empty list; singleton list and the cons of an item and the tail of a list.
The most useful pattern is probably the cons of an item and the rest of the list, represented in Haskell as \texttt{(x:xs)}.
This is because it allows the user to recursively type check an item and then the end of the list.
The other two patterns are useful when defining the base case of the recursive type rule, such as if the list is empty or if the list is the singleton list.
Being able to pattern match on a singleton list allows the user to assert within the type rule that the list cannot be empty.
The list notation can be used to both just perform a check, but also is able to return a type or a list of types.
Returning a list of types is useful when type checking a function or procedure call, as seen in the Ad language, to check that the types of expressions (given as an expression list) matches the expected types of the procedure or function.

\subsection{List Notation in Type Premises}
Using list notation in type premises is similar to how normal type premises are used.
The one difference is instead of encasing the premise with parenthesis, we instead encase it within square brackets.
This so that it matches how lists are defined in Haskell, therefore, the users should be familiar with square brackets being associated with lists.
The example for the list of statements is given as \texttt{\{ctx\} |- [Stmnt stmnts] }.
As can be seen it is exactly the same is a normal type premise, with only the type of brackets used being different so that both the user and the tool can tell the difference.
The code that is produced from this type of premise is also very similar to the code produced from a normal premise: \texttt{var1 <- checkStmntList stmnts ctx}.
How the parameters are passed, along with how the result of the premise is bound to a variable, is exactly the same.
The only difference being is the function name has \texttt{List} as a suffix.
If we did not have this suffix then we would call \texttt{checkStmnt} which would lead to a type error as \texttt{checkStmnt} has the type: \texttt{Stmnt -> Context -> JetError a}, whereas the function we are trying to call would have to have type: \texttt{[Stmnt] -> Context -> JetError a}.
The difference being, one expects a statement, the other expects a list of statements.

\subsection{List Notation Type Rules}
We have discusses how list notation is used in premises, however, that is not useful unless code can be generated for type rules using list notation.
Type rules for list notation bear a resemblance to normal type rules, however, they do vary in sum fundamental ways.

%\section{Utilising the Context}
Handling the context in JET is done through Haskell side condition to call functions that will get data from or insert data to the context.
As the user can write any code they want it may seem initiative as to what is the best way to interact with the context.
This is why the tool will provide the modules that can be seen in\todo{autoref to haskell modules appendix}, which will allow the user to instantiate a small number of Haskell type classes\todo{cite type classes} that will give the user a common interface with their defined context.
It is important to note that these modules do not define the context for the user, the user will have to define their own context type and instantiate the Haskell type classes for their own context type.
There is an example provided to the user of how to do this used Haskell Data Maps\todo{cite data maps} so that the user is not on their own in this regard.
\subsection{Empty Contexts and Blocks}
The most basic things you can do with a context is create an empty context or add a new block.
An empty context should do exactly what the name suggests, it should create a new context that has no items in it.
Since the user implements all the functions they should ensure that their function meets these semantic requirements.
We will assume that all languages will have a block structure, and if a language does not that this can be modelled by having a context which will only ever have one block.
How the context represents blocks is up to the user, however in our witness languages a single block is a map and our context is made up of a list of blocks.
So when we implement out new block function for our context we will prepend to the list a new empty map, this is to meet the requirement of the new block function that each new block should have no items in it.
Adding a new block is fairly simple with in the type rule, within the context definition of a type premise within a type rule you call the new block function on the context that you wish to add a new block in, for example:\\\texttt{{(newBlock ctx)} |- ($TypePremise$)}
this also produces very simply code where we would usually have the variable \texttt{ctx} we now have the code \texttt{(newBlock ctx)} giving us the whole type premise of\\\texttt{$variable$ <- $TypePremiseFunc$ $TypePremiseParams$ (newBlock ctx)}
\subsection{Variables}
The most common use of the context is when attempting to type check the use of variables within a program.
We will want to do two main actions with variables, the first is to retrieve the type of a variable from the context, the second is to add a new variable with a given type to the context.
Retrieving the type of a variable from the context is given by the natural deduction rule in \autoref{fig:varTypeRule}, as can be seen the only thing we need to do is to perform an existence check that the variable given is in the context, we also need to get the type of that variable from the context.
There is no way to express this meaning using normal type premises, instead we will have to use a Haskell side condition (similar to the side condition seen in the natural deduction rule).
However the consequent is very trivial and is the same as the ones we have seen for other expression which have arbitrary types associated with them, the consequent in \author{lst:jetSTLCIfExpr} is an example of the consequent required here.
We still need to retrieve the arbitrary type $\tau$ from the context.
We can use do this by using the function, given by the \texttt{JetContext} type class, \texttt{lookupContext}.
This function, given some a parameter to identify the item in the context (such as the variable name), will return the type associated with the input, or will return some fail message within the \texttt{JetError} monad.
When using this in a haskell side condition we will want to bind the resulting type from the context to the variable that we are going to use as the return type variable of the consequent.
If we say that the return variable of the consequent is \texttt{t}, then we will want to write the side condition as \texttt{\{t <- lookupContext var ctx\}}, where \texttt{var} is the variable to lookup and \texttt{ctx} is the given context to perform the lookup in.
We do not do anything special when generating the code from a Haskell side condition, we simply put it directly into the generated code in the position it occurs in the premise list (In this example it is the only premise so that does not matter just as long as it appears before the return monadic statement).

\begin{figure}
    \begin{prooftree}
        \LeftLabel{VarT: }
        \RightLabel{$[v : \tau \in \Gamma]$}
        \AxiomC{}
        \UnaryInfC{$\Gamma \vdash v : \tau$}
    \end{prooftree}
    \label{fig:varTypeRule}
    \caption{Type rule for using a variable in the Simply Typed Lambda Calculus and Ad languages}
\end{figure}

We can now retrieve types of variables from the context, however, that is not useful unless we can add variables to the context.
We will be using the example that can be seen in \autoref{fig:varDeclTypeRule} taken from the Ad language.
In this case we don't have any side conditions or premises, we only have the consequent which has the context $\Gamma$ but also has the variable and type added to the context.
We therefore need a way to represent this behavior in the return part of our type rule so that we can give this new context to the rest of the program that requires it.
We do this by writing the required Haskell at the end of the type rule after the return keyword.
The code written here is the code that will be return to the calling function so that it has all the required information from its child node, this is required when type checking a list of declarations.
In order to add the variable to the context will want to use the function, given by the \texttt{JetContext} type class, \texttt{expandContext} which takes in something to identify the item (the variable), the type to be associated and the context to add the item to.
This function may fail depending upon what the required semantics the user wants, for example a user may not want to be able to declare the same variable twice, and in other languages this may be allowed.
The user is able to define this functionality when defining the instance of the type class.
To use this function in the type rule we will want write something similar that was seen in the use of a variable but is slightly more complicated.
We will want to write \texttt{\{expandContext var t ctx\}}, this will add (if it is possible) the variable \texttt{var} with type \texttt{t} to the context given by \texttt{ctx}.
this gives us the whole type rule \texttt{(VarDecl var t) return \{expandContext var t ctx\}}.
where the \texttt{expandContext} function call replaces the \texttt{()} that we have used previously.

\begin{figure}
    \begin{prooftree}
        \LeftLabel{AdVarDecl: }
        \AxiomC{}
        \UnaryInfC{$\Gamma, v : \tau \vdash \lfloor v : \tau \rfloor\ valid$}
    \end{prooftree}
    \label{fig:varDeclTypeRule}
    \caption{Type rule for variable declaration in the Ad language}
\end{figure}

\subsection{Procedures and Functions}
Type checking functions in the Simply Typed Lambda Calculus is exactly the same as type checking variables as functions are first class citizens in the Simply Typed Lambda Calculus.
However, this is not the same in the Ad language (or in many other similar imperative languages such as C or Ada).

In order to type check procedures and functions we will want to discuss the notion of namespaces within JET.
Namespaces is a way that we can write our context in such a way that we can type check functions procedures and variables and be able to tell the difference between them based upon how they are used.
For example, in our Ad language we have to ways to call a subprogram (function or procedure), one is an expression therefore it requires a return type, the other is a statement and therefore requires not to have a return type.
From this we can identify in the expression call we need to check for a function, and in our statement call we need to check for a procedure.
We can also check for the difference between a function call and the use of a variable.
These will therefore all come under different "contexts" within our actual context, we will refer to these as namespaces to avoid the use of conflicting names.
We can interact with different namespaces by defined our new type, which will be our context identifier type.
In this type we will have multiple constructors, one for variables, one for function, and one for procedures, this allows us to tell the difference between the variables procedures and function within our JetContext instance and perform different actions accordingly.

Now we can discus how to type check functions and procedures.
They are very similar in structure and semantics, they both have a list of variable declaration to type check, they also both have a block structure as their body.
These consist of the premises.
We then want to add the whole subprogram declaration to the outer scope that it is contained in so that we can be used in the block it is declared in.
We will also want to add the subprogram declaration to the inner scope so that we can use the subprogram recursively.
So if we have a consequent \texttt{(ProcDecl PDecl ident vdecls decls stmnts)}, where ident is the name of the procedure, vdecls are the variable declaration for the parameters of the procedure, decls is the initial declaration block of the procedure, and stmnts is the statement block of the procedure.
We will want the type rule given in \autoref{lst:jetProcPremises}.

\begin{lstlisting}[caption = Jet premise list for procedures., label=lst:jetProcPremises]
If {let ts = map (\(VDecl _ t) -> t) vdecls},
    {ctx' <- expandContext (Proc ident ts) TNNone (updateRetType ctx TNNone)}
    {(newBlock ctx')} |- [VarDecl vdecls], 
    {var1} |- [Decl decls], 
    {var2} |- [Stmnt stmnts]
then (ProcDecl PDecl ident vdecls decls stmnt) return {return ctx'}
\end{lstlisting}

The first 2 premises are side conditions to define variables that can be used later on in the premise list or in the consequent.
The first simply gets the list of types from the list of variable declarations.
Next we add the procedure to the given context as this will be required in another type premise and in the consequent.
We then are able to type check the rest of the procedure declaration, the first thing we should do is type check the variable declaration to make sure that they are well formed before we can type check anything else, it is worth to note that here we add a new block to the context so as the parameters should belong to a new block.
We can then type check the list of normal declarations, with the new context returned by the variable declarations (this makes sure that we do not redeclare a parameter in the declaration list).
Finally we can check the body of the function, we give this the context given by the previous type premise, this is so it has visibility of all of the declarations and the parameters.
It will also have the procedure in the context so that we can recursively call the procedure.
We have already covered the consequent, the next thing to cover then is the return statement.
This is simply just to return the context that has the procedure declaration added to it.
We should not need to cover the generated code for this as we have discussed all the features that we have used to write this type rule previously, we are simply demonstrating how you would use those features to type check a procedure.]

Functions are more or less identical expect we have the return type of the function that goes in place of the \texttt{TNNone} placeholder.
We will also want to set the current return type so that when we are type checking a return statement we can successfully check the type of the expression to return and the expected return type.

\subsection{Record Types and Type Synonyms}
\todo{I might just say that this is possible to do, but that the notation is not nice (Might therefore be better to go in evaluation.)}
\chapter{Evaluation}
Given all of the features and type rule patterns we have discussed in \autoref{chap:Method} we can successfully type check both of our witness languages.
As per the aims given in \todo{autoref to aims}, we have discussed how to type check all but user defined types.
The full type specification in JET along with the natural deduction rules that we are implementing in our JET specification can also be seen in \autoref{appendix:witnessLanguages}.
While we have not discussed how to type check user defined types and record types, the JET specification for the Ad language does support these language features and does successfully type check them.
We did not discuss these type rules because there are no other features in the language that allow for a better representation of user defined types and record types than the representation that can be seen in the JET specification of the Ad language.
Both of our example languages have varying differing paradigms and language structure meeting one of our first main criteria of the project which was to be able type check simple programming languages from different paradigms.
The first of our languages, Ad was a small imperative block structured language which had variable, function and procedures along with a few primitive statements and expressions.
Where as STLC was a functional programming language which allowed for the use of functions as first class citizens along with higher order functions.
These two languages represent two main paradigms in programming languages and provide a base from which to implement other features in the future which would allow the extension of the type systems in these languages.
The type system features that could be added to these languages are, in brief, for Ad the use of extending record type to have the ability of subtyping and also implementing full object orientations; and in STLC the addition of recursive types\cite{pierce2002types,cardelli1996type} and polymorphic types\cite{Cardelli:1985:UTD:6041.6042}.
While these features are were not a requirement, they were mentioned as possible end goals that a full tool (JET is currently still a prototype) would need to support, otherwise there would be too many languages that JET would not support, therefore JET would  only be useful to generate code for simple and small languages.  

\section{Language Structure and Extensions}
Syntactically, the language is similar to natural deduction rules, which sets it apart from a tool such as typical\cite{grimm2007typical} which have is a programming language with some extra syntactic features to help with writing type checkers.
This is main advantage of this tool that it is designed to represent the natural deduction rules that it seeks to implement, however, it also allows the user to write their own Haskell to add functionality that otherwise JET would not be able to support.  

One of the major short comings of JET is the representation of the context, both how it is defined and used throughout a given type system.
As mentioned in ?? \todo{ref to typical}, I think the representation of the context in typical is one of the best, it gives the user the ability to customise the type system as they wish but also provides a notation that is intuitive and easy to use.
We tried, as much as possible to make the definition of the type classes, used to define the contexts, as easy to use as possible and to help give the user an idea of what information may be required to build a specification in JET.
However a nicer notation for defining namespaces would most likely make the tool easier to use than it is in its current form, currently you have to write a fair amount of Haskell in order to get the context to act as required.
Where as the notation given in \textcite{grimm2007typical} is a much nicer way of representing namespaces.
The inclusion of the predefined instance of the type class for \texttt{JetContextMap} does help provide a base for how the user will define their namespaces.
The lack of language support for functions defined by the Haskell type class reduces the usability of the tool.
This could be fixed by adding extra statements in the language.
For example is we were to attempt to lookup an item in the context in STLC, instead of writing the premise using inline haskell as \texttt{\{t <- lookupContext var ctx\}}.
We could instead have a piece of syntax in JET that encodes this meaning, for example we could have a grammar rule that has a keyword lookup and takes an item to lookup and a context to lookup withing, an example BNF grammar rule can be seen in \autoref{fig:LookupBNF}.
Using this grammar rule the type premise would be rewritten as \texttt{var lookup context : t}.
This would then be generated into the code that appeared in the initial type premise.

\begin{figure}[]
    \centering
    \begin{grammar}
        <TypePremise> ::= \dots \alt <Identifier> `lookup' <Identifier> `:' <Identifier>
    \end{grammar}
    \caption{}
    \label{fig:LookupBNF}
\end{figure}

I have mentioned that we can type check user defined types.
This is true, however, it requires excessive use of user defined Haskell functions, and also takes advantages of how the language is structured and how we generate the output code.
Using odd side affects of the language design to get the required behaviour is obviously leads to unintuitive and hard to read type rules.
Therefore, an area of future work to be performed, for JET specifically, would be to design and implement a notation that provides a more intuitive method of performing type checking for user defined types along with record types etc.
This would greatly improve the usability of the tool, making it easier to implement a type checker for a language with a type system involving user defined types.
Possibly even making it quite simple to type check object oriented language which is class of language that we have not discussed but in the current state would require similar type rules to the ones in Ad for record types.

The type inference algorithm that JET generates is very simple, and is simply used to retrieve the type of expressions within a given language.
Therefore, JET has no built-in support for type reconstruction or unification\cite{pierce2002types,cardelli1996type}.
This because there is no built-in way of specifying constraint typing relation\cite{pierce2002types}, this would require a notation in JET that allowed for the definition and use of type constraints.
We then be required to generate some type unification algorithm such that we can find a valid substitution that satisfies the constraints given. 
If we were able to add these features, then we would be able to start type checking languages of a similar class to standard ML\cite{milner1997definition}, which as of at the moment would either require vast amounts of inline Haskell or cannot be done at all with JET.

\section{Generated Code Structure}

While we are able to generate correct code that meets our objectives, we are not generating particularly efficient code.
Since JET is still a prototype tool we do not worry at this moment in time about the efficiency of the code generated.
However, that is not to say that we should completely ignore the idea of generating more efficient code.
The biggest issue with efficiency is the order of generated code in our generated check functions with a given type.
For example, the code we generate for our if-expressions in STLC is:
\begin{lstlisting}
checkExpr (If e1 e2 e3) jetCheckType ctx = do
    var1 <- checkExpr e1 TBool ctx
    t <- inferExpr e2 ctx
    var2 <- checkExpr e3 t ctx
    if jetCheckType == t then Succ () else Fail "Type error"
\end{lstlisting}
In the event that our type \texttt{t} is not the same the type we wish to check, \texttt{jetCheckType}, then we will end up evaluating all the way to the if condition.
This is not the most efficient code we can generate because we know \texttt{jetCheckType} is going to have to be the same as \texttt{t}, therefore we can say \texttt{jetCheckType} is \texttt{t} and turn the \texttt{inferExpr} into a \texttt{checkExpr}, so we would instead be generating the following code:
\begin{lstlisting}
checkExpr (If e1 e2 e3) t ctx = do
    var1 <- checkExpr e1 TBool ctx
    var2 <- checkExpr e2 t ctx
    var3 <- checkExpr e3 t ctx
    Succ ()
\end{lstlisting}
In this case we will fail as soon as possible if \texttt{e2} does not have the correct type where as before we would not fail until after everything else had been evaluated.

Inefficient ctx parameter position.

Generate quick check code for type classes to make sure they are implementing the semantics correct.

\section{Operational Semantics and Big Step Notation}
Long term extensions are to add ability to specify operation semantics.


\printbibliography

\appendix

\chapter{JET Language}
\label{appendix:jetLanguage}
\section{Grammar}

\chapter{Ad Language}
\label{appendix:AdLanguage}
The Ad language is a test case language to express the features that the tool can generate code for.
It is a simple imperative language.
\section{Grammar}
\begin{grammar}
    <FuncDecl> ::= "function" <Ident> "(" <ListVarDecl> ")"
    \\"return" <TypeName> <Block>
    
    <ProcDecl> ::= "procedure" <Ident> "(" <ListVarDecl> ")"

    <VarDecl> ::= <Ident> ":" <TypeName>

    <ListVarDecl> ::= $\epsilon$ | <VarDecl> | <VarDecl> ";" <ListVarDecl>

    <Block> ::= "declare" <ListDecl> "begin" <ListStmnt> "end";

    <Decl> ::= <FuncDecl> | <ProcDecl> | <VarDecl>

    <ListDecl> ::= $\epsilon$ | <Decl> ";" <ListDecl>

    <Stmnt> ::= "null"
    \alt "print" <Expr>
    \alt <Ident> ":=" <Expr>
    \alt "if" <Expr> "then" <ListStmnt> "else" <ListStmnt> "end"
    \alt "while" <Expr> "do" <ListStmnt> "end"
    \alt <Block>
    \alt Ident "(" <ListExpr> ")"

    <ListStmnt> ::= <Stmnt> ";" | <Stmnt> ";" <ListStmnt>

    <Expr> ::= <Integer> | "true" | "false" | "(" <Expr> ")"
    \alt <Ident> "(" <ListExpr> ")" | "neg" <Expr> | "not" <Expr>
    \alt <Expr> "*" <Expr> | <Expr> "and" <Expr>
    \alt <Expr> "+" <Expr> | <Expr> "or" <Expr>
    \alt <Expr> "=" <Expr> | <Expr> "<" <Expr>

    <ListExpr> ::= $\epsilon$ | <Expr> | <Expr> "," <ListExpr>

    <TypeName> ::= "bool" | "int"
\end{grammar}
\section{LBNF Grammar}
\begin{lstlisting}
Prog.    Program ::= FuncDecl;
FDecl.   FuncDecl ::= "function" Ident "(" [VarDecl] ")" 
    "return" TypeName Block;
PDecl.   ProcDecl ::= "procedure" Ident "(" [VarDecl] ")" Block;
VDecl.   VarDecl  ::= Ident ":" TypeName; 
separator VarDecl ";";

Blck.    Block ::= "declare" [Decl] "begin" [Stmnt] "end";

DFunc.   Decl ::= FuncDecl;
DProc.   Decl ::= ProcDecl;
DVar.    Decl ::= VarDecl;
terminator Decl ";";

SNull.   Stmnt ::= "null";
SPrint.  Stmnt ::= "print" Expr;
SAssign. Stmnt ::= Ident ":=" Expr;
SIf.     Stmnt ::= "if" Expr "then" [Stmnt] "else" [Stmnt] "end";
SWhile.  Stmnt ::= "while" Expr "do" [Stmnt] "end";
SBlock.  Stmnt ::= Block;
SCall.   Stmnt ::= Ident "(" [Expr] ")";
terminator nonempty Stmnt ";";

EInt.    Expr4 ::= Integer;
ETrue.   Expr4 ::= "true";
EFalse.  Expr4 ::= "false";

ECall.   Expr3 ::= Ident "(" [Expr] ")";
ENeg.    Expr3 ::= "neg" Expr4;
ENot.    Expr3 ::= "not" Expr4;

EMul.    Expr2 ::= Expr2 "*" Expr3;
EAnd.    Expr2 ::= Expr2 "and" Expr3;

EAdd.    Expr1 ::= Expr1 "+" Expr2;
EOr.     Expr1 ::= Expr1 "or" Expr2;

EEq.     Expr  ::= Expr "=" Expr1;
ELT.     Expr  ::= Expr "<" Expr1;
coercions Expr 4;
separator Expr ",";

TNBool.  TypeName ::= "bool";
TNInt.   TypeName ::= "int";   
\end{lstlisting}
\section{AST}
\begin{lstlisting}
newtype Ident = Ident String
data Program = Prog FuncDecl

data FuncDecl = FDecl Ident [VarDecl] TypeName Block

data ProcDecl = PDecl Ident [VarDecl] Block

data VarDecl = VDecl Ident TypeName

data Block = Blck [Decl] [Stmnt]

data Decl = DFunc FuncDecl | DProc ProcDecl | DVar VarDecl

data Stmnt
    = SNull
    | SPrint Expr
    | SAssign Ident Expr
    | SIf Expr [Stmnt] [Stmnt]
    | SWhile Expr [Stmnt]
    | SBlock Block
    | SCall Ident [Expr]
    deriving (Eq, Ord, Show, Read)

data Expr
    = EInt Integer
    | ETrue
    | EFalse
    | ECall Ident [Expr]
    | ENeg Expr
    | ENot Expr
    | EMul Expr Expr
    | EAnd Expr Expr
    | EAdd Expr Expr
    | EOr Expr Expr
    | EEq Expr Expr
    | ELT Expr Expr

data TypeName = TNBool | TNInt
\end{lstlisting}

\section{Type Rules}
\subsection{Statements}
\begin{figure}[H]
    \begin{prooftree}
        \LeftLabel{AdNullT}
        \AxiomC{}
        \UnaryInfC{$\Gamma \vdash \lfloor null \rfloor\ valid$}
    \end{prooftree}
    \caption{Type rule for null statement}
    \label{fig:adnullt}
\end{figure}
\subsection{Expressions}
\begin{figure}[H]
    \begin{prooftree}
        \LeftLabel{AdTrueT}
        \AxiomC{}
        \UnaryInfC{$\Gamma \vdash true : TBool$}
    \end{prooftree}
    \caption{Type rule for true literal expression}
    \label{fig:adtruet}
\end{figure}


\end{document}