\documentclass{UoYCSproject}

\usepackage{bussproofs}
\usepackage{float}
\usepackage{todonotes}
\usepackage{syntax}
\usepackage{listings}
\usepackage{amssymb}
\usepackage{caption}
\usepackage{subcaption}

\addbibresource{ref.bib}

\author{Steven M. Tomlinson}
\title{Generation of Type Checking Code}
\date{2018-September-21}
\supervisor{Jeremy L. Jacob}
\BEng

\dedication{For my mum}

\acknowledgements{
  I give thanks to my supervisor Jeremy Jacob for his continued help during the process of writing this project.
  I would also like to give thanks to my friends and family who have supported (and coped) throughout the project.
}

\lstset{
  aboveskip=0mm,
  belowskip=0mm,
  showstringspaces=true,
  columns=flexible,
  basicstyle={\footnotesize\ttfamily},
  breaklines=true,
  breakatwhitespace=true,
  tabsize=3,
  numbers=left
}
  % More definitions & declarations in example.ldf
  \begin{document}
  \pagenumbering{roman}
  \maketitle
  \renewcommand*{\lstlistlistingname}{List of Listings}
  \lstlistoflistings

  \listoffigures
  %\lstlistoflistings

  \begin{summary}
    Type systems are used to reduce the number of incorrect programs that a compiler will accept and generate code for.
    They are often the first line in defence against stopping bugs from being introduced into programs.
    For many years type systems have had a formal notation to specify them.
    There are a small number of tools that will generate code for type checkers, some of which have designed their notation such that it is akin to the formal notation.
    However, the feature sets that these tools support vary by a large amount, and many of these tools are can be complicated and not beginner friendly.
    We are therefore going to devising a notation that can describe type systems in a machine readable format.
    The notation should also be intuitive such that compiler writers not familiar with the formal notation type systems will still be able to understand the type systems written in our notation.
  
    The notation should be able to support simple languages from different programming paradigms, such as imperative and function paradigms.
    We should be able to support arithmetic and Boolean expressions, such as addition, equality, etc.
    There should be the facility to support the use and declaration of variables and subprograms.
    We should be able to support more complicated types than just integer and Boolean, such as array types and record types.
    However, given the time scale for the project we are not expected to be able to type check object orientated languages, such as Java,  and other languages of similar complexity, such as Haskell so only the language features discussed are required and any extra languages features are to be implemented if time permits.
    
    The notation we have devised is similar to the formal notation of type systems but its syntax is devised in such a way that it used programming constructs that developers should be familiar with, such as an if-then statement to assert properties within the type system and a return statement that can be used to return information.
    The notation differs from the notation used in other tools in a number of ways.
    Many other notations force the user to use a particular set of tools to perform lexical analysis and syntax analysis.
    The notation we have devised, and subsequent tool, do not have this issue because we use the abstract syntax definition of the language to both represent the type system in our notation and to perform the type checking in the generated code.
    The provides the user with the ability to use any parser and lexer they wish.
    The one stipulation we have, is the parser and lexer must be written in Haskell, or a tool that generates Haskell, this is because the target language (the output language of the generated code) of the code we generate is Haskell.
    We generate Haskell code so that we make use of some of the high level language features in Haskell which makes traversing abstract syntax trees trivial.
    The algorithms that we generate are based off of the algorithms discussed in \textcite{ranta2012implementing}.
    Our notation allows the use to write their own Haskell code so that they interact with the context.
    Interacting with the context is important because the context stores all the information about variables and subprograms, it may also contain user defined type information.
    We let the user define their own context so that we do not limit what kinds of languages we support.
    However, to help the user to define their context we have provided a Haskell type class with function signatures for common actions that would be performed on the context.
    The functions in these type classes are functions to make an empty context, add a new block to the context, lookup an item and retrieve its type from the context, and add a new item to the context with a given type.

    Our notation and tool is able to represent and generate code for two example languages that we have used as test cases to prove that we can type check all of the features that we have specified we should be able to type check.
    These two languages are very different, one is a simple imperative language, the other is a simple functional programming language.
    This demonstrates that the kinds of languages that our tool can represent and generate code for is diverse.
    However, our notation is far from perfect, it cannot type check every type of language, in particular it is unable to type check languages similar to ML and Haskell that have polymorphic type systems.
    The code we generate is also not as efficient as it could be, especially in an error case.
    There are also an number extensions we could make for the representation of the context and its given functions, especially in the use of its given functions.

    \subsubsection{Statement of Ethics}
    There are very few ethical considerations to make.
    We have not used any people or animals as part of the testing process for the tool.
    There have been no experiments performed during the development of the tool or in any of its subsequent testing.
    Any software that was used during the development of the tool or used as part of the discussing during this report has been credited to the relevant people, we have not claimed any of the software as our own.

    \end{summary}
\chapter{Introduction}
\label{chap:intro}
Programming languages have had type systems and type checkers dating back to Fortran\cite{Backus:1978:HFI:960118.808380}.
Since then, type systems have gone through many changes and an academic body of research has developed out of type systems to explore what information they can represent\cite{cardelli1996type}.
Over the past few decades type systems for modern programming languages have grown in sophistication and as a consequence type checkers have grown more complicated.
As software becomes more complex, the likelihood for errors occurring becomes higher.
If there is an error in a type checker then there is the possibility of code containing errors being passed through the compiler successfully, whereas if the compiler were to behave correctly, such an error could have been caught.

This leads to the possibility of generating the code for the type checkers based upon the specification of the type systems in order to remove some of the complexity of developing the type checker.
Therefore, a notation will be presented that can be used to represent a specification of a type system
From this notation code can be generated that implements the specification it is representing.
The notation should be simple to use and easy to understand, The simplicity of such a notation should reduce the number of errors that occur in these complicated pieces of software.
It should also be able to generate code for a large variety of programming languages with many different language features.
The following list is the list of required features that our notation should have or be able to represent:
\begin{itemize}
    \label{itms:aims}
    \item Simple notation similar to the current method of representing type systems in academic literature but that should also be easy to understand by people not familiar with the academia.
    \item Be able to generate code for many different languages, for example for imperative and function programming languages.
    \item Be able to generate code for simple arithmetic expressions and simple statements, i.e. an addition expression and loop statements.
    \item Be able to generate code for variables and subprograms (functions and procedures).
    \item Be able to represent more complex data types, ranging from array types and record types, to functions types (i.e. lambda expressions).
\end{itemize}

These required features should allow for the notation to represent many kinds of type systems.
They should also allow for the notation, and corresponding generated code, to be sophisticated enough to type check most simple languages whilst also allow for the possibility for extensions to add more supported kinds of languages in the future.
This list of required features is how we will check for the success of the project and the notation.
However, there is another set of features that are desirable but not required features.
For example, all things permitting, it should be able to generate code for and represent user defined types types and type synonyms (i.e. typedef's in C\cite{kernighan2006c}).
Another possible feature is to generate code for Hindley-Milner type inference\cite{MILNER1978348}, this would allow for the ability to type check languages similar to ML\cite{milner1997definition}. 
This is not a required feature but it would greatly expand the scope of what the desired notation would be capable or representing.

The next chapter will discuss the background of type systems and type checkers, along with some previous type checker generators and discuss their advantages and disadvantages.
In \autoref{chap:Method}, the notation for representing type systems will be given, along with the code that we generate from that notation.
The different features in the notation will be expanded upon, examples of how the notation can, and is intended to be, used will then be given.
In \autoref{chap:eval}, it will be discussed whether or not the notation was able to successfully represent and generate code for all of the required features.
Then, there will be a discussion of any potential problems with the notation or with the code that is generated.
Finally any further work that can be done to extend the notation or the code generated will be presented, the possible usefulness of such extensions will also be discussed.

\chapter{State of the Art}
\label{chap:sota}
The idea of generating code to be a part of the compiler pipeline is not new.
The first two stages of the front-end of a compiler, the lexical analyser and the syntax analyser (more commonly known as the lexer and the parser), have general tools to generate their code using Domain Specific Languages (DSL)\cite{Bentley:1986:PPL:6424.315691,van2000domain}.
Two such of these languages are Flex\cite{Levine:2009:FB:1696439}, as the lexer generator, and Bison\cite{Levine:2009:FB:1696439}, as the parser generator.
These tools generate C code to be used to build the rest of a compiler, possibly including a type checker and machine code generation.
These programs have been around for many years, and many derivatives of them have been made to be used with different general purpose programming languages, such as Alex and Happy for Haskell or Jlex and Cup for Java\cite{ranta2012implementing}.
Many languages have been made using these such programs and their respective DSL's, for example the parser for Haskell is self-generated by a grammar defined in a piece of Happy code.
However, not all parts of the compiler pipeline have general tools that are widely used.
Although, some such tools do exist\cite{grimm2007typical,dijkstra2006ruler,Gray:1992:ECF:129630.129637}.
\section{Type Systems}
Before we can start discussing type checkers, we need to discuss the type systems which the checkers implement.

The concept and implementation of type systems has existed for almost as long as the concept of higher level language to perform computation\cite{Backus:1978:HFI:960118.808380}.
Many languages have a type system in one form or another with varying degrees of complexity,
ranging from relatively simple type systems in languages such as C to fairly complex and expressive type systems such as those in Haskell and Agda.

A type systems is a specification of a given programming language's type rules\cite{cardelli1996type}.
So from this we know that a type system is a specification and we can say that a type checker is an implementation of the specification.
Cardelli goes on to say that, we should separate the two, and that a type system is part of the language specification and the type checker is part of the compiler's implementation of the language\cite{cardelli1996type}.
This is similar to the how we view the formal grammar is defined by the language, and the parser is how the compiler implements that formal grammar\cite{cardelli1996type}.

\subsection{Usefulness of Type Systems}
\todo{Need to research}


\subsection{Type Rules}

Much like how we have a formal representation for context free grammars as Backus-Naur form\cite{Backus1960,aho2003compilers,ranta2012implementing}, we also have a formal way of representing type systems and type rules.
In the literature, the standard method we have of representing type systems and type rules is through the use of natural deduction rules\cite{cardelli1996type,ranta2012implementing}.

Natural deduction rules have a form of: given some premises we conclude the consequent\cite{prawitz2006natural,ranta2012implementing}, for example in \autoref{fig:generalNatDectRule} the premises are denoted as $J_k$ and the consequent is denoted as $C$.

\begin{figure}[htbp]
    \begin{prooftree}
        \LeftLabel{Rule Label:\quad}
        \RightLabel{$[Side\ Conditions]$}
        \AxiomC{$J_1\ J_2\ \ ..\ \ J_n$}
        \UnaryInfC{$C$}
    \end{prooftree}
    \caption{General form of a natural deduction rule}
    \label{fig:generalNatDectRule}
\end{figure}

We can then use these Natural Deduction rules to specify the type rules in a given type system\cite{ranta2012implementing,cardelli1996type,}.
\autoref{fig:simpleTypeRule} is an example of how we would write a simple type rule for an arithmetic expression using the natural deduction logic.
It is important to break this rule into its constituent parts to understand what it is trying to specify.
Starting from the consequent $\Gamma \vdash e_1 + e_2 : Int$, $\Gamma$ is the current context which we are applying the rule to.
The $\vdash$ is piece of syntax to say given the left hand side we can prove the right hand side.
Finally for the consequent $e_1 + e_2 : Int$ means that the piece of concrete syntax $e_1 + e_2$ must have type $Int$.
The top half of the rule are the things that must hold for the bottom half to be valid.\
So simply $e_1$ must be of type $Int$ given the context $\Gamma$, and the same is for the second premise\cite{cardelli1996type,ranta2012implementing}.

\begin{figure}[htbp]
    \begin{prooftree}
        \LeftLabel{Add Expression:\quad}
        \AxiomC{$\Gamma \vdash e_1 : Int$}
        \AxiomC{$\Gamma \vdash e_2 : Int$}
        \BinaryInfC{$\Gamma \vdash e_1 + e_2 : Int$}
    \end{prooftree}
    \caption{Simple arithmetic expression type rule}
    \label{fig:simpleTypeRule}
\end{figure}
%\input{chapters/stateOfTheArt/TypeCheckers.tex}
\section{Type Checker Generators}
There have been multiple attempts to make the writing of type checkers easier on the compiler writers by creating new tools or notations.
They can vary widely in how the express type systems and what notations they use.
Some use the notation of attribute grammars to represent type systems, along with the rest of the compiler, where as other use more specialised Domain Specific Languages to represent their type rules and type checkers.

\subsection{Attribute Grammars}
Attribute Grammars are a way of specifying the semantics of grammar rules for a Context Free Grammar.
This makes them seem a likely candidate to look at when attempting to generate type checking code since type checking is a form of semantic analysis on some input code.
In Knuth's paper\cite{Knuth1968}, he gives an example of the attributes required to evaluate a simple language to emulate a Turing Machine.
Even though he discusses the evaluation of a turing machine it is plain to see that on each node of the tree an attribute could be given to it that defines the type of the node.

\subsection{Eli}
\begin{figure}[]
    \centering
    \begin{prooftree}
        \AxiomC{$\Gamma \vdash e_1 : \tau$}
        \AxiomC{$\Gamma \vdash e_2 : \tau$}
        \RightLabel{$[\tau \in \{Bool, Int\}]$}
        \BinaryInfC{$\Gamma \vdash e_1 = e_2 : Bool$}
    \end{prooftree}
    \caption{Type rule for equality for the example language for Eli}
    \label{fig:oilNatRule}
\end{figure}
Eli is a tool which, given a set of specifications, will generate code for a whole compiler based upon those specifications\cite{Gray:1992:ECF:129630.129637}.
The way Eli generates all the code is by the use of an attribute grammar system and numerous internal tools that generate code for each different part of the compiler pipeline.
The internal tools use DSL's which define a specification for part of the language.
One of these DSL's is called \textit{Operator Identification Language} (OIL), which is used to define the typing relations on operators.
The definition of these type rules is fairly simple, however in their example there are a lot of repeated definitions for operators, for example there are multiple definitions to define the equality operator, one for each of the required types that the operator can act on.
Where as in the natural deduction rules this would only have to be defined once where the type is some arbitrary type $\tau$  with some constraint on $\tau$ such that it can only be from a set of types that an equality check can be performed on, see the natural deduction rule given in \autoref{fig:oilNatRule}.
The tools lacks built-in support for constructs in the language that can have arbitrary types.
However, OIL is a small and simple language that is easy to understand, but, the other constituent parts of ELI required to build a tool are not as simple and the specifications written in them can get much more complicated for larger projects.

\subsection{Ruler}
Ruler is a DSL that bears close resembles to natural deduction rules that formally define type systems\cite{dijkstra2006ruler}.
The notation used to describe the natural deduction rules is intuitive to anyone who has experience with such notation.
However, other possible compiler writers may have never come across the notation meaning the syntax of Ruler could be un-intuitive to such compiler writers.
The rules can be compiled to either \LaTeX, or to an attribute grammar system in Haskell.
However this means that rules have to be written multiple times for each target that the user may want to output to.
This creates unnecessary duplication of types rules when so that they can be compiled to executable code or to a formal specification given in latex.
Since the only executable code compiles to an attribute grammar, it forces the user to use the attribute grammar for the whole frontend of their compiler, where as using the abstract syntax allows the user to choose any parser generator (or write their own parser) as long they provide the abstract syntax to the tool.
They also make a claim that they support Hindley-Milner type inference, while this to some extent is true, it requires the user to write a rather large amount of Haskell code in order for the tool to generate correct code for Hindley-Milner type inference.
This means that a large part of the type checker still has to be written by hand because many non-trivial type rules will require access to the context or types to be unified for which code is not generated for.

\subsection{Typical}
\label{sec:typical}
The final tool we will be discussing is Typical\cite{grimm2007typical}.
Typical is a language that uses ML\cite{milner1997definition} as the base of its language this means that all terms in Typical are typed and such as a proof of concept of Typical, the type checker for Typical is built using Typical.
Typical also has some extra declarations that are not in ML, these are the ability to define name-spaces for the context, and scoping rules, which in the tools we have discusses so far the users have had to write such features of their type system.
They also have in built functions that retrieve types from the context and also add new items to a context, which is again features that the previous tools did not have.
Unlike our previous examples that use attribute grammars to perform their type checking, Typical uses the abstract syntax tree given by the parser.
This means that Typical can be used with any parser generator that targets Typical's target language (Java), although they do admit that some modification may be required.
They have performed the required modification on the Rats! parser generator\cite{Grimm:2006:BET:1133255.1133987}.
To understand a type system written in Typical you will need to be familiar with ML or another language similar to is such as Haskell since the vast majority of the code in Typical is ML.

\chapter{Method}
I decided that I would design and implement a Domain Specific Language to specify type systems, such that the language is analogous to the natural deduction rules used to define type systems in the mathematics.
This appeared to make the most sense as other parts of the compiler pipeline already have domain specific languages which are used to generate code.

The domain specific language, rather than using the concrete syntax that is used in the natural deduction rules, uses the abstract syntax provided by the user to define the type rules.
Type rules in the language have a similar form to the natural deduction rules, such that there would be a rule name provided followed by some judgements and then the consequent.
\chapter{Evaluation}
Given all of the features and type rule patterns we have discussed in \autoref{chap:Method} we can successfully type check both of our witness languages.
As per the aims given in \todo{autoref to aims}, we have discussed how to type check all but user defined types.
The full type specification in JET along with the natural deduction rules that we are implementing in our JET specification can also be seen in \autoref{appendix:witnessLanguages}.
While we have not discussed how to type check user defined types and record types, the JET specification for the Ad language does support these language features and does successfully type check them.
We did not discuss these type rules because there are no other features in the language that allow for a better representation of user defined types and record types than the representation that can be seen in the JET specification of the Ad language.

\section{Language Structure and Extensions}
Support for context function directly in the grammar i.e x in y : t should generate t <- lookupContext x y
Also built in support for namespaces, generalise contexts a bit.

Language support for user defined types for user defined types.

Unify algorithm and type reconstruction support and notation on how to specify how to unify types, will need to cite hindly milner type inference.

\section{Generated Code Structure}
Inefficient code generated in terms of monad statement order.

Inefficient ctx parameter position.

Generate quick check code for type classes to make sure they are implementing the semantics correct.

\section{Operational Semantics and Big Step Notation}
Long term extensions are to add ability to specify operation semantics.

\printbibliography

\appendix
\chapter{N Language}
The N language is a test case language to express the features that the tool can generate code for.
It is a simple imperative language.
\section{Grammar}

\section{LBNF Grammar}
\section{AST Types}
\section{Type Rules}

\end{document}